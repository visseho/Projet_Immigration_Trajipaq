% Options for packages loaded elsewhere
\PassOptionsToPackage{unicode}{hyperref}
\PassOptionsToPackage{hyphens}{url}
%
\documentclass[
]{article}
\usepackage{amsmath,amssymb}
\usepackage{iftex}
\ifPDFTeX
  \usepackage[T1]{fontenc}
  \usepackage[utf8]{inputenc}
  \usepackage{textcomp} % provide euro and other symbols
\else % if luatex or xetex
  \usepackage{unicode-math} % this also loads fontspec
  \defaultfontfeatures{Scale=MatchLowercase}
  \defaultfontfeatures[\rmfamily]{Ligatures=TeX,Scale=1}
\fi
\usepackage{lmodern}
\ifPDFTeX\else
  % xetex/luatex font selection
\fi
% Use upquote if available, for straight quotes in verbatim environments
\IfFileExists{upquote.sty}{\usepackage{upquote}}{}
\IfFileExists{microtype.sty}{% use microtype if available
  \usepackage[]{microtype}
  \UseMicrotypeSet[protrusion]{basicmath} % disable protrusion for tt fonts
}{}
\makeatletter
\@ifundefined{KOMAClassName}{% if non-KOMA class
  \IfFileExists{parskip.sty}{%
    \usepackage{parskip}
  }{% else
    \setlength{\parindent}{0pt}
    \setlength{\parskip}{6pt plus 2pt minus 1pt}}
}{% if KOMA class
  \KOMAoptions{parskip=half}}
\makeatother
\usepackage{xcolor}
\usepackage{color}
\usepackage{fancyvrb}
\newcommand{\VerbBar}{|}
\newcommand{\VERB}{\Verb[commandchars=\\\{\}]}
\DefineVerbatimEnvironment{Highlighting}{Verbatim}{commandchars=\\\{\}}
% Add ',fontsize=\small' for more characters per line
\usepackage{framed}
\definecolor{shadecolor}{RGB}{248,248,248}
\newenvironment{Shaded}{\begin{snugshade}}{\end{snugshade}}
\newcommand{\AlertTok}[1]{\textcolor[rgb]{0.94,0.16,0.16}{#1}}
\newcommand{\AnnotationTok}[1]{\textcolor[rgb]{0.56,0.35,0.01}{\textbf{\textit{#1}}}}
\newcommand{\AttributeTok}[1]{\textcolor[rgb]{0.77,0.63,0.00}{#1}}
\newcommand{\BaseNTok}[1]{\textcolor[rgb]{0.00,0.00,0.81}{#1}}
\newcommand{\BuiltInTok}[1]{#1}
\newcommand{\CharTok}[1]{\textcolor[rgb]{0.31,0.60,0.02}{#1}}
\newcommand{\CommentTok}[1]{\textcolor[rgb]{0.56,0.35,0.01}{\textit{#1}}}
\newcommand{\CommentVarTok}[1]{\textcolor[rgb]{0.56,0.35,0.01}{\textbf{\textit{#1}}}}
\newcommand{\ConstantTok}[1]{\textcolor[rgb]{0.00,0.00,0.00}{#1}}
\newcommand{\ControlFlowTok}[1]{\textcolor[rgb]{0.13,0.29,0.53}{\textbf{#1}}}
\newcommand{\DataTypeTok}[1]{\textcolor[rgb]{0.13,0.29,0.53}{#1}}
\newcommand{\DecValTok}[1]{\textcolor[rgb]{0.00,0.00,0.81}{#1}}
\newcommand{\DocumentationTok}[1]{\textcolor[rgb]{0.56,0.35,0.01}{\textbf{\textit{#1}}}}
\newcommand{\ErrorTok}[1]{\textcolor[rgb]{0.64,0.00,0.00}{\textbf{#1}}}
\newcommand{\ExtensionTok}[1]{#1}
\newcommand{\FloatTok}[1]{\textcolor[rgb]{0.00,0.00,0.81}{#1}}
\newcommand{\FunctionTok}[1]{\textcolor[rgb]{0.00,0.00,0.00}{#1}}
\newcommand{\ImportTok}[1]{#1}
\newcommand{\InformationTok}[1]{\textcolor[rgb]{0.56,0.35,0.01}{\textbf{\textit{#1}}}}
\newcommand{\KeywordTok}[1]{\textcolor[rgb]{0.13,0.29,0.53}{\textbf{#1}}}
\newcommand{\NormalTok}[1]{#1}
\newcommand{\OperatorTok}[1]{\textcolor[rgb]{0.81,0.36,0.00}{\textbf{#1}}}
\newcommand{\OtherTok}[1]{\textcolor[rgb]{0.56,0.35,0.01}{#1}}
\newcommand{\PreprocessorTok}[1]{\textcolor[rgb]{0.56,0.35,0.01}{\textit{#1}}}
\newcommand{\RegionMarkerTok}[1]{#1}
\newcommand{\SpecialCharTok}[1]{\textcolor[rgb]{0.00,0.00,0.00}{#1}}
\newcommand{\SpecialStringTok}[1]{\textcolor[rgb]{0.31,0.60,0.02}{#1}}
\newcommand{\StringTok}[1]{\textcolor[rgb]{0.31,0.60,0.02}{#1}}
\newcommand{\VariableTok}[1]{\textcolor[rgb]{0.00,0.00,0.00}{#1}}
\newcommand{\VerbatimStringTok}[1]{\textcolor[rgb]{0.31,0.60,0.02}{#1}}
\newcommand{\WarningTok}[1]{\textcolor[rgb]{0.56,0.35,0.01}{\textbf{\textit{#1}}}}
\usepackage{longtable,booktabs,array}
\usepackage{calc} % for calculating minipage widths
% Correct order of tables after \paragraph or \subparagraph
\usepackage{etoolbox}
\makeatletter
\patchcmd\longtable{\par}{\if@noskipsec\mbox{}\fi\par}{}{}
\makeatother
% Allow footnotes in longtable head/foot
\IfFileExists{footnotehyper.sty}{\usepackage{footnotehyper}}{\usepackage{footnote}}
\makesavenoteenv{longtable}
\usepackage{graphicx}
\makeatletter
\def\maxwidth{\ifdim\Gin@nat@width>\linewidth\linewidth\else\Gin@nat@width\fi}
\def\maxheight{\ifdim\Gin@nat@height>\textheight\textheight\else\Gin@nat@height\fi}
\makeatother
% Scale images if necessary, so that they will not overflow the page
% margins by default, and it is still possible to overwrite the defaults
% using explicit options in \includegraphics[width, height, ...]{}
\setkeys{Gin}{width=\maxwidth,height=\maxheight,keepaspectratio}
% Set default figure placement to htbp
\makeatletter
\def\fps@figure{htbp}
\makeatother
\setlength{\emergencystretch}{3em} % prevent overfull lines
\providecommand{\tightlist}{%
  \setlength{\itemsep}{0pt}\setlength{\parskip}{0pt}}
\setcounter{secnumdepth}{5}
\usepackage{booktabs}
\usepackage{fancyhdr}
\pagestyle{fancy}
\fancyhf{}
\renewcommand{\headrulewidth}{0pt}
\fancyfoot[R]{\thepage}
\ifLuaTeX
  \usepackage{selnolig}  % disable illegal ligatures
\fi
\IfFileExists{bookmark.sty}{\usepackage{bookmark}}{\usepackage{hyperref}}
\IfFileExists{xurl.sty}{\usepackage{xurl}}{} % add URL line breaks if available
\urlstyle{same}
\hypersetup{
  pdftitle={COVID-19 Disparities in Quebec (Canada)},
  pdfauthor={V. Adjiwanou, G. NGalé, S. Lardoux, N. Zahedinameghi},
  hidelinks,
  pdfcreator={LaTeX via pandoc}}


%%%%%%%%%%%%%%%%
% FRANCISATION DES TERMES
%%%%%%%%%%%%%%%%

\renewcommand\contentsname{Table des mati\`eres}
\renewcommand\listfigurename{Liste des figure}
\renewcommand\listtablename{Liste des tableaux}
\renewcommand\indexname{Index}
\renewcommand\figurename{Figure}
\renewcommand\tablename{Tableau}
\renewcommand\partname{Partie}
\renewcommand\appendixname{Annexe}  % annexe
\renewcommand\abstractname{R\'esum\'e}


%%%%%%%%%%%%%%%%
% PAGE TITRE
%%%%%%%%%%%%%%%%



\def\maketitle{\begin{titlepage}
\setmainfont{Liberation Sans} 
{\raggedleft\sffamily\color{black}
Vincent Bertrand-Huneaut
\par}

{\raggedleft\sffamily\color{black}
Jeanne Craig-Larouche
\par}

{\raggedleft\sffamily\color{black}
Théa Carlier
\par}

{\raggedleft\sffamily\color{black}
Noura Kone
\par}

{\raggedleft\sffamily\color{black}
\ \ Antoine Martin
\par}


\bigskip


\bigskip


\bigskip


\bigskip


\bigskip

{\centering\sffamily\color{black}
Article
\par}

{\centering\sffamily\color{black}
COVID-19 Disparities in Quebec (Canada)
\par}


\bigskip


\bigskip

{\centering\sffamily\color{black}
Présenté à :
\par}

{\centering\sffamily\color{black}
Vissého Adjiwanou
\par}


\bigskip


\bigskip


\bigskip


\bigskip


\bigskip


\bigskip


\bigskip


{\centering\sffamily\color{black}

\par}


\bigskip


\bigskip


\bigskip


\bigskip


\bigskip


\bigskip


\bigskip


\bigskip


\bigskip

{\centering\sffamily\color{black}
mai 2, 2023
\par}


\bigskip


\bigskip


\bigskip


\bigskip


\bigskip


{\centering\sffamily\color{black}
Université du Québec À Montréal and Université De Montréal
\par}
\end{titlepage}
}




\begin{document}
\maketitle
\newpage
\begin{abstract}
The immigrant population is one of the most vulnerable in Canada, with a relatively high risk of unemployment, a low social support system, and linguistic barriers. However, their experiences with public health interventions during the COVID-19 pandemic have not been adequately explored. The purpose of this article is to analyze Covid-19 exposure risk and its family implications for both natives and immigrants in the province of Quebec. Additionally, we analyze immigrants' access to government aid and compare it to their native-born counterparts. The study consists of survey data collected from 1,000 immigrants and 500 non-immigrants Quebecers between August and November 2020. Our descriptive and exploratory analysis reveals significant differences between the two groups regarding exposure risk, family consequences, and access to resources. We conclude that the current pandemic exacerbates health disparities in the province, with particularly dire repercussions for immigrants' families. We also discuss several socio-cultural strategies to mitigate inequality and recommend policies to address the needs of vulnerable populations.
\end{abstract}

\newpage
{
\setcounter{tocdepth}{2}
\tableofcontents\
\newpage
}
\begin{Shaded}
\begin{Highlighting}[]
\CommentTok{\# Importer package}
\FunctionTok{library}\NormalTok{(dplyr)}
\end{Highlighting}
\end{Shaded}

\begin{verbatim}
## Warning: le package 'dplyr' a été compilé avec la version R 4.2.3
\end{verbatim}

\begin{verbatim}
## 
## Attachement du package : 'dplyr'
\end{verbatim}

\begin{verbatim}
## Les objets suivants sont masqués depuis 'package:stats':
## 
##     filter, lag
\end{verbatim}

\begin{verbatim}
## Les objets suivants sont masqués depuis 'package:base':
## 
##     intersect, setdiff, setequal, union
\end{verbatim}

\begin{Shaded}
\begin{Highlighting}[]
\FunctionTok{library}\NormalTok{(tidyverse)}
\end{Highlighting}
\end{Shaded}

\begin{verbatim}
## Warning: le package 'tidyverse' a été compilé avec la version R 4.2.3
\end{verbatim}

\begin{verbatim}
## Warning: le package 'ggplot2' a été compilé avec la version R 4.2.3
\end{verbatim}

\begin{verbatim}
## Warning: le package 'tibble' a été compilé avec la version R 4.2.3
\end{verbatim}

\begin{verbatim}
## Warning: le package 'tidyr' a été compilé avec la version R 4.2.3
\end{verbatim}

\begin{verbatim}
## Warning: le package 'readr' a été compilé avec la version R 4.2.3
\end{verbatim}

\begin{verbatim}
## Warning: le package 'forcats' a été compilé avec la version R 4.2.3
\end{verbatim}

\begin{verbatim}
## Warning: le package 'lubridate' a été compilé avec la version R 4.2.3
\end{verbatim}

\begin{verbatim}
## -- Attaching core tidyverse packages ------------------------ tidyverse 2.0.0 --
## v forcats   1.0.0     v readr     2.1.4
## v ggplot2   3.4.2     v stringr   1.5.0
## v lubridate 1.9.2     v tibble    3.2.1
## v purrr     1.0.1     v tidyr     1.3.0
\end{verbatim}

\begin{verbatim}
## -- Conflicts ------------------------------------------ tidyverse_conflicts() --
## x dplyr::filter() masks stats::filter()
## x dplyr::lag()    masks stats::lag()
## i Use the ]8;;http://conflicted.r-lib.org/conflicted package]8;; to force all conflicts to become errors
\end{verbatim}

\begin{Shaded}
\begin{Highlighting}[]
\FunctionTok{library}\NormalTok{(summarytools)}
\end{Highlighting}
\end{Shaded}

\begin{verbatim}
## Warning: le package 'summarytools' a été compilé avec la version R 4.2.3
\end{verbatim}

\begin{verbatim}
## 
## Attachement du package : 'summarytools'
## 
## L'objet suivant est masqué depuis 'package:tibble':
## 
##     view
\end{verbatim}

\begin{Shaded}
\begin{Highlighting}[]
\FunctionTok{library}\NormalTok{(readr)}
\FunctionTok{library}\NormalTok{(questionr)}
\end{Highlighting}
\end{Shaded}

\begin{verbatim}
## 
## Attachement du package : 'questionr'
## 
## L'objet suivant est masqué depuis 'package:summarytools':
## 
##     freq
\end{verbatim}

\begin{Shaded}
\begin{Highlighting}[]
\FunctionTok{library}\NormalTok{(rcompanion)}
\FunctionTok{library}\NormalTok{(pander)}
\end{Highlighting}
\end{Shaded}

\begin{Shaded}
\begin{Highlighting}[]
\CommentTok{\# Set numero graphique}
\NormalTok{no\_graphic }\OtherTok{\textless{}{-}} \SpecialCharTok{{-}}\DecValTok{1}
\end{Highlighting}
\end{Shaded}

\hypertarget{introduction-background}{%
\section{Introduction / background}\label{introduction-background}}

The COVID-19 pandemic has highlighted the vast ethnic, social, economic, and
health disparities in high-income countries, including Canada. One of the most
vulnerable segments of these affluent societies is the immigrant population,
that has become widely diversified with more and more people coming from
developing countries, and for whom integration into their new society may be
more difficult. The coronavirus pandemic brings a new challenge for these
immigrant populations, directly or indirectly through the tensions between the
ethnic groups that it helps to generate (\protect\hyperlink{ref-coates2020Covid19}{\textbf{coates2020Covid19?}}). These immigrants
face language barriers, low social capital, housing insecurity, and severe
poverty, rendering them particularly susceptible to the effects of the current
pandemic. Immigrants are also more likely to be employed as essential workers
where social distancing, sanitary precautions, and self-isolation, measures put
in place to mitigate the pandemy, are challenging (\protect\hyperlink{ref-reid2021Migrant}{\textbf{reid2021Migrant?}}~; \protect\hyperlink{ref-yi2021Health}{\textbf{yi2021Health?}}~; \protect\hyperlink{ref-zhang2021COVID19}{\textbf{zhang2021COVID19?}}). More importantly, migrants may encounter
various forms of discrimination, leading to social exclusion and restricted
access to healthcare and social services (\protect\hyperlink{ref-spiritus-beerden2021Mental}{\textbf{spiritus-beerden2021Mental?}}). It is
therefore necessary to understand how this pandemic affects these immigrant
populations differently and the native populations in Quebec. In doing so, this
article will reveal a better understanding of how the current pandemic
aggravates existing inequalities and challenges in the province for immigrants
and their descendants.

Since the start of the pandemic, most of the confirmed cases were identified in
Montreal and areas with high immigrant populations (\protect\hyperlink{ref-miconi2021Ethnocultural}{\textbf{miconi2021Ethnocultural?}}~; \protect\hyperlink{ref-shields2020COVID19}{\textbf{shields2020COVID19?}}). The coronavirus pandemic has infected nearly 450,000
people in Quebec and has resulted in more than 11,000 deaths
(\url{https://www.quebec.ca/sante/problemes-de-sante/a-z/coronavirus-2019/situation-coronavirus-quebec},
consulted on octobre 29, 2021). This makes it the worst affected province in
Canada and one of the worst affected regions in the world during the first wave
of the pandemic. Although the majority of these deaths took place in elderly's
homes, and therefore affected more this segment of the population, its overall
influence on individuals is not to be overlooked. It has been shown that, on the
one hand, immigrants and their families have a higher risk of being exposed to
the pathogen, and, at the same time, they are more vulnerable to racism and
discrimination and face barriers to accessing social support (\protect\hyperlink{ref-2021CRARR}{\textbf{2021CRARR?}}) and
information, especially for those who didn't speak or understand the two
official langages.
Several
studies have thus shown excess mortality in ethnocultural minority populations
(\protect\hyperlink{ref-bertocchi2020Covid19}{\textbf{bertocchi2020Covid19?}}), higher unemployment, greater loss of income and
negative consequences for children and families in several countries. This
differentiated effect can be explained in different ways. On the one hand, in
terms of the direct consequences of the pandemic, the situation of
precariousness and greater poverty among immigrants exposes them more to the
more harmful consequences of the disease on their state of health. Likewise,
immigrants often work in jobs where social distancing is difficult to enforce.
On the other hand, indirectly, the solutions to the crisis affect immigrants and
natives differently. The complete shutdown of tye economy with its set of
adjustments has greatly impeded the employability of immigrants, especially
newcomers who have not yet grasped the workings of society and who cannot
benefit from state subsidies. But, it is also shown that these effects are not
uniform on the whole of the immigrant population, itself, very diverse.

Despite the growing focus on the differential impacts of the COVID-19 on
immigrants and non-immigrants segment of the population, there is still very
little knowledge about how the pandemic and its related public policy impact the
lives of immigrants and their families. The existing health datasets (largely
used to assess this effect) are not necessarily representative of the immigrant
population and do not offer an accurate picture of the complex social status and
health outcomes among this group (\protect\hyperlink{ref-choi2021Studying}{\textbf{choi2021Studying?}}~; \protect\hyperlink{ref-gagnon2021Immigration}{\textbf{gagnon2021Immigration?}}~; \protect\hyperlink{ref-khan2015Visible}{\textbf{khan2015Visible?}}). In most cases, these datasets lack data on the residence
length of immigrants and combine newcomers with long-term settlers
(\protect\hyperlink{ref-edge2013Discrimination}{\textbf{edge2013Discrimination?}}~; \protect\hyperlink{ref-hayward2021Clinical}{\textbf{hayward2021Clinical?}}). As a result, they may not allow
a detailed assessment of the nature and extent of risk exposure in sub-segments
of populations, as well as whether interventions are reaching those most at
risk. In addition, the family-level impacts of the pandemic are still poorly
understood. While a body of research has explored the association between the
pandemic and its potential effects on intimate partner violence and marital
dissolution (\protect\hyperlink{ref-bhatnagar2021Navigating}{\textbf{bhatnagar2021Navigating?}}~; \protect\hyperlink{ref-koshan2020COVID19}{\textbf{koshan2020COVID19?}}~; \protect\hyperlink{ref-morrison2021Chapter}{\textbf{morrison2021Chapter?}}~; \protect\hyperlink{ref-prime2020Risk}{\textbf{prime2020Risk?}}), few studies have examined negative familial repercussions in
the context of immigration status. There is ample evidence that sustained family
relationships greatly enhance immigrants' lives on several fronts, including
fostering economic prosperity (\protect\hyperlink{ref-marger2001Sociala}{\textbf{marger2001Sociala?}}), psychological well-being
(\protect\hyperlink{ref-beiser2002Povertya}{\textbf{beiser2002Povertya?}}), social network development (\protect\hyperlink{ref-evansluong2017Familya}{\textbf{evansluong2017Familya?}}), and
educational success for children (\protect\hyperlink{ref-munroe-blum1989Immigranta}{\textbf{munroe-blum1989Immigranta?}}). The family is,
therefore, a key component of long-term social mobility and economic success of
immigrants.

Based on an original biographic survey collected around the first wave of the
pandemic in august-november 2020 on immigrants and the native population of
Québec, this study mainly aims to understand the family exposure, consequences,
and responses of the pandemic on immigrant and native Quebec populations. We
assess a whole range of different measures of each outcome. We divided the
consequences in three dimensions, 1) access to ressources, 2) ability to perform
duties and 3) employment and income. We seek also to analyse in what ways these
outcomes differ according to the immigrants length of residence. By doing so,
this study provide a clear and global picture of the situation of different
population groups in Quebec in the face of the COVID-19.

\newpage

\hypertarget{current-body-of-literature-and-knowledge-gaps}{%
\section{Current body of literature and knowledge gaps}\label{current-body-of-literature-and-knowledge-gaps}}

The literature on the racial and ethnic health disparities during the COVID-19
crisis can be divided into two main categories: studies addressing the
socio-economic determinants of health and those examining the delivery of health
interventions and policies. The first category is primarily comprised of
research that focuses on the link between socio-economic status (SES) of
high-risk minority groups and the differential impact of the pandemic
(\protect\hyperlink{ref-drefahl2020Populationbased}{\textbf{drefahl2020Populationbased?}}~; \protect\hyperlink{ref-greenaway2020COVID19}{\textbf{greenaway2020COVID19?}}~; \protect\hyperlink{ref-lopez2021Racial}{\textbf{lopez2021Racial?}}~; \protect\hyperlink{ref-pacheco2020Job}{\textbf{pacheco2020Job?}}~; \protect\hyperlink{ref-sundaram2020Individual}{\textbf{sundaram2020Individual?}}). For instance, an inter-country study
has found that Middle Eastern, African, and Nordic immigrants in Sweden have a
higher risk of being hospitalized and even dying from COVID-19 when compared to
the Swedish born population. In the same study, the SES, proportion of
working-age households, and neighborhood density were significantly associated
with the mortality rate (\protect\hyperlink{ref-rostila2021Disparities}{\textbf{rostila2021Disparities?}}). This trend of research
demonstrates that racial and ethnic minorities are more likely to experience
stress factors, not only due to the hazards of infection but also because of
economic and social marginalization (\protect\hyperlink{ref-alahmad2020COVID19}{\textbf{alahmad2020COVID19?}}~; \protect\hyperlink{ref-choi2020Buffering}{\textbf{choi2020Buffering?}}~; \protect\hyperlink{ref-endale2020COVID19}{\textbf{endale2020COVID19?}}~; \protect\hyperlink{ref-fitzpatrick2020Fear}{\textbf{fitzpatrick2020Fear?}}~; \protect\hyperlink{ref-serafini2021Psychological}{\textbf{serafini2021Psychological?}}). The
intersectionality of race, sex, and health conditions has also been explored in
the Canadian context of the pandemic. (\protect\hyperlink{ref-ingen2021Neighbourhoodlevel}{\textbf{ingen2021Neighbourhoodlevel?}}) demonstrated
that Ontarian neighborhoods with immigrant and racialized populations have a
disproportionate share of COVID-19 infections. In addition, areas with high SES
report a lesser number of COVID cases than low SES ones
(\protect\hyperlink{ref-ingen2021Neighbourhoodlevel}{\textbf{ingen2021Neighbourhoodlevel?}}). Another study conducted in Quebec confirms the
strong association between immigrants' backgrounds, SES, and COVID-19 related
psychological stress (\protect\hyperlink{ref-miconi2021Ethnocultural}{\textbf{miconi2021Ethnocultural?}}). The most recent reviews on the
topic suggest that immigrants disproportionately live in crowded and inadequate
housing and are employed in essential industries, leaving them with severe
income insecurities and poor physiological/psychological well-being
(\protect\hyperlink{ref-chew2020Clinical}{\textbf{chew2020Clinical?}}~; \protect\hyperlink{ref-davis2020Impact}{\textbf{davis2020Impact?}}). Studies on SES and health disparities
have emphasized that this situation is further complicated for immigrant women,
constrained by gender and afflicted by a significant power imbalance in the
household and workplace (\protect\hyperlink{ref-alarcao2021Intersections}{\textbf{alarcao2021Intersections?}}~; \protect\hyperlink{ref-esegbona-adeigbe2020COVID19}{\textbf{esegbona-adeigbe2020COVID19?}}~; \protect\hyperlink{ref-jehn2021COVID19}{\textbf{jehn2021COVID19?}}~; \protect\hyperlink{ref-mo2020Differential}{\textbf{mo2020Differential?}}~; \protect\hyperlink{ref-nardon2021Skilled}{\textbf{nardon2021Skilled?}}~; \protect\hyperlink{ref-sabri2020Effect}{\textbf{sabri2020Effect?}}~; \protect\hyperlink{ref-st-denis2020Sociodemographic}{\textbf{st-denis2020Sociodemographic?}}). According
to Lightman, immigrants and women of color are overrepresented in long-term care
facilities in Canada, where COVID infection rates have been exceptionally high.
They are also subject to exclusionary social and economic practices
(\protect\hyperlink{ref-lightman2021Caring}{\textbf{lightman2021Caring?}}).

The second category of the literature is primarily comprised of policy analyses,
research on public health communications, examining the efficacy of COVID
initiatives, the optimal allocation of government aids, and ethical issues
associated with applying new health interventions to vulnerable populations and
their communities (\protect\hyperlink{ref-aragona2020Negative}{\textbf{aragona2020Negative?}}~; \protect\hyperlink{ref-cross2021Coronavirus}{\textbf{cross2021Coronavirus?}}~; \protect\hyperlink{ref-desai2020COVID19}{\textbf{desai2020COVID19?}}~; \protect\hyperlink{ref-doyle2020Migrant}{\textbf{doyle2020Migrant?}}~; \protect\hyperlink{ref-gilman2020Modelling}{\textbf{gilman2020Modelling?}}~; \protect\hyperlink{ref-hargreaves2020Targeting}{\textbf{hargreaves2020Targeting?}}~; \protect\hyperlink{ref-langellier2020Policy}{\textbf{langellier2020Policy?}}~; \protect\hyperlink{ref-nezafatmaldonado2020Engaging}{\textbf{nezafatmaldonado2020Engaging?}}~; \protect\hyperlink{ref-silverman2020Ethics}{\textbf{silverman2020Ethics?}}~; \protect\hyperlink{ref-wilson2020Ethics}{\textbf{wilson2020Ethics?}}). Some of the questions that this body
of research addresses include: how best to respond to COVID-19 in immigrant
communities? How do public health policies impact the socio-economic status and
psychological well-being of different sub-populations? What is the structure of
the planning and implementation of government aid programs during the current
crisis? And, how can we design public health communication strategies in a way
that is culturally sensitive and inclusive?

There are several knowledge gaps in the existing literature. Most of the cited
studies do not include household data and family structure of immigrants as
variables that affect exposure risks and health outcomes of COVID-19.
Consequently, the relationship between immigrant status and COVID-19 is not
accounted for by household size, and family characteristics (i.e., income,
education, religion). A key reason is that most of the health data used in such
research projects are acquired from national registers or administrative
databases, which provide little information on immigration status and kinship.
However, as the study of Thomeer and her colleagues theorizes, the extent of
adverse social and economic impacts of COVID-19 is significantly conditional on
the composition of an immigrant's family (\protect\hyperlink{ref-thomeer2020How}{\textbf{thomeer2020How?}}). Households that
include a spouse/partner, children, or multigenerational family members most
likely experience a different level of social, psychological, and medical
vulnerability than single or nuclear ones. Therefore, studies of immigrant
populations should account for the breakdown of family configuration to better
understand the mechanisms linking social determinants and health outcomes.
Another knowledge gap concerns the current state of Canadian studies of COVID-19
and its public health crisis. There is a dearth of literature on the challenges
faced by immigrants and their families in the country. Although the burden of
COVID-19 infection is disproportionately higher among these marginalized groups
(St-Denis, 2020; Sundaram et al., 2020), research on the issues they are
experiencing is sparse and often limited to one aspect.

\newpage

\hypertarget{theoretical-framework}{%
\section{Theoretical framework}\label{theoretical-framework}}

Family as a unit of analysis can be particularly fruitful when studying health
inequalities in the context of a societal-wide crisis. Through the lens of the
COVID-19 pandemic, the following points illustrate why exploring family dynamics
provides additional insight into the underlying mechanisms of health
disparities. First, the family is the most fundamental unit of social
organization. It facilitates access to resources and the transmission of social
and psychological capital from one generation to the next. Therefore, a better
analysis of how crises and adverse life events affect people requires a better
understanding of how families relate to each other and interact with external
forces. Second, during public health emergencies, families may be thrown into
different structures due to unprecedented financial, mental, and physical
challenges. The risk of family dissolution, for example, can be heightened when
members must spend unusually long periods together at home and have limited
opportunities to engage with friends and the wider community. Conversely,
families can develop closer bonds through self-sacrifice, shared
responsibilities, and increased care and support. As the dependency status of
seniors increases, the elderlies are also more likely to be integrated into
formerly nucleus families. We posit that the extent to which a family can adapt
its functions to meet new challenges is a predictor of the well-being of its
members and the resilience of its broader population. Third, the family is a
crucial link to social institutions, such as the health care system, schools,
and government agencies whose decisions have significant downstream impacts.
Families are viewed by many public-level entities as a reference when allocating
resources, enforcing new standards, and responding to emerging needs and
demands. From this perspective, the family operates at the nexus of individual
and societal structures, embedding the tripple effects of public policy in its
most consequential ways. Finally, the family has a crucial role to play in
determining the life course of its members. As COVID-19 elevates the importance
of familial bonds, the post-pandemic period might be marked by an increased
influence of families on the lives of their members and the social structures
they inhabit.

The above conceptualization echoes several other approaches to studying
long-term health inequalities. For instance, life course theory emphasizes the
significance of family ties and social environment as they structure prospects
for individuals across their life span (Elder, 1977; Elder Jr., 1998; Macmillan,
2005; Settersten et al., 2020). Likewise, social capital theory and the
stress-buffering hypothesis consider family systems a crucial factor in health
and well-being (Choi and Jun, 2020; Cohen and McKay, 1984; Cohen and Wills,
1985; Harpham et al., 2004; Kawachi and Berkman, 2001; Lavee et al., 1987;
Mandelbaum et al., 2020). They show that the family can offer tangible (e.g.,
food, shelter, and money) and intangible resources (e.g., support, social
status, psychological comfort, and institutional advantages), protecting its
members from adverse life events. Similarly, the social-ecological model and the
theory of fundamental causes highlight how the SES, familial and extrafamilial
factors pattern people's life trajectories and health outcomes (Barkan and
Rocque, 2018; Clouston and Link, 2021; Figge et al., 2018; Kim, 2020; Libman et
al., 2012; Ungar, 2015). However, all these approaches are ostensibly focused on
life periods where the general social context is relatively stable. In this
article, we present our conceptualization of the family in times of
unprecedented social upheaval. Specifically, our framework seeks to characterize
family dynamics as a suitable lens to analyze health inequalities in the face of
a virulent public health crisis.

\newpage

\hypertarget{methodology}{%
\section{Methodology}\label{methodology}}

\hypertarget{data}{%
\subsection{Data}\label{data}}

The data for this study come from an original collection of biographical data on
immigrants and the native population of Quebec to analyse their participation in
Quebec society in various fields (social, economic, cultural, etc.). The
immigrant population is limited to recent immigration that has taken place over
the past 15 years. This survey has several components that retrace the life
course of individuals over a long period of time. The pandemic component uses a
reduced version of the \href{https://www.phenxtoolkit.org/toolkit_content/PDF/CPTS_CEFIS_FullSurveyEng.pdf}{harmonized questionnaire on exposure and the impact of
Covid19 on the family
(CEFIS)}.
The questions are divided into three parts, on exposure to the disease, its
consequences and the benefit of the policies put in place by both the federal
and provincial governments. The other parts of the questionnaire focus on the
characteristics of the respondents, their immigration history, their educational
and family background. In short, it is a detailed questionnaire that covers the
life of immigrants from birth and especially over the past 15 years. The data
collection lasted three months, from August to November 2020. In total, 1,500
people were surveyed, including around 1,000 immigrants and 500 natives
throughout Quebec.

\hypertarget{dependent-variables}{%
\subsubsection{Dependent variables}\label{dependent-variables}}

We distinguish three groups of dependent variables:

\textbf{1. Disease exposure variables}: these are measured by six questions that seek
to document whether the individual through work or a family member may have been
exposed to the disease. For example, it is asked if a family member has had
symptoms of the disease, or if a member has been hospitalized due to the
contraction of the disease.

\textbf{2. Variables on the family consequences of the disease}: we measure three
dimensions of consequences. The first group of measures assess the consequences
on accessing ressources by the respondent or any member of his family. The
second group of measure assess whether the respondent or any family member have
difficulties to perform usual duties toward any member of the family. Finally,
the last group of measures measure the consequences of the pandemic on
employment and income of any family member.

\textbf{3. Variables on government assistance}: Finally, government support is
measured by the following two questions that assessed whether the respondent or
any family member has benefited of the temporary assistance for workers from the
Government of Quebec (PATT) or of the Canadian Federal Government Emergency
Benefit (CEP). The table XXX in the apendix presents the full set of indicators
of the dependents variables

\hypertarget{main-independent-variable}{%
\subsubsection{Main independent variable}\label{main-independent-variable}}

The main independent variable is immigrant status coupled with visible minority
status. This variable takes the following modalities: 1 if the respondent is a
non-racialized native, 2 if he is a non-racialized immigrant, 3 if he is a
racialized immigrant, and 4 if the respondent is a racialized native or an
immigrant who has not answered the visible minority question. The other
variables in the analysis are: age, sex, length of residence, minority status,
size of social network.

\hypertarget{analysis-methods}{%
\subsection{Analysis methods}\label{analysis-methods}}

We use two main methods of analysis, a descriptive method and an explanatory
method. In the descriptive method, we consider each of the independent variables
in isolation and describe how they vary depending on the main independent
variable. The explanatory methods goes further by assessing whether the effect
remain when controlling for other covariates. Thus, by controlling for the other
explanatory variables, we determine the effect of immigrant status on exposure
to the pandemic, on its consequences and on the fact of benefiting or not of
government assistance. In a second series of modelling, we assess whether the
effect that we observed varies by the immigrant's length of residence or by
marital status. We use either a logistic regression on the outcome variables
that are binary and an ordonned logistic regression for the outcome that are
measure with a 5-point Lickert scale.

\newpage

\hypertarget{results}{%
\section{3. Results}\label{results}}

\hypertarget{sample-distribution}{%
\subsection{3.1. Sample distribution}\label{sample-distribution}}

\hypertarget{exposure-by-immigration-status}{%
\subsection{3.2. Exposure by immigration status}\label{exposure-by-immigration-status}}

Figure 1 presents the exposure to the pandemy accroding to the immigrants and
minority visible status. For all the six measures of exposure, it appears that
the majority group is the most exposed to the COVID-19 in all instance. The big
difference between the groups is about the exposure trhough working outside.
Around 50\% of the member of this group or a family member work outside the home
at the time of the survey. This percentage is only 30\% for the racialised
immigrants and 25\% for the non-racialised immigrants.

\begin{figure}
\centering
\includegraphics[width=0.6\textwidth,height=\textheight]{Figures/Exposition1.png}
\caption{Figure 1.1: Exposure}
\end{figure}

\begin{figure}
\centering
\includegraphics{Figures/Exposure.jpeg}
\caption{Figure 1.2: Explicative model}
\end{figure}

\hypertarget{consequences-by-immigrations-status}{%
\subsection{3.3. Consequences by immigrations status}\label{consequences-by-immigrations-status}}

The Figure 2 presents the consequences of the COVID-19 for a family member to
access to ressources. It shows that in all accounts but one (getting medecine),
the racialised immigrants express more diffciult to assess ressources. Around
20\% of them stated having difficulty to get health care compare, around 10\% for
obtaining food and 7\% for obtaining family insurrance. Comparatively, these
proportion are respectively 13\%, 7\% and 4\% for people and thier family member in
the majority group and 19\%, 3\% and 4\% respectively among the racialised
immigrants.

\begin{figure}
\centering
\includegraphics[width=0.8\textwidth,height=\textheight]{Figures/Famille1.png}
\caption{Figure 3.1: Consequences on accessing
ressources}
\end{figure}

\begin{figure}
\centering
\includegraphics{Figures/Ressources.jpeg}
\caption{Figure 3.2: Explicative model}
\end{figure}

\hypertarget{consequences-on-performing-usual-duties}{%
\subsection{3.4. Consequences on performing usual duties}\label{consequences-on-performing-usual-duties}}

The Figure 4 compares the response from the three groups on performing usual
duties. It appears that for all groups the situation caused by the COVID-19 has
inexpectedly improved the situation in various way, irrespective of the
immigrants status of the respondents.

\begin{figure}
\centering
\includegraphics{Figures/Entente1.png}
\caption{Figure 4.1: Consequences on performing usual duties}
\end{figure}

\begin{figure}
\centering
\includegraphics{Figures/Abilities.jpeg}
\caption{Figure 4.2: Explicative model}
\end{figure}

\hypertarget{consequences-on-employment-by-immigration-status}{%
\subsection{3.5. Consequences on employment by immigration status}\label{consequences-on-employment-by-immigration-status}}

The consequences on employment are mixed, with the severe consequences more
prevalent among the racialised immigrants. The Figure 3 shows for instance that
close to 45\% of the racialised immigrants have stated that their family income
has decreased compare to just 33\% of the majority group and 37\% of the
non-racialed immigrants. In addition, 11\% of the racialised immigrants reveals
having loss their job compared to 9\% among the non-racialised immigrants and 2\%
among members of the majority group. At the opposite, the members of the
majority groups are more likely to report a reduction of the working hours or
for stopping momentarily their work.

\begin{figure}
\centering
\includegraphics{Figures/Emploi1.png}
\caption{Figure 5.1: Consequences on employment and income}
\end{figure}

\begin{figure}
\centering
\includegraphics{Figures/Employment.jpeg}
\caption{Figure 5.2: Explicative model}
\end{figure}

\hypertarget{help-received-by-immigrations-status}{%
\subsection{3.7 Help received by immigrations status}\label{help-received-by-immigrations-status}}

Finally, we compared the different group on benefitting of the government
assistance. It appears that racialised immigrants with 35\% of the case have
expressed having received the Canada Emergency Response Benefit wheras this
proportion is 31\% for the non-racialised immigrants and 25\% for the member of
the majority group. For the province Temporary Aid for Workers Program, around
5\% of the respondents have received it.

\includegraphics{Figures/Aide1.png}

\begin{figure}
\centering
\includegraphics{Figures/Government.jpeg}
\caption{Figure 3.2: Explicative model}
\end{figure}

\newpage

\hypertarget{conclusion}{%
\section{Conclusion}\label{conclusion}}

The preliminary results of this study show that the consequences of the pandemic
on the Quebec population differ according to the population groups, but also
according to the type of consequences.

\newpage

\hypertarget{ruxe9fuxe9rences}{%
\section{Références}\label{ruxe9fuxe9rences}}

\url{http://www.oecd.org/coronavirus/policy-responses/what-is-the-impact-of-the-covid-19-pandemic-on-immigrants-and-their-children-e7cbb7de/}

Guttmann A, F. et al.~(2020), COVID-19 in Immigrants, Refugees and Other
Newcomers in Ontario: Characteristics of Those Tested and Those Confirmed
Positive, as of June 13, 2020, ICES, \url{http://www.ices.on.ca}.

Hu, Y. (2020), ``Intersecting ethnic and native--migrant inequalities in the
economic impact of the COVID-19 pandemic in the UK'', Research in Social
Stratification and Mobility, Vol. 68, p.~100528,
\url{http://dx.doi.org/10.1016/j.rssm.2020.100528}.

Mcginnity, F. and G. Kingston (2017), ``An Irish Welcome? Changing Irish
Attitudes to Immigrants and Immigration: The Role of Recession and Immigration'',
The Economic and social review, Vol. 48/3, pp.~281-304,
\url{https://www.researchgate.net/publication/319670153_An_Irish_Welcome_Changing_Irish_Attitudes_to_Immigrants_and_Immigration_The_Role_of_Recession_and_Immigration}
(accessed on 2 October 2020).

Raisi-Estabragh, Z. et al.~(2020), ``Greater risk of severe COVID-19 in Black,
Asian and Minority Ethnic populations is not explained by cardiometabolic,
socioeconomic or behavioural factors, or by 25(OH)-vitamin D status: study of
1326 cases from the UK Biobank'', Journal of Public Health, Vol. 42/3, pp.
451-460, \url{http://dx.doi.org/10.1093/pubmed/fdaa095}.

Apea, V. et al.~(2020), Ethnicity and outcomes in patients hospitalised with
COVID-19 infection in East London: an observational cohort study, Cold Spring
Harbor Laboratory, \url{http://dx.doi.org/10.1101/2020.06.10.20127621}.

Bertocchi, G. and A. Dimico (2020), ``Covid-19, Race, and Redlining'', IZA
Discussion Paper, Vol. 13467,
\url{https://papers.ssrn.com/sol3/papers.cfm?abstract_id=3648807} (accessed on 22
September 2020).

Brun, S. and P. Simon (2020), ``Inégalités ethno-raciales et coronavirus'', De
Facto 19.

\begin{Shaded}
\begin{Highlighting}[]
\CommentTok{\# Importer package}
\FunctionTok{library}\NormalTok{(dplyr)}
\FunctionTok{library}\NormalTok{(tidyverse)}
\FunctionTok{library}\NormalTok{(summarytools)}
\FunctionTok{library}\NormalTok{(readr)}
\FunctionTok{library}\NormalTok{(questionr)}
\FunctionTok{library}\NormalTok{(rcompanion)}
\FunctionTok{library}\NormalTok{(pander)}
\end{Highlighting}
\end{Shaded}

\begin{Shaded}
\begin{Highlighting}[]
\CommentTok{\# Set numero graphique}
\NormalTok{no\_graphic }\OtherTok{\textless{}{-}} \SpecialCharTok{{-}}\DecValTok{1}
\end{Highlighting}
\end{Shaded}

\hypertarget{introduction-background-1}{%
\section{Introduction / background}\label{introduction-background-1}}

The COVID-19 pandemic has highlighted the vast ethnic, social, economic, and
health disparities in high-income countries, including Canada. One of the most
vulnerable segments of these affluent societies is the immigrant population,
that has become widely diversified with more and more people coming from
developing countries, and for whom integration into their new society may be
more difficult. The coronavirus pandemic brings a new challenge for these
immigrant populations, directly or indirectly through the tensions between the
ethnic groups that it helps to generate (\protect\hyperlink{ref-coates2020Covid19}{\textbf{coates2020Covid19?}}). These immigrants
face language barriers, low social capital, housing insecurity, and severe
poverty, rendering them particularly susceptible to the effects of the current
pandemic. Immigrants are also more likely to be employed as essential workers
where social distancing, sanitary precautions, and self-isolation, measures put
in place to mitigate the pandemy, are challenging (\protect\hyperlink{ref-reid2021Migrant}{\textbf{reid2021Migrant?}}~; \protect\hyperlink{ref-yi2021Health}{\textbf{yi2021Health?}}~; \protect\hyperlink{ref-zhang2021COVID19}{\textbf{zhang2021COVID19?}}). More importantly, migrants may encounter
various forms of discrimination, leading to social exclusion and restricted
access to healthcare and social services (\protect\hyperlink{ref-spiritus-beerden2021Mental}{\textbf{spiritus-beerden2021Mental?}}). It is
therefore necessary to understand how this pandemic affects these immigrant
populations differently and the native populations in Quebec. In doing so, this
article will reveal a better understanding of how the current pandemic
aggravates existing inequalities and challenges in the province for immigrants
and their descendants.

Since the start of the pandemic, most of the confirmed cases were identified in
Montreal and areas with high immigrant populations (\protect\hyperlink{ref-miconi2021Ethnocultural}{\textbf{miconi2021Ethnocultural?}}~; \protect\hyperlink{ref-shields2020COVID19}{\textbf{shields2020COVID19?}}). The coronavirus pandemic has infected nearly 450,000
people in Quebec and has resulted in more than 11,000 deaths
(\url{https://www.quebec.ca/sante/problemes-de-sante/a-z/coronavirus-2019/situation-coronavirus-quebec},
consulted on octobre 29, 2021). This makes it the worst affected province in
Canada and one of the worst affected regions in the world during the first wave
of the pandemic. Although the majority of these deaths took place in elderly's
homes, and therefore affected more this segment of the population, its overall
influence on individuals is not to be overlooked. It has been shown that, on the
one hand, immigrants and their families have a higher risk of being exposed to
the pathogen, and, at the same time, they are more vulnerable to racism and
discrimination and face barriers to accessing social support (\protect\hyperlink{ref-2021CRARR}{\textbf{2021CRARR?}}) and
information, especially for those who didn't speak or understand the two
official langages.
Several
studies have thus shown excess mortality in ethnocultural minority populations
(\protect\hyperlink{ref-bertocchi2020Covid19}{\textbf{bertocchi2020Covid19?}}), higher unemployment, greater loss of income and
negative consequences for children and families in several countries. This
differentiated effect can be explained in different ways. On the one hand, in
terms of the direct consequences of the pandemic, the situation of
precariousness and greater poverty among immigrants exposes them more to the
more harmful consequences of the disease on their state of health. Likewise,
immigrants often work in jobs where social distancing is difficult to enforce.
On the other hand, indirectly, the solutions to the crisis affect immigrants and
natives differently. The complete shutdown of tye economy with its set of
adjustments has greatly impeded the employability of immigrants, especially
newcomers who have not yet grasped the workings of society and who cannot
benefit from state subsidies. But, it is also shown that these effects are not
uniform on the whole of the immigrant population, itself, very diverse.

Despite the growing focus on the differential impacts of the COVID-19 on
immigrants and non-immigrants segment of the population, there is still very
little knowledge about how the pandemic and its related public policy impact the
lives of immigrants and their families. The existing health datasets (largely
used to assess this effect) are not necessarily representative of the immigrant
population and do not offer an accurate picture of the complex social status and
health outcomes among this group (\protect\hyperlink{ref-choi2021Studying}{\textbf{choi2021Studying?}}~; \protect\hyperlink{ref-gagnon2021Immigration}{\textbf{gagnon2021Immigration?}}~; \protect\hyperlink{ref-khan2015Visible}{\textbf{khan2015Visible?}}). In most cases, these datasets lack data on the residence
length of immigrants and combine newcomers with long-term settlers
(\protect\hyperlink{ref-edge2013Discrimination}{\textbf{edge2013Discrimination?}}~; \protect\hyperlink{ref-hayward2021Clinical}{\textbf{hayward2021Clinical?}}). As a result, they may not allow
a detailed assessment of the nature and extent of risk exposure in sub-segments
of populations, as well as whether interventions are reaching those most at
risk. In addition, the family-level impacts of the pandemic are still poorly
understood. While a body of research has explored the association between the
pandemic and its potential effects on intimate partner violence and marital
dissolution (\protect\hyperlink{ref-bhatnagar2021Navigating}{\textbf{bhatnagar2021Navigating?}}~; \protect\hyperlink{ref-koshan2020COVID19}{\textbf{koshan2020COVID19?}}~; \protect\hyperlink{ref-morrison2021Chapter}{\textbf{morrison2021Chapter?}}~; \protect\hyperlink{ref-prime2020Risk}{\textbf{prime2020Risk?}}), few studies have examined negative familial repercussions in
the context of immigration status. There is ample evidence that sustained family
relationships greatly enhance immigrants' lives on several fronts, including
fostering economic prosperity (\protect\hyperlink{ref-marger2001Sociala}{\textbf{marger2001Sociala?}}), psychological well-being
(\protect\hyperlink{ref-beiser2002Povertya}{\textbf{beiser2002Povertya?}}), social network development (\protect\hyperlink{ref-evansluong2017Familya}{\textbf{evansluong2017Familya?}}), and
educational success for children (\protect\hyperlink{ref-munroe-blum1989Immigranta}{\textbf{munroe-blum1989Immigranta?}}). The family is,
therefore, a key component of long-term social mobility and economic success of
immigrants.

Based on an original biographic survey collected around the first wave of the
pandemic in august-november 2020 on immigrants and the native population of
Québec, this study mainly aims to understand the family exposure, consequences,
and responses of the pandemic on immigrant and native Quebec populations. We
assess a whole range of different measures of each outcome. We divided the
consequences in three dimensions, 1) access to ressources, 2) ability to perform
duties and 3) employment and income. We seek also to analyse in what ways these
outcomes differ according to the immigrants length of residence. By doing so,
this study provide a clear and global picture of the situation of different
population groups in Quebec in the face of the COVID-19.

\newpage

\newpage

\hypertarget{current-body-of-literature-and-knowledge-gaps-1}{%
\section{Current body of literature and knowledge gaps}\label{current-body-of-literature-and-knowledge-gaps-1}}

The literature on the racial and ethnic health disparities during the COVID-19
crisis can be divided into two main categories: studies addressing the
socio-economic determinants of health and those examining the delivery of health
interventions and policies. The first category is primarily comprised of
research that focuses on the link between socio-economic status (SES) of
high-risk minority groups and the differential impact of the pandemic
(\protect\hyperlink{ref-drefahl2020Populationbased}{\textbf{drefahl2020Populationbased?}}~; \protect\hyperlink{ref-greenaway2020COVID19}{\textbf{greenaway2020COVID19?}}~; \protect\hyperlink{ref-lopez2021Racial}{\textbf{lopez2021Racial?}}~; \protect\hyperlink{ref-pacheco2020Job}{\textbf{pacheco2020Job?}}~; \protect\hyperlink{ref-sundaram2020Individual}{\textbf{sundaram2020Individual?}}). For instance, an inter-country study
has found that Middle Eastern, African, and Nordic immigrants in Sweden have a
higher risk of being hospitalized and even dying from COVID-19 when compared to
the Swedish born population. In the same study, the SES, proportion of
working-age households, and neighborhood density were significantly associated
with the mortality rate (\protect\hyperlink{ref-rostila2021Disparities}{\textbf{rostila2021Disparities?}}). This trend of research
demonstrates that racial and ethnic minorities are more likely to experience
stress factors, not only due to the hazards of infection but also because of
economic and social marginalization (\protect\hyperlink{ref-alahmad2020COVID19}{\textbf{alahmad2020COVID19?}}~; \protect\hyperlink{ref-choi2020Buffering}{\textbf{choi2020Buffering?}}~; \protect\hyperlink{ref-endale2020COVID19}{\textbf{endale2020COVID19?}}~; \protect\hyperlink{ref-fitzpatrick2020Fear}{\textbf{fitzpatrick2020Fear?}}~; \protect\hyperlink{ref-serafini2021Psychological}{\textbf{serafini2021Psychological?}}). The
intersectionality of race, sex, and health conditions has also been explored in
the Canadian context of the pandemic. (\protect\hyperlink{ref-ingen2021Neighbourhoodlevel}{\textbf{ingen2021Neighbourhoodlevel?}}) demonstrated
that Ontarian neighborhoods with immigrant and racialized populations have a
disproportionate share of COVID-19 infections. In addition, areas with high SES
report a lesser number of COVID cases than low SES ones
(\protect\hyperlink{ref-ingen2021Neighbourhoodlevel}{\textbf{ingen2021Neighbourhoodlevel?}}). Another study conducted in Quebec confirms the
strong association between immigrants' backgrounds, SES, and COVID-19 related
psychological stress (\protect\hyperlink{ref-miconi2021Ethnocultural}{\textbf{miconi2021Ethnocultural?}}). The most recent reviews on the
topic suggest that immigrants disproportionately live in crowded and inadequate
housing and are employed in essential industries, leaving them with severe
income insecurities and poor physiological/psychological well-being
(\protect\hyperlink{ref-chew2020Clinical}{\textbf{chew2020Clinical?}}~; \protect\hyperlink{ref-davis2020Impact}{\textbf{davis2020Impact?}}). Studies on SES and health disparities
have emphasized that this situation is further complicated for immigrant women,
constrained by gender and afflicted by a significant power imbalance in the
household and workplace (\protect\hyperlink{ref-alarcao2021Intersections}{\textbf{alarcao2021Intersections?}}~; \protect\hyperlink{ref-esegbona-adeigbe2020COVID19}{\textbf{esegbona-adeigbe2020COVID19?}}~; \protect\hyperlink{ref-jehn2021COVID19}{\textbf{jehn2021COVID19?}}~; \protect\hyperlink{ref-mo2020Differential}{\textbf{mo2020Differential?}}~; \protect\hyperlink{ref-nardon2021Skilled}{\textbf{nardon2021Skilled?}}~; \protect\hyperlink{ref-sabri2020Effect}{\textbf{sabri2020Effect?}}~; \protect\hyperlink{ref-st-denis2020Sociodemographic}{\textbf{st-denis2020Sociodemographic?}}). According
to Lightman, immigrants and women of color are overrepresented in long-term care
facilities in Canada, where COVID infection rates have been exceptionally high.
They are also subject to exclusionary social and economic practices
(\protect\hyperlink{ref-lightman2021Caring}{\textbf{lightman2021Caring?}}).

The second category of the literature is primarily comprised of policy analyses,
research on public health communications, examining the efficacy of COVID
initiatives, the optimal allocation of government aids, and ethical issues
associated with applying new health interventions to vulnerable populations and
their communities (\protect\hyperlink{ref-aragona2020Negative}{\textbf{aragona2020Negative?}}~; \protect\hyperlink{ref-cross2021Coronavirus}{\textbf{cross2021Coronavirus?}}~; \protect\hyperlink{ref-desai2020COVID19}{\textbf{desai2020COVID19?}}~; \protect\hyperlink{ref-doyle2020Migrant}{\textbf{doyle2020Migrant?}}~; \protect\hyperlink{ref-gilman2020Modelling}{\textbf{gilman2020Modelling?}}~; \protect\hyperlink{ref-hargreaves2020Targeting}{\textbf{hargreaves2020Targeting?}}~; \protect\hyperlink{ref-langellier2020Policy}{\textbf{langellier2020Policy?}}~; \protect\hyperlink{ref-nezafatmaldonado2020Engaging}{\textbf{nezafatmaldonado2020Engaging?}}~; \protect\hyperlink{ref-silverman2020Ethics}{\textbf{silverman2020Ethics?}}~; \protect\hyperlink{ref-wilson2020Ethics}{\textbf{wilson2020Ethics?}}). Some of the questions that this body
of research addresses include: how best to respond to COVID-19 in immigrant
communities? How do public health policies impact the socio-economic status and
psychological well-being of different sub-populations? What is the structure of
the planning and implementation of government aid programs during the current
crisis? And, how can we design public health communication strategies in a way
that is culturally sensitive and inclusive?

There are several knowledge gaps in the existing literature. Most of the cited
studies do not include household data and family structure of immigrants as
variables that affect exposure risks and health outcomes of COVID-19.
Consequently, the relationship between immigrant status and COVID-19 is not
accounted for by household size, and family characteristics (i.e., income,
education, religion). A key reason is that most of the health data used in such
research projects are acquired from national registers or administrative
databases, which provide little information on immigration status and kinship.
However, as the study of Thomeer and her colleagues theorizes, the extent of
adverse social and economic impacts of COVID-19 is significantly conditional on
the composition of an immigrant's family (\protect\hyperlink{ref-thomeer2020How}{\textbf{thomeer2020How?}}). Households that
include a spouse/partner, children, or multigenerational family members most
likely experience a different level of social, psychological, and medical
vulnerability than single or nuclear ones. Therefore, studies of immigrant
populations should account for the breakdown of family configuration to better
understand the mechanisms linking social determinants and health outcomes.
Another knowledge gap concerns the current state of Canadian studies of COVID-19
and its public health crisis. There is a dearth of literature on the challenges
faced by immigrants and their families in the country. Although the burden of
COVID-19 infection is disproportionately higher among these marginalized groups
(St-Denis, 2020; Sundaram et al., 2020), research on the issues they are
experiencing is sparse and often limited to one aspect.

\newpage

\hypertarget{theoretical-framework-1}{%
\section{Theoretical framework}\label{theoretical-framework-1}}

Family as a unit of analysis can be particularly fruitful when studying health
inequalities in the context of a societal-wide crisis. Through the lens of the
COVID-19 pandemic, the following points illustrate why exploring family dynamics
provides additional insight into the underlying mechanisms of health
disparities. First, the family is the most fundamental unit of social
organization. It facilitates access to resources and the transmission of social
and psychological capital from one generation to the next. Therefore, a better
analysis of how crises and adverse life events affect people requires a better
understanding of how families relate to each other and interact with external
forces. Second, during public health emergencies, families may be thrown into
different structures due to unprecedented financial, mental, and physical
challenges. The risk of family dissolution, for example, can be heightened when
members must spend unusually long periods together at home and have limited
opportunities to engage with friends and the wider community. Conversely,
families can develop closer bonds through self-sacrifice, shared
responsibilities, and increased care and support. As the dependency status of
seniors increases, the elderlies are also more likely to be integrated into
formerly nucleus families. We posit that the extent to which a family can adapt
its functions to meet new challenges is a predictor of the well-being of its
members and the resilience of its broader population. Third, the family is a
crucial link to social institutions, such as the health care system, schools,
and government agencies whose decisions have significant downstream impacts.
Families are viewed by many public-level entities as a reference when allocating
resources, enforcing new standards, and responding to emerging needs and
demands. From this perspective, the family operates at the nexus of individual
and societal structures, embedding the tripple effects of public policy in its
most consequential ways. Finally, the family has a crucial role to play in
determining the life course of its members. As COVID-19 elevates the importance
of familial bonds, the post-pandemic period might be marked by an increased
influence of families on the lives of their members and the social structures
they inhabit.

The above conceptualization echoes several other approaches to studying
long-term health inequalities. For instance, life course theory emphasizes the
significance of family ties and social environment as they structure prospects
for individuals across their life span (Elder, 1977; Elder Jr., 1998; Macmillan,
2005; Settersten et al., 2020). Likewise, social capital theory and the
stress-buffering hypothesis consider family systems a crucial factor in health
and well-being (Choi and Jun, 2020; Cohen and McKay, 1984; Cohen and Wills,
1985; Harpham et al., 2004; Kawachi and Berkman, 2001; Lavee et al., 1987;
Mandelbaum et al., 2020). They show that the family can offer tangible (e.g.,
food, shelter, and money) and intangible resources (e.g., support, social
status, psychological comfort, and institutional advantages), protecting its
members from adverse life events. Similarly, the social-ecological model and the
theory of fundamental causes highlight how the SES, familial and extrafamilial
factors pattern people's life trajectories and health outcomes (Barkan and
Rocque, 2018; Clouston and Link, 2021; Figge et al., 2018; Kim, 2020; Libman et
al., 2012; Ungar, 2015). However, all these approaches are ostensibly focused on
life periods where the general social context is relatively stable. In this
article, we present our conceptualization of the family in times of
unprecedented social upheaval. Specifically, our framework seeks to characterize
family dynamics as a suitable lens to analyze health inequalities in the face of
a virulent public health crisis.

\newpage

\hypertarget{methodology-1}{%
\section{Methodology}\label{methodology-1}}

\hypertarget{data-1}{%
\subsection{Data}\label{data-1}}

The data for this study come from an original collection of biographical data on
immigrants and the native population of Quebec to analyse their participation in
Quebec society in various fields (social, economic, cultural, etc.). The
immigrant population is limited to recent immigration that has taken place over
the past 15 years. This survey has several components that retrace the life
course of individuals over a long period of time. The pandemic component uses a
reduced version of the \href{https://www.phenxtoolkit.org/toolkit_content/PDF/CPTS_CEFIS_FullSurveyEng.pdf}{harmonized questionnaire on exposure and the impact of
Covid19 on the family
(CEFIS)}.
The questions are divided into three parts, on exposure to the disease, its
consequences and the benefit of the policies put in place by both the federal
and provincial governments. The other parts of the questionnaire focus on the
characteristics of the respondents, their immigration history, their educational
and family background. In short, it is a detailed questionnaire that covers the
life of immigrants from birth and especially over the past 15 years. The data
collection lasted three months, from August to November 2020. In total, 1,500
people were surveyed, including around 1,000 immigrants and 500 natives
throughout Quebec.

\hypertarget{dependent-variables-1}{%
\subsubsection{Dependent variables}\label{dependent-variables-1}}

We distinguish three groups of dependent variables:

\textbf{1. Disease exposure variables}: these are measured by six questions that seek
to document whether the individual through work or a family member may have been
exposed to the disease. For example, it is asked if a family member has had
symptoms of the disease, or if a member has been hospitalized due to the
contraction of the disease.

\textbf{2. Variables on the family consequences of the disease}: we measure three
dimensions of consequences. The first group of measures assess the consequences
on accessing ressources by the respondent or any member of his family. The
second group of measure assess whether the respondent or any family member have
difficulties to perform usual duties toward any member of the family. Finally,
the last group of measures measure the consequences of the pandemic on
employment and income of any family member.

\textbf{3. Variables on government assistance}: Finally, government support is
measured by the following two questions that assessed whether the respondent or
any family member has benefited of the temporary assistance for workers from the
Government of Quebec (PATT) or of the Canadian Federal Government Emergency
Benefit (CEP). The table XXX in the apendix presents the full set of indicators
of the dependents variables

\hypertarget{main-independent-variable-1}{%
\subsubsection{Main independent variable}\label{main-independent-variable-1}}

The main independent variable is immigrant status coupled with visible minority
status. This variable takes the following modalities: 1 if the respondent is a
non-racialized native, 2 if he is a non-racialized immigrant, 3 if he is a
racialized immigrant, and 4 if the respondent is a racialized native or an
immigrant who has not answered the visible minority question. The other
variables in the analysis are: age, sex, length of residence, minority status,
size of social network.

\hypertarget{analysis-methods-1}{%
\subsection{Analysis methods}\label{analysis-methods-1}}

We use two main methods of analysis, a descriptive method and an explanatory
method. In the descriptive method, we consider each of the independent variables
in isolation and describe how they vary depending on the main independent
variable. The explanatory methods goes further by assessing whether the effect
remain when controlling for other covariates. Thus, by controlling for the other
explanatory variables, we determine the effect of immigrant status on exposure
to the pandemic, on its consequences and on the fact of benefiting or not of
government assistance. In a second series of modelling, we assess whether the
effect that we observed varies by the immigrant's length of residence or by
marital status. We use either a logistic regression on the outcome variables
that are binary and an ordonned logistic regression for the outcome that are
measure with a 5-point Lickert scale.

\newpage

\hypertarget{results-1}{%
\section{3. Results}\label{results-1}}

\hypertarget{sample-distribution-1}{%
\subsection{3.1. Sample distribution}\label{sample-distribution-1}}

\hypertarget{exposure-by-immigration-status-1}{%
\subsection{3.2. Exposure by immigration status}\label{exposure-by-immigration-status-1}}

Figure 1 presents the exposure to the pandemy accroding to the immigrants and
minority visible status. For all the six measures of exposure, it appears that
the majority group is the most exposed to the COVID-19 in all instance. The big
difference between the groups is about the exposure trhough working outside.
Around 50\% of the member of this group or a family member work outside the home
at the time of the survey. This percentage is only 30\% for the racialised
immigrants and 25\% for the non-racialised immigrants.

\begin{figure}
\centering
\includegraphics[width=0.6\textwidth,height=\textheight]{Figures/Exposition1.png}
\caption{Figure 1.1: Exposure}
\end{figure}

\begin{figure}
\centering
\includegraphics{Figures/Exposure.jpeg}
\caption{Figure 1.2: Explicative model}
\end{figure}

\hypertarget{consequences-by-immigrations-status-1}{%
\subsection{3.3. Consequences by immigrations status}\label{consequences-by-immigrations-status-1}}

The Figure 2 presents the consequences of the COVID-19 for a family member to
access to ressources. It shows that in all accounts but one (getting medecine),
the racialised immigrants express more diffciult to assess ressources. Around
20\% of them stated having difficulty to get health care compare, around 10\% for
obtaining food and 7\% for obtaining family insurrance. Comparatively, these
proportion are respectively 13\%, 7\% and 4\% for people and thier family member in
the majority group and 19\%, 3\% and 4\% respectively among the racialised
immigrants.

\begin{figure}
\centering
\includegraphics[width=0.8\textwidth,height=\textheight]{Figures/Famille1.png}
\caption{Figure 3.1: Consequences on accessing
ressources}
\end{figure}

\begin{figure}
\centering
\includegraphics{Figures/Ressources.jpeg}
\caption{Figure 3.2: Explicative model}
\end{figure}

\hypertarget{consequences-on-performing-usual-duties-1}{%
\subsection{3.4. Consequences on performing usual duties}\label{consequences-on-performing-usual-duties-1}}

The Figure 4 compares the response from the three groups on performing usual
duties. It appears that for all groups the situation caused by the COVID-19 has
inexpectedly improved the situation in various way, irrespective of the
immigrants status of the respondents.

\begin{figure}
\centering
\includegraphics{Figures/Entente1.png}
\caption{Figure 4.1: Consequences on performing usual duties}
\end{figure}

\begin{figure}
\centering
\includegraphics{Figures/Abilities.jpeg}
\caption{Figure 4.2: Explicative model}
\end{figure}

\hypertarget{consequences-on-employment-by-immigration-status-1}{%
\subsection{3.5. Consequences on employment by immigration status}\label{consequences-on-employment-by-immigration-status-1}}

The consequences on employment are mixed, with the severe consequences more
prevalent among the racialised immigrants. The Figure 3 shows for instance that
close to 45\% of the racialised immigrants have stated that their family income
has decreased compare to just 33\% of the majority group and 37\% of the
non-racialed immigrants. In addition, 11\% of the racialised immigrants reveals
having loss their job compared to 9\% among the non-racialised immigrants and 2\%
among members of the majority group. At the opposite, the members of the
majority groups are more likely to report a reduction of the working hours or
for stopping momentarily their work.

\begin{figure}
\centering
\includegraphics{Figures/Emploi1.png}
\caption{Figure 5.1: Consequences on employment and income}
\end{figure}

\begin{figure}
\centering
\includegraphics{Figures/Employment.jpeg}
\caption{Figure 5.2: Explicative model}
\end{figure}

\hypertarget{help-received-by-immigrations-status-1}{%
\subsection{3.7 Help received by immigrations status}\label{help-received-by-immigrations-status-1}}

Finally, we compared the different group on benefitting of the government
assistance. It appears that racialised immigrants with 35\% of the case have
expressed having received the Canada Emergency Response Benefit wheras this
proportion is 31\% for the non-racialised immigrants and 25\% for the member of
the majority group. For the province Temporary Aid for Workers Program, around
5\% of the respondents have received it.

\includegraphics{Figures/Aide1.png}

\begin{figure}
\centering
\includegraphics{Figures/Government.jpeg}
\caption{Figure 3.2: Explicative model}
\end{figure}

\newpage

\hypertarget{conclusion-1}{%
\section{Conclusion}\label{conclusion-1}}

The preliminary results of this study show that the consequences of the pandemic
on the Quebec population differ according to the population groups, but also
according to the type of consequences. \newpage

\hypertarget{references}{%
\section{References}\label{references}}

\url{http://www.oecd.org/coronavirus/policy-responses/what-is-the-impact-of-the-covid-19-pandemic-on-immigrants-and-their-children-e7cbb7de/}

Guttmann A, F. et al.~(2020), COVID-19 in Immigrants, Refugees and Other
Newcomers in Ontario: Characteristics of Those Tested and Those Confirmed
Positive, as of June 13, 2020, ICES, \url{http://www.ices.on.ca}.

Hu, Y. (2020), ``Intersecting ethnic and native--migrant inequalities in the
economic impact of the COVID-19 pandemic in the UK'', Research in Social
Stratification and Mobility, Vol. 68, p.~100528,
\url{http://dx.doi.org/10.1016/j.rssm.2020.100528}.

Mcginnity, F. and G. Kingston (2017), ``An Irish Welcome? Changing Irish
Attitudes to Immigrants and Immigration: The Role of Recession and Immigration'',
The Economic and social review, Vol. 48/3, pp.~281-304,
\url{https://www.researchgate.net/publication/319670153_An_Irish_Welcome_Changing_Irish_Attitudes_to_Immigrants_and_Immigration_The_Role_of_Recession_and_Immigration}
(accessed on 2 October 2020).

Raisi-Estabragh, Z. et al.~(2020), ``Greater risk of severe COVID-19 in Black,
Asian and Minority Ethnic populations is not explained by cardiometabolic,
socioeconomic or behavioural factors, or by 25(OH)-vitamin D status: study of
1326 cases from the UK Biobank'', Journal of Public Health, Vol. 42/3, pp.
451-460, \url{http://dx.doi.org/10.1093/pubmed/fdaa095}.

Apea, V. et al.~(2020), Ethnicity and outcomes in patients hospitalised with
COVID-19 infection in East London: an observational cohort study, Cold Spring
Harbor Laboratory, \url{http://dx.doi.org/10.1101/2020.06.10.20127621}.

Bertocchi, G. and A. Dimico (2020), ``Covid-19, Race, and Redlining'', IZA
Discussion Paper, Vol. 13467,
\url{https://papers.ssrn.com/sol3/papers.cfm?abstract_id=3648807} (accessed on 22
September 2020).

Brun, S. and P. Simon (2020), ``Inégalités ethno-raciales et coronavirus'', De
Facto 19.

\end{document}
