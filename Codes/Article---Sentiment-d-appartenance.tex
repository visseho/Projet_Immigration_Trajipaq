% Options for packages loaded elsewhere
\PassOptionsToPackage{unicode}{hyperref}
\PassOptionsToPackage{hyphens}{url}
%
\documentclass[
]{article}
\usepackage{amsmath,amssymb}
\usepackage{lmodern}
\usepackage{iftex}
\ifPDFTeX
  \usepackage[T1]{fontenc}
  \usepackage[utf8]{inputenc}
  \usepackage{textcomp} % provide euro and other symbols
\else % if luatex or xetex
  \usepackage{unicode-math}
  \defaultfontfeatures{Scale=MatchLowercase}
  \defaultfontfeatures[\rmfamily]{Ligatures=TeX,Scale=1}
\fi
% Use upquote if available, for straight quotes in verbatim environments
\IfFileExists{upquote.sty}{\usepackage{upquote}}{}
\IfFileExists{microtype.sty}{% use microtype if available
  \usepackage[]{microtype}
  \UseMicrotypeSet[protrusion]{basicmath} % disable protrusion for tt fonts
}{}
\makeatletter
\@ifundefined{KOMAClassName}{% if non-KOMA class
  \IfFileExists{parskip.sty}{%
    \usepackage{parskip}
  }{% else
    \setlength{\parindent}{0pt}
    \setlength{\parskip}{6pt plus 2pt minus 1pt}}
}{% if KOMA class
  \KOMAoptions{parskip=half}}
\makeatother
\usepackage{xcolor}
\usepackage[margin=1in]{geometry}
\usepackage{color}
\usepackage{fancyvrb}
\newcommand{\VerbBar}{|}
\newcommand{\VERB}{\Verb[commandchars=\\\{\}]}
\DefineVerbatimEnvironment{Highlighting}{Verbatim}{commandchars=\\\{\}}
% Add ',fontsize=\small' for more characters per line
\usepackage{framed}
\definecolor{shadecolor}{RGB}{248,248,248}
\newenvironment{Shaded}{\begin{snugshade}}{\end{snugshade}}
\newcommand{\AlertTok}[1]{\textcolor[rgb]{0.94,0.16,0.16}{#1}}
\newcommand{\AnnotationTok}[1]{\textcolor[rgb]{0.56,0.35,0.01}{\textbf{\textit{#1}}}}
\newcommand{\AttributeTok}[1]{\textcolor[rgb]{0.77,0.63,0.00}{#1}}
\newcommand{\BaseNTok}[1]{\textcolor[rgb]{0.00,0.00,0.81}{#1}}
\newcommand{\BuiltInTok}[1]{#1}
\newcommand{\CharTok}[1]{\textcolor[rgb]{0.31,0.60,0.02}{#1}}
\newcommand{\CommentTok}[1]{\textcolor[rgb]{0.56,0.35,0.01}{\textit{#1}}}
\newcommand{\CommentVarTok}[1]{\textcolor[rgb]{0.56,0.35,0.01}{\textbf{\textit{#1}}}}
\newcommand{\ConstantTok}[1]{\textcolor[rgb]{0.00,0.00,0.00}{#1}}
\newcommand{\ControlFlowTok}[1]{\textcolor[rgb]{0.13,0.29,0.53}{\textbf{#1}}}
\newcommand{\DataTypeTok}[1]{\textcolor[rgb]{0.13,0.29,0.53}{#1}}
\newcommand{\DecValTok}[1]{\textcolor[rgb]{0.00,0.00,0.81}{#1}}
\newcommand{\DocumentationTok}[1]{\textcolor[rgb]{0.56,0.35,0.01}{\textbf{\textit{#1}}}}
\newcommand{\ErrorTok}[1]{\textcolor[rgb]{0.64,0.00,0.00}{\textbf{#1}}}
\newcommand{\ExtensionTok}[1]{#1}
\newcommand{\FloatTok}[1]{\textcolor[rgb]{0.00,0.00,0.81}{#1}}
\newcommand{\FunctionTok}[1]{\textcolor[rgb]{0.00,0.00,0.00}{#1}}
\newcommand{\ImportTok}[1]{#1}
\newcommand{\InformationTok}[1]{\textcolor[rgb]{0.56,0.35,0.01}{\textbf{\textit{#1}}}}
\newcommand{\KeywordTok}[1]{\textcolor[rgb]{0.13,0.29,0.53}{\textbf{#1}}}
\newcommand{\NormalTok}[1]{#1}
\newcommand{\OperatorTok}[1]{\textcolor[rgb]{0.81,0.36,0.00}{\textbf{#1}}}
\newcommand{\OtherTok}[1]{\textcolor[rgb]{0.56,0.35,0.01}{#1}}
\newcommand{\PreprocessorTok}[1]{\textcolor[rgb]{0.56,0.35,0.01}{\textit{#1}}}
\newcommand{\RegionMarkerTok}[1]{#1}
\newcommand{\SpecialCharTok}[1]{\textcolor[rgb]{0.00,0.00,0.00}{#1}}
\newcommand{\SpecialStringTok}[1]{\textcolor[rgb]{0.31,0.60,0.02}{#1}}
\newcommand{\StringTok}[1]{\textcolor[rgb]{0.31,0.60,0.02}{#1}}
\newcommand{\VariableTok}[1]{\textcolor[rgb]{0.00,0.00,0.00}{#1}}
\newcommand{\VerbatimStringTok}[1]{\textcolor[rgb]{0.31,0.60,0.02}{#1}}
\newcommand{\WarningTok}[1]{\textcolor[rgb]{0.56,0.35,0.01}{\textbf{\textit{#1}}}}
\usepackage{graphicx}
\makeatletter
\def\maxwidth{\ifdim\Gin@nat@width>\linewidth\linewidth\else\Gin@nat@width\fi}
\def\maxheight{\ifdim\Gin@nat@height>\textheight\textheight\else\Gin@nat@height\fi}
\makeatother
% Scale images if necessary, so that they will not overflow the page
% margins by default, and it is still possible to overwrite the defaults
% using explicit options in \includegraphics[width, height, ...]{}
\setkeys{Gin}{width=\maxwidth,height=\maxheight,keepaspectratio}
% Set default figure placement to htbp
\makeatletter
\def\fps@figure{htbp}
\makeatother
\setlength{\emergencystretch}{3em} % prevent overfull lines
\providecommand{\tightlist}{%
  \setlength{\itemsep}{0pt}\setlength{\parskip}{0pt}}
\setcounter{secnumdepth}{-\maxdimen} % remove section numbering
\ifLuaTeX
  \usepackage{selnolig}  % disable illegal ligatures
\fi
\IfFileExists{bookmark.sty}{\usepackage{bookmark}}{\usepackage{hyperref}}
\IfFileExists{xurl.sty}{\usepackage{xurl}}{} % add URL line breaks if available
\urlstyle{same} % disable monospaced font for URLs
\hypersetup{
  pdftitle={Article - Sentiment d'appartenance},
  pdfauthor={Thea},
  hidelinks,
  pdfcreator={LaTeX via pandoc}}

\title{Article - Sentiment d'appartenance}
\author{Thea}
\date{2023-05-17}

\begin{document}
\maketitle

\begin{Shaded}
\begin{Highlighting}[]
\FunctionTok{library}\NormalTok{(summarytools)}
\end{Highlighting}
\end{Shaded}

\begin{verbatim}
## Warning: le package 'summarytools' a été compilé avec la version R 4.2.3
\end{verbatim}

\begin{Shaded}
\begin{Highlighting}[]
\FunctionTok{library}\NormalTok{(tidyverse)}
\end{Highlighting}
\end{Shaded}

\begin{verbatim}
## Warning: le package 'tidyverse' a été compilé avec la version R 4.2.3
\end{verbatim}

\begin{verbatim}
## Warning: le package 'ggplot2' a été compilé avec la version R 4.2.3
\end{verbatim}

\begin{verbatim}
## Warning: le package 'tibble' a été compilé avec la version R 4.2.3
\end{verbatim}

\begin{verbatim}
## Warning: le package 'tidyr' a été compilé avec la version R 4.2.3
\end{verbatim}

\begin{verbatim}
## Warning: le package 'readr' a été compilé avec la version R 4.2.3
\end{verbatim}

\begin{verbatim}
## Warning: le package 'dplyr' a été compilé avec la version R 4.2.3
\end{verbatim}

\begin{verbatim}
## Warning: le package 'forcats' a été compilé avec la version R 4.2.3
\end{verbatim}

\begin{verbatim}
## Warning: le package 'lubridate' a été compilé avec la version R 4.2.3
\end{verbatim}

\begin{verbatim}
## -- Attaching core tidyverse packages ------------------------ tidyverse 2.0.0 --
## v dplyr     1.1.2     v readr     2.1.4
## v forcats   1.0.0     v stringr   1.5.0
## v ggplot2   3.4.2     v tibble    3.2.1
## v lubridate 1.9.2     v tidyr     1.3.0
## v purrr     1.0.1     
## -- Conflicts ------------------------------------------ tidyverse_conflicts() --
## x dplyr::filter() masks stats::filter()
## x dplyr::lag()    masks stats::lag()
## x tibble::view()  masks summarytools::view()
## i Use the ]8;;http://conflicted.r-lib.org/conflicted package]8;; to force all conflicts to become errors
\end{verbatim}

\begin{Shaded}
\begin{Highlighting}[]
\FunctionTok{library}\NormalTok{(ggthemes)}
\end{Highlighting}
\end{Shaded}

\begin{verbatim}
## Warning: le package 'ggthemes' a été compilé avec la version R 4.2.3
\end{verbatim}

\begin{Shaded}
\begin{Highlighting}[]
\FunctionTok{library}\NormalTok{(haven)}
\end{Highlighting}
\end{Shaded}

\begin{verbatim}
## Warning: le package 'haven' a été compilé avec la version R 4.2.3
\end{verbatim}

\begin{Shaded}
\begin{Highlighting}[]
\NormalTok{Base\_Trajipaq\_Finale\_Ok }\OtherTok{\textless{}{-}} \FunctionTok{read\_dta}\NormalTok{(}\StringTok{"C:/Users/Théo CARLIER/OneDrive/Bureau/Base\_Trajipaq\_Finale\_Ok.dta"}\NormalTok{)}
\end{Highlighting}
\end{Shaded}

\hypertarget{question-de-recherche-de-quelles-fauxe7ons-les-discriminations-que-subissent-les-personnes-immigrantes-influencent-leur-sentiment-dappartenance-au-quuxe9bec}{%
\section{Question de recherche : de quelles façons les discriminations
que subissent les personnes immigrantes influencent leur sentiment
d'appartenance au Québec
?}\label{question-de-recherche-de-quelles-fauxe7ons-les-discriminations-que-subissent-les-personnes-immigrantes-influencent-leur-sentiment-dappartenance-au-quuxe9bec}}

\hypertarget{variable-induxe9pendante-les-discriminations-que-subissent-les-personnes-immigrantes}{%
\subsection{Variable indépendante = Les discriminations que subissent
les personnes immigrantes
:}\label{variable-induxe9pendante-les-discriminations-que-subissent-les-personnes-immigrantes}}

\begin{itemize}
\tightlist
\item
  dans l'emploi
\item
  dans la recherche de logement
\item
  dans la santé
\item
  sur le lieu de formation
\end{itemize}

Type de variable = factoriel Niveau de mesure = ordinal (échelle de
Likert)

Réponses possibles :

\begin{enumerate}
\def\labelenumi{\arabic{enumi}.}
\tightlist
\item
  Tout à fait d'accord
\item
  Plutôt d'accord
\item
  Plutôt pas d'accord
\item
  Pas du tout d'accord
\item
  Je ne sais pas
\item
  Je ne préfère pas répondre
\end{enumerate}

\hypertarget{variable-duxe9pendante-le-sentiment-dappartenance-au-quuxe9bec}{%
\subsection{Variable dépendante = le sentiment d'appartenance au Québec
:}\label{variable-duxe9pendante-le-sentiment-dappartenance-au-quuxe9bec}}

\begin{itemize}
\tightlist
\item
  se sentir chez soi au Québec
\item
  la nécessité d'oublier sa culture d'origine pour s'intégrer au Québec
\item
  être vu(e) comme un(e) Québécois(e)
\item
  se sentir Québécois(e)
\item
  se sentir Canadien(ne)
\item
  avoir un sentiment d'appartenance avec son pays d'origine
\end{itemize}

Type de variable = factoriel Niveau de mesure = ordinal (échelle de
Likert)

Réponses possibles :

\begin{enumerate}
\def\labelenumi{\arabic{enumi}.}
\tightlist
\item
  Tout à fait d'accord
\item
  Plutôt d'accord
\item
  Plutôt pas d'accord
\item
  Pas du tout d'accord
\item
  Je ne sais pas
\item
  Je ne préfère pas répondre
\end{enumerate}

\hypertarget{variables-de-contruxf4le-caractuxe9ristiques-socioduxe9mographiques}{%
\subsection{Variables de contrôle = caractéristiques sociodémographiques
:}\label{variables-de-contruxf4le-caractuxe9ristiques-socioduxe9mographiques}}

\begin{itemize}
\tightlist
\item
  sexe
\item
  groupe d'âge
\item
  statut matrimonial
\item
  pays d'origine
\item
  statut d'immigration au Québec
\item
  statut d'immigration pré-résidence
\item
  niveau d'éducation
\item
  religion
\item
  compétence linguistique (français)
\item
  compétence linguistique (anglais)
\item
  état de santé actuel
\item
  état de santé depuis 12 mois
\item
  niveau d'éducation de la mère
\item
  niveau d'éducation du père
\end{itemize}

\hypertarget{pruxe9sentation-des-variables-de-contruxf4le}{%
\section{Présentation des variables de
contrôle}\label{pruxe9sentation-des-variables-de-contruxf4le}}

\hypertarget{q10.-uxeates-vous-un-homme-une-femme-ou-autre}{%
\subsection{Q10. Êtes-vous un homme, une femme ou autre
?}\label{q10.-uxeates-vous-un-homme-une-femme-ou-autre}}

\begin{enumerate}
\def\labelenumi{\arabic{enumi}.}
\tightlist
\item
  Homme
\item
  Femme
\item
  Autre
\end{enumerate}

Type de variable = catégorielle Niveau de mesure = nominal

\begin{Shaded}
\begin{Highlighting}[]
\NormalTok{Base\_Trajipaq\_Finale\_Ok }\OtherTok{\textless{}{-}}
\NormalTok{  Base\_Trajipaq\_Finale\_Ok }\SpecialCharTok{\%\textgreater{}\%}
  \FunctionTok{mutate}\NormalTok{(}\AttributeTok{Q10\_F =} \FunctionTok{case\_when}\NormalTok{(}
\NormalTok{    Q10 }\SpecialCharTok{==} \DecValTok{1} \SpecialCharTok{\textasciitilde{}} \StringTok{"Homme"}\NormalTok{,}
\NormalTok{    Q10 }\SpecialCharTok{==} \DecValTok{2} \SpecialCharTok{\textasciitilde{}} \StringTok{"Femme"}\NormalTok{,}
\NormalTok{    Q10 }\SpecialCharTok{==} \DecValTok{97} \SpecialCharTok{\textasciitilde{}} \StringTok{"Autre"}
\NormalTok{  ))}
\end{Highlighting}
\end{Shaded}

\begin{Shaded}
\begin{Highlighting}[]
\FunctionTok{freq}\NormalTok{(Base\_Trajipaq\_Finale\_Ok}\SpecialCharTok{$}\NormalTok{Q10\_F)}
\end{Highlighting}
\end{Shaded}

\begin{verbatim}
## Frequencies  
## Base_Trajipaq_Finale_Ok$Q10_F  
## Type: Character  
## 
##               Freq   % Valid   % Valid Cum.   % Total   % Total Cum.
## ----------- ------ --------- -------------- --------- --------------
##       Femme    886     56.69          56.69     56.69          56.69
##       Homme    677     43.31         100.00     43.31         100.00
##        <NA>      0                               0.00         100.00
##       Total   1563    100.00         100.00    100.00         100.00
\end{verbatim}

\begin{Shaded}
\begin{Highlighting}[]
\FunctionTok{ggplot}\NormalTok{(}\AttributeTok{data =}\NormalTok{ Base\_Trajipaq\_Finale\_Ok, }\AttributeTok{mapping =} \FunctionTok{aes}\NormalTok{(}\AttributeTok{x =}\NormalTok{ Q10\_F)) }\SpecialCharTok{+}
  \FunctionTok{geom\_bar}\NormalTok{(}\AttributeTok{fill =} \StringTok{"pink"}\NormalTok{) }\SpecialCharTok{+}
  \FunctionTok{theme\_bw}\NormalTok{()}
\end{Highlighting}
\end{Shaded}

\includegraphics{Article---Sentiment-d-appartenance_files/figure-latex/unnamed-chunk-5-1.pdf}

La réponse modale est ``femme'' (56.69\%).

\hypertarget{q8b-a-quel-groupe-duxe2ge-appartenez-vous}{%
\subsection{Q8B A quel groupe d'âge appartenez-vous
?}\label{q8b-a-quel-groupe-duxe2ge-appartenez-vous}}

\begin{enumerate}
\def\labelenumi{\arabic{enumi}.}
\tightlist
\item
  0 -- 19 ans
\item
  20 -- 24 ans
\item
  25 -- 29 ans
\item
  30 -- 34 ans
\item
  35 -- 39 ans
\item
  40 -- 44 ans
\item
  45 -- 49 ans
\item
  50 -- 60 ans
\item
  61 ans et plus
\item
  Je préfère ne pas répondre
\end{enumerate}

Type de variable = factoriel Niveau de mesure = ordinal

\begin{Shaded}
\begin{Highlighting}[]
\FunctionTok{freq}\NormalTok{(Base\_Trajipaq\_Finale\_Ok}\SpecialCharTok{$}\NormalTok{QAGE)}
\end{Highlighting}
\end{Shaded}

\begin{verbatim}
## Frequencies  
## Base_Trajipaq_Finale_Ok$QAGE  
## Label: Vous appartenez à quel groupe d’âge?  
## Type: Numeric  
## 
##               Freq   % Valid   % Valid Cum.   % Total   % Total Cum.
## ----------- ------ --------- -------------- --------- --------------
##           2     76      4.86           4.86      4.86           4.86
##           3    136      8.70          13.56      8.70          13.56
##           4    259     16.57          30.13     16.57          30.13
##           5    385     24.63          54.77     24.63          54.77
##           6    328     20.99          75.75     20.99          75.75
##           7    182     11.64          87.40     11.64          87.40
##           8    125      8.00          95.39      8.00          95.39
##           9     72      4.61         100.00      4.61         100.00
##        <NA>      0                               0.00         100.00
##       Total   1563    100.00         100.00    100.00         100.00
\end{verbatim}

\begin{Shaded}
\begin{Highlighting}[]
\FunctionTok{ggplot}\NormalTok{(}\AttributeTok{data =}\NormalTok{ Base\_Trajipaq\_Finale\_Ok, }\AttributeTok{mapping =} \FunctionTok{aes}\NormalTok{(}\AttributeTok{x =}\NormalTok{ QAGE)) }\SpecialCharTok{+}
  \FunctionTok{geom\_bar}\NormalTok{(}\AttributeTok{fill =} \StringTok{"pink"}\NormalTok{, }\AttributeTok{na.rm =} \ConstantTok{TRUE}\NormalTok{) }\SpecialCharTok{+}
  \FunctionTok{theme\_bw}\NormalTok{()}
\end{Highlighting}
\end{Shaded}

\includegraphics{Article---Sentiment-d-appartenance_files/figure-latex/unnamed-chunk-7-1.pdf}

La distribution est symétrique. Le groupe d'âge modal est ``35-39 ans''
(24.63\%).

\hypertarget{statut.-uxeates-vous-actuellement}{%
\subsection{STATUT. Êtes-vous actuellement\ldots{}
?}\label{statut.-uxeates-vous-actuellement}}

\begin{enumerate}
\def\labelenumi{\arabic{enumi}.}
\tightlist
\item
  Célibataire/jamais marié(e) ou conjoint(e) de fait
\item
  Marié(e)
\item
  Conjoint(e) de fait/en union libre
\item
  Séparé(e)
\item
  Divorcé(e)
\item
  Veuf(ve)
\end{enumerate}

Type de variable = catégorielle Niveau de mesure = nominal

\begin{Shaded}
\begin{Highlighting}[]
\FunctionTok{freq}\NormalTok{(Base\_Trajipaq\_Finale\_Ok}\SpecialCharTok{$}\NormalTok{STATUT)}
\end{Highlighting}
\end{Shaded}

\begin{verbatim}
## Frequencies  
## Base_Trajipaq_Finale_Ok$STATUT  
## Label: Êtes-vous actuellement...?  
## Type: Numeric  
## 
##               Freq   % Valid   % Valid Cum.   % Total   % Total Cum.
## ----------- ------ --------- -------------- --------- --------------
##           1    363     24.26          24.26     23.22          23.22
##           2    711     47.53          71.79     45.49          68.71
##           3    331     22.13          93.92     21.18          89.89
##           4     39      2.61          96.52      2.50          92.39
##           5     46      3.07          99.60      2.94          95.33
##           6      6      0.40         100.00      0.38          95.71
##        <NA>     67                               4.29         100.00
##       Total   1563    100.00         100.00    100.00         100.00
\end{verbatim}

\begin{Shaded}
\begin{Highlighting}[]
\FunctionTok{ggplot}\NormalTok{(}\AttributeTok{data =}\NormalTok{ Base\_Trajipaq\_Finale\_Ok, }\AttributeTok{mapping =} \FunctionTok{aes}\NormalTok{(}\AttributeTok{x =}\NormalTok{ STATUT)) }\SpecialCharTok{+}
  \FunctionTok{geom\_bar}\NormalTok{(}\AttributeTok{fill =} \StringTok{"pink"}\NormalTok{, }\AttributeTok{na.rm =} \ConstantTok{TRUE}\NormalTok{) }\SpecialCharTok{+}
  \FunctionTok{theme\_bw}\NormalTok{()}
\end{Highlighting}
\end{Shaded}

\includegraphics{Article---Sentiment-d-appartenance_files/figure-latex/unnamed-chunk-9-1.pdf}

La réponse modale est ``Marié(e)'' (47.53\%)

\hypertarget{q11.-dans-quel-pays-uxeates-vous-nuxe9}{%
\subsection{Q11. Dans quel pays êtes-vous né
?}\label{q11.-dans-quel-pays-uxeates-vous-nuxe9}}

Pays. (liste déroulante) 998. Je ne sais pas 999. Je préfère ne pas
répondre

Type de variable = catégorielle Niveau de mesure = nominal

\begin{Shaded}
\begin{Highlighting}[]
\FunctionTok{freq}\NormalTok{(Base\_Trajipaq\_Finale\_Ok}\SpecialCharTok{$}\NormalTok{Q11)}
\end{Highlighting}
\end{Shaded}

\begin{verbatim}
## Frequencies  
## Base_Trajipaq_Finale_Ok$Q11  
## Label: Dans quel pays êtes-vous né(e) ?Si vous ne vous en souvenez pas, inscrire : « Je  
## Type: Numeric  
## 
##               Freq   % Valid   % Valid Cum.   % Total   % Total Cum.
## ----------- ------ --------- -------------- --------- --------------
##           2      1     0.090          0.090     0.064          0.064
##           4      1     0.090          0.180     0.064          0.128
##           5     74     6.649          6.828     4.734          4.862
##           6      6     0.539          7.367     0.384          5.246
##          12      2     0.180          7.547     0.128          5.374
##          13      8     0.719          8.266     0.512          5.886
##          14      1     0.090          8.356     0.064          5.950
##          16      2     0.180          8.535     0.128          6.078
##          18      1     0.090          8.625     0.064          6.142
##          21      3     0.270          8.895     0.192          6.334
##          23      3     0.270          9.164     0.192          6.526
##          24     18     1.617         10.782     1.152          7.678
##          26      4     0.359         11.141     0.256          7.933
##          35     24     2.156         13.297     1.536          9.469
##          37     12     1.078         14.376     0.768         10.237
##          38      4     0.359         14.735     0.256         10.493
##          39      3     0.270         15.004     0.192         10.685
##          40      1     0.090         15.094     0.064         10.749
##          42      1     0.090         15.184     0.064         10.813
##          43     30     2.695         17.880     1.919         12.732
##          45      1     0.090         17.969     0.064         12.796
##          46      1     0.090         18.059     0.064         12.860
##          47     35     3.145         21.204     2.239         15.099
##          50      1     0.090         21.294     0.064         15.163
##          52     36     3.235         24.528     2.303         17.466
##          54      6     0.539         25.067     0.384         17.850
##          55      1     0.090         25.157     0.064         17.914
##          58      1     0.090         25.247     0.064         17.978
##          59      1     0.090         25.337     0.064         18.042
##          60     12     1.078         26.415     0.768         18.810
##          62     15     1.348         27.763     0.960         19.770
##          65      1     0.090         27.853     0.064         19.834
##          66      3     0.270         28.122     0.192         20.026
##          68      6     0.539         28.661     0.384         20.409
##          69      3     0.270         28.931     0.192         20.601
##          70      1     0.090         29.021     0.064         20.665
##          73      4     0.359         29.380     0.256         20.921
##          75     17     1.527         30.907     1.088         22.009
##          79      1     0.090         30.997     0.064         22.073
##          81    297    26.685         57.682    19.002         41.075
##          82      1     0.090         57.772     0.064         41.139
##          86      1     0.090         57.862     0.064         41.203
##          88      1     0.090         57.951     0.064         41.267
##          89      1     0.090         58.041     0.064         41.331
##          91      1     0.090         58.131     0.064         41.395
##          95      4     0.359         58.491     0.256         41.651
##         100     28     2.516         61.006     1.791         43.442
##         103     10     0.898         61.905     0.640         44.082
##         106      1     0.090         61.995     0.064         44.146
##         107     13     1.168         63.163     0.832         44.978
##         109      8     0.719         63.881     0.512         45.489
##         113      6     0.539         64.420     0.384         45.873
##         114      6     0.539         64.960     0.384         46.257
##         115      2     0.180         65.139     0.128         46.385
##         116      6     0.539         65.678     0.384         46.769
##         119      3     0.270         65.948     0.192         46.961
##         121      1     0.090         66.038     0.064         47.025
##         127      1     0.090         66.128     0.064         47.089
##         128     35     3.145         69.272     2.239         49.328
##         136      1     0.090         69.362     0.064         49.392
##         140      4     0.359         69.721     0.256         49.648
##         143     57     5.121         74.843     3.647         53.295
##         146      8     0.719         75.562     0.512         53.807
##         149     14     1.258         76.819     0.896         54.702
##         151     16     1.438         78.257     1.024         55.726
##         152      1     0.090         78.347     0.064         55.790
##         161      1     0.090         78.437     0.064         55.854
##         162      2     0.180         78.616     0.128         55.982
##         167      1     0.090         78.706     0.064         56.046
##         169      1     0.090         78.796     0.064         56.110
##         170      1     0.090         78.886     0.064         56.174
##         171      3     0.270         79.155     0.192         56.366
##         174      2     0.180         79.335     0.128         56.494
##         176      1     0.090         79.425     0.064         56.558
##         177     12     1.078         80.503     0.768         57.326
##         178      8     0.719         81.222     0.512         57.837
##         180      2     0.180         81.402     0.128         57.965
##         181      1     0.090         81.491     0.064         58.029
##         183      2     0.180         81.671     0.128         58.157
##         185      2     0.180         81.851     0.128         58.285
##         186     32     2.875         84.726     2.047         60.333
##         187      7     0.629         85.355     0.448         60.781
##         188     11     0.988         86.343     0.704         61.484
##         189      1     0.090         86.433     0.064         61.548
##         200      1     0.090         86.523     0.064         61.612
##         205      6     0.539         87.062     0.384         61.996
##         210      1     0.090         87.152     0.064         62.060
##         211      1     0.090         87.242     0.064         62.124
##         216      2     0.180         87.421     0.128         62.252
##         217      1     0.090         87.511     0.064         62.316
##         218      8     0.719         88.230     0.512         62.828
##         222     12     1.078         89.308     0.768         63.596
##         224      3     0.270         89.578     0.192         63.788
##         226      3     0.270         89.847     0.192         63.980
##         232      4     0.359         90.207     0.256         64.235
##         234      1     0.090         90.296     0.064         64.299
##         236     27     2.426         92.722     1.727         66.027
##         239      4     0.359         93.082     0.256         66.283
##         241     12     1.078         94.160     0.768         67.051
##         242      7     0.629         94.789     0.448         67.498
##         244     14     1.258         96.047     0.896         68.394
##         247     10     0.898         96.945     0.640         69.034
##         249      1     0.090         97.035     0.064         69.098
##         998      9     0.809         97.844     0.576         69.674
##         999     24     2.156        100.000     1.536         71.209
##        <NA>    450                             28.791        100.000
##       Total   1563   100.000        100.000   100.000        100.000
\end{verbatim}

\begin{Shaded}
\begin{Highlighting}[]
\FunctionTok{ggplot}\NormalTok{(}\AttributeTok{data =}\NormalTok{ Base\_Trajipaq\_Finale\_Ok, }\AttributeTok{mapping =} \FunctionTok{aes}\NormalTok{(}\AttributeTok{x =} \FunctionTok{as.factor}\NormalTok{(Q11))) }\SpecialCharTok{+}
  \FunctionTok{geom\_bar}\NormalTok{(}\AttributeTok{fill =} \StringTok{"pink"}\NormalTok{) }\SpecialCharTok{+}
  \FunctionTok{theme\_bw}\NormalTok{()}
\end{Highlighting}
\end{Shaded}

\includegraphics{Article---Sentiment-d-appartenance_files/figure-latex/unnamed-chunk-11-1.pdf}

Les réponses modales sont ``France'' (26.685\%) et ``Algérie''
(6.649\%).

\hypertarget{q4.-uxeates-vous-nuxe9-au-canada}{%
\subsection{Q4. Êtes-vous né au Canada
?}\label{q4.-uxeates-vous-nuxe9-au-canada}}

Type de variable = catégoriel Niveau de mesure = nominal

\begin{Shaded}
\begin{Highlighting}[]
\FunctionTok{freq}\NormalTok{(Base\_Trajipaq\_Finale\_Ok}\SpecialCharTok{$}\NormalTok{Q4)}
\end{Highlighting}
\end{Shaded}

\begin{verbatim}
## Frequencies  
## Base_Trajipaq_Finale_Ok$Q4  
## Label: Êtes-vous né(e) au Canada ?  
## Type: Numeric  
## 
##               Freq   % Valid   % Valid Cum.   % Total   % Total Cum.
## ----------- ------ --------- -------------- --------- --------------
##           1   1113     71.21          71.21     71.21          71.21
##           2    450     28.79         100.00     28.79         100.00
##        <NA>      0                               0.00         100.00
##       Total   1563    100.00         100.00    100.00         100.00
\end{verbatim}

\begin{Shaded}
\begin{Highlighting}[]
\FunctionTok{ggplot}\NormalTok{(}\AttributeTok{data =}\NormalTok{ Base\_Trajipaq\_Finale\_Ok, }\AttributeTok{mapping =} \FunctionTok{aes}\NormalTok{(}\AttributeTok{x =}\NormalTok{ Q4)) }\SpecialCharTok{+}
  \FunctionTok{geom\_bar}\NormalTok{(}\AttributeTok{fill =} \StringTok{"pink"}\NormalTok{, }\AttributeTok{na.rm =} \ConstantTok{TRUE}\NormalTok{) }\SpecialCharTok{+}
  \FunctionTok{theme\_bw}\NormalTok{()}
\end{Highlighting}
\end{Shaded}

\includegraphics{Article---Sentiment-d-appartenance_files/figure-latex/unnamed-chunk-13-1.pdf}

La réponse modale est ``deuxième génération'' (92.87\%).

\hypertarget{q15i.-statut-migratoire-avant-dobtenir-la-ruxe9sidence-permanente}{%
\subsection{Q15I. Statut migratoire avant d'obtenir la résidence
permanente}\label{q15i.-statut-migratoire-avant-dobtenir-la-ruxe9sidence-permanente}}

\begin{enumerate}
\def\labelenumi{\arabic{enumi}.}
\tightlist
\item
  Travailleur temporaire (visa de travail ou permis de travail)
\item
  Demandeur d'asile
\item
  Étudiant étranger (visa d'études)
\item
  Touriste
\item
  Autre
\item
  Je ne sais pas
\item
  Je préfère ne pas répondre
\end{enumerate}

Type de variable = catégoriel Niveau de mesure = nominal

\begin{Shaded}
\begin{Highlighting}[]
\NormalTok{Base\_Trajipaq\_Finale\_Ok }\OtherTok{\textless{}{-}}
\NormalTok{  Base\_Trajipaq\_Finale\_Ok }\SpecialCharTok{\%\textgreater{}\%}
  \FunctionTok{mutate}\NormalTok{(}\AttributeTok{Q15M1 =} \FunctionTok{case\_when}\NormalTok{(}
\NormalTok{    Q15M1 }\SpecialCharTok{==} \DecValTok{1} \SpecialCharTok{\textasciitilde{}} \DecValTok{1}\NormalTok{,}
\NormalTok{    Q15M1 }\SpecialCharTok{==} \DecValTok{2} \SpecialCharTok{\textasciitilde{}} \DecValTok{2}\NormalTok{,}
\NormalTok{    Q15M1 }\SpecialCharTok{==} \DecValTok{3} \SpecialCharTok{\textasciitilde{}} \DecValTok{3}\NormalTok{,}
\NormalTok{    Q15M1 }\SpecialCharTok{==} \DecValTok{4} \SpecialCharTok{\textasciitilde{}} \DecValTok{4}\NormalTok{,}
\NormalTok{    Q15M1 }\SpecialCharTok{==} \DecValTok{97} \SpecialCharTok{\textasciitilde{}} \DecValTok{5}\NormalTok{,}
\NormalTok{    Q15M1 }\SpecialCharTok{==} \DecValTok{98} \SpecialCharTok{\textasciitilde{}} \ConstantTok{NA}\NormalTok{,}
\NormalTok{    Q15M1 }\SpecialCharTok{==} \DecValTok{99} \SpecialCharTok{\textasciitilde{}} \ConstantTok{NA}
\NormalTok{  ))}
\end{Highlighting}
\end{Shaded}

\begin{Shaded}
\begin{Highlighting}[]
\FunctionTok{freq}\NormalTok{(Base\_Trajipaq\_Finale\_Ok}\SpecialCharTok{$}\NormalTok{Q15M1)}
\end{Highlighting}
\end{Shaded}

\begin{verbatim}
## Frequencies  
## Base_Trajipaq_Finale_Ok$Q15M1  
## Type: Numeric  
## 
##               Freq   % Valid   % Valid Cum.   % Total   % Total Cum.
## ----------- ------ --------- -------------- --------- --------------
##           1    241     57.79          57.79     15.42          15.42
##           2     14      3.36          61.15      0.90          16.31
##           3     82     19.66          80.82      5.25          21.56
##           4     35      8.39          89.21      2.24          23.80
##           5     45     10.79         100.00      2.88          26.68
##        <NA>   1146                              73.32         100.00
##       Total   1563    100.00         100.00    100.00         100.00
\end{verbatim}

\begin{Shaded}
\begin{Highlighting}[]
\FunctionTok{ggplot}\NormalTok{(}\AttributeTok{data =}\NormalTok{ Base\_Trajipaq\_Finale\_Ok, }\AttributeTok{mapping =} \FunctionTok{aes}\NormalTok{(}\AttributeTok{x =}\NormalTok{ Q15M1)) }\SpecialCharTok{+}
  \FunctionTok{geom\_bar}\NormalTok{(}\AttributeTok{fill =} \StringTok{"pink"}\NormalTok{, }\AttributeTok{na.rm =} \ConstantTok{TRUE}\NormalTok{) }\SpecialCharTok{+}
  \FunctionTok{theme\_bw}\NormalTok{()}
\end{Highlighting}
\end{Shaded}

\includegraphics{Article---Sentiment-d-appartenance_files/figure-latex/unnamed-chunk-16-1.pdf}

La réponse modale est ``travailleur temporaire'' (57.78\%).

\hypertarget{q16.-quel-est-votre-plus-haut-dipluxf4me-duxe9tudes-obtenu}{%
\subsection{Q16. Quel est votre plus haut diplôme d'études obtenu
?}\label{q16.-quel-est-votre-plus-haut-dipluxf4me-duxe9tudes-obtenu}}

\begin{enumerate}
\def\labelenumi{\arabic{enumi}.}
\tightlist
\item
  Moins que le secondaire
\item
  Secondaire ou équivalent
\item
  Collégial
\item
  Formation professionnelle ou post-secondaire
\item
  Université 1er cycle
\item
  Université 2ème cycle et plus
\item
  Je ne sais pas
\item
  Je préfère ne pas répondre
\end{enumerate}

Type de variable = factorielle Niveau de mesure = ordinal

\begin{Shaded}
\begin{Highlighting}[]
\NormalTok{Base\_Trajipaq\_Finale\_Ok }\OtherTok{\textless{}{-}}
\NormalTok{  Base\_Trajipaq\_Finale\_Ok }\SpecialCharTok{\%\textgreater{}\%}
  \FunctionTok{mutate}\NormalTok{(}\AttributeTok{Q16 =} \FunctionTok{case\_when}\NormalTok{(}
\NormalTok{    Q16 }\SpecialCharTok{==} \DecValTok{1} \SpecialCharTok{\textasciitilde{}} \DecValTok{1}\NormalTok{,}
\NormalTok{    Q16 }\SpecialCharTok{==} \DecValTok{2} \SpecialCharTok{\textasciitilde{}} \DecValTok{2}\NormalTok{,}
\NormalTok{    Q16 }\SpecialCharTok{==} \DecValTok{3} \SpecialCharTok{\textasciitilde{}} \DecValTok{3}\NormalTok{,}
\NormalTok{    Q16 }\SpecialCharTok{==} \DecValTok{4} \SpecialCharTok{\textasciitilde{}} \DecValTok{4}\NormalTok{,}
\NormalTok{    Q16 }\SpecialCharTok{==} \DecValTok{5} \SpecialCharTok{\textasciitilde{}} \DecValTok{5}\NormalTok{,}
\NormalTok{    Q16 }\SpecialCharTok{==} \DecValTok{6} \SpecialCharTok{\textasciitilde{}} \DecValTok{6}\NormalTok{,}
\NormalTok{    Q16 }\SpecialCharTok{==} \DecValTok{98} \SpecialCharTok{\textasciitilde{}} \ConstantTok{NA}\NormalTok{,}
\NormalTok{    Q16 }\SpecialCharTok{==} \DecValTok{99} \SpecialCharTok{\textasciitilde{}} \ConstantTok{NA}
\NormalTok{  ))}
\end{Highlighting}
\end{Shaded}

\begin{Shaded}
\begin{Highlighting}[]
\FunctionTok{freq}\NormalTok{(Base\_Trajipaq\_Finale\_Ok}\SpecialCharTok{$}\NormalTok{Q16)}
\end{Highlighting}
\end{Shaded}

\begin{verbatim}
## Frequencies  
## Base_Trajipaq_Finale_Ok$Q16  
## Type: Numeric  
## 
##               Freq   % Valid   % Valid Cum.   % Total   % Total Cum.
## ----------- ------ --------- -------------- --------- --------------
##           1     11      0.71           0.71      0.70           0.70
##           2    101      6.51           7.22      6.46           7.17
##           3    254     16.38          23.60     16.25          23.42
##           4    141      9.09          32.69      9.02          32.44
##           5    520     33.53          66.22     33.27          65.71
##           6    524     33.78         100.00     33.53          99.23
##        <NA>     12                               0.77         100.00
##       Total   1563    100.00         100.00    100.00         100.00
\end{verbatim}

\begin{Shaded}
\begin{Highlighting}[]
\FunctionTok{ggplot}\NormalTok{(}\AttributeTok{data =}\NormalTok{ Base\_Trajipaq\_Finale\_Ok, }\AttributeTok{mapping =} \FunctionTok{aes}\NormalTok{(}\AttributeTok{x =}\NormalTok{ Q16)) }\SpecialCharTok{+}
  \FunctionTok{geom\_bar}\NormalTok{(}\AttributeTok{fill =} \StringTok{"pink"}\NormalTok{, }\AttributeTok{na.rm =} \ConstantTok{TRUE}\NormalTok{) }\SpecialCharTok{+}
  \FunctionTok{theme\_bw}\NormalTok{()}
\end{Highlighting}
\end{Shaded}

\includegraphics{Article---Sentiment-d-appartenance_files/figure-latex/unnamed-chunk-19-1.pdf}

La distribution du niveau d'éducation est asymétrique vers la gauche.
L'échantillon est composé à 67.31\% de personnes titulaires d'un diplôme
universitaire.

\hypertarget{q17.-actuellement-avez-vous-une-religion}{%
\subsection{Q17. Actuellement, avez-vous une religion
?}\label{q17.-actuellement-avez-vous-une-religion}}

\begin{enumerate}
\def\labelenumi{\arabic{enumi}.}
\setcounter{enumi}{-1}
\tightlist
\item
  Sans religion
\item
  Catholique
\item
  Protestante
\item
  Évangéliste
\item
  Mormone
\item
  Témoins de Jéhovah
\item
  Orthodoxe
\item
  Autres Chrétiens
\item
  Musulmans
\item
  Juif
\item
  Bouddhiste
\item
  Hindou
\item
  Animiste
\item
  Autre
\item
  Je ne sais pas
\item
  Je préfère ne pas répondre
\end{enumerate}

Type de variable = catégorielle Niveau de mesure = nominal

\begin{Shaded}
\begin{Highlighting}[]
\NormalTok{Base\_Trajipaq\_Finale\_Ok }\OtherTok{\textless{}{-}}
\NormalTok{  Base\_Trajipaq\_Finale\_Ok }\SpecialCharTok{\%\textgreater{}\%}
  \FunctionTok{mutate}\NormalTok{(}\AttributeTok{Q17 =} \FunctionTok{case\_when}\NormalTok{(}
\NormalTok{    Q17 }\SpecialCharTok{==} \DecValTok{0} \SpecialCharTok{\textasciitilde{}} \DecValTok{0}\NormalTok{,}
\NormalTok{    Q17 }\SpecialCharTok{==} \DecValTok{1} \SpecialCharTok{\textasciitilde{}} \DecValTok{1}\NormalTok{,}
\NormalTok{    Q17 }\SpecialCharTok{==} \DecValTok{2} \SpecialCharTok{\textasciitilde{}} \DecValTok{2}\NormalTok{,}
\NormalTok{    Q17 }\SpecialCharTok{==} \DecValTok{3} \SpecialCharTok{\textasciitilde{}} \DecValTok{3}\NormalTok{,}
\NormalTok{    Q17 }\SpecialCharTok{==} \DecValTok{4} \SpecialCharTok{\textasciitilde{}} \DecValTok{4}\NormalTok{,}
\NormalTok{    Q17 }\SpecialCharTok{==} \DecValTok{5} \SpecialCharTok{\textasciitilde{}} \DecValTok{5}\NormalTok{,}
\NormalTok{    Q17 }\SpecialCharTok{==} \DecValTok{6} \SpecialCharTok{\textasciitilde{}} \DecValTok{6}\NormalTok{,}
\NormalTok{    Q17 }\SpecialCharTok{==} \DecValTok{7} \SpecialCharTok{\textasciitilde{}} \DecValTok{7}\NormalTok{,}
\NormalTok{    Q17 }\SpecialCharTok{==} \DecValTok{8} \SpecialCharTok{\textasciitilde{}} \DecValTok{8}\NormalTok{,}
\NormalTok{    Q17 }\SpecialCharTok{==} \DecValTok{9} \SpecialCharTok{\textasciitilde{}} \DecValTok{9}\NormalTok{,}
\NormalTok{    Q17 }\SpecialCharTok{==} \DecValTok{10} \SpecialCharTok{\textasciitilde{}} \DecValTok{10}\NormalTok{,}
\NormalTok{    Q17 }\SpecialCharTok{==} \DecValTok{11} \SpecialCharTok{\textasciitilde{}} \DecValTok{11}\NormalTok{,}
\NormalTok{    Q17 }\SpecialCharTok{==} \DecValTok{12} \SpecialCharTok{\textasciitilde{}} \DecValTok{12}\NormalTok{,}
\NormalTok{    Q17 }\SpecialCharTok{==} \DecValTok{97} \SpecialCharTok{\textasciitilde{}} \DecValTok{13}\NormalTok{,}
\NormalTok{    Q17 }\SpecialCharTok{==} \DecValTok{98} \SpecialCharTok{\textasciitilde{}} \ConstantTok{NA}\NormalTok{,}
\NormalTok{    Q17 }\SpecialCharTok{==} \DecValTok{99} \SpecialCharTok{\textasciitilde{}} \ConstantTok{NA}
\NormalTok{  ))}
\end{Highlighting}
\end{Shaded}

\begin{Shaded}
\begin{Highlighting}[]
\FunctionTok{freq}\NormalTok{(Base\_Trajipaq\_Finale\_Ok}\SpecialCharTok{$}\NormalTok{Q17)}
\end{Highlighting}
\end{Shaded}

\begin{verbatim}
## Frequencies  
## Base_Trajipaq_Finale_Ok$Q17  
## Type: Numeric  
## 
##               Freq   % Valid   % Valid Cum.   % Total   % Total Cum.
## ----------- ------ --------- -------------- --------- --------------
##           0    552     36.36          36.36     35.32          35.32
##           1    489     32.21          68.58     31.29          66.60
##           2     42      2.77          71.34      2.69          69.29
##           3     23      1.52          72.86      1.47          70.76
##           4      2      0.13          72.99      0.13          70.89
##           5      5      0.33          73.32      0.32          71.21
##           6     89      5.86          79.18      5.69          76.90
##           7     17      1.12          80.30      1.09          77.99
##           8    210     13.83          94.14     13.44          91.43
##           9     18      1.19          95.32      1.15          92.58
##          10     22      1.45          96.77      1.41          93.99
##          11     11      0.72          97.50      0.70          94.69
##          12      2      0.13          97.63      0.13          94.82
##          13     36      2.37         100.00      2.30          97.12
##        <NA>     45                               2.88         100.00
##       Total   1563    100.00         100.00    100.00         100.00
\end{verbatim}

\begin{Shaded}
\begin{Highlighting}[]
\FunctionTok{ggplot}\NormalTok{(}\AttributeTok{data =}\NormalTok{ Base\_Trajipaq\_Finale\_Ok, }\AttributeTok{mapping =} \FunctionTok{aes}\NormalTok{(}\AttributeTok{x =}\NormalTok{ Q17)) }\SpecialCharTok{+}
  \FunctionTok{geom\_bar}\NormalTok{(}\AttributeTok{fill =} \StringTok{"pink"}\NormalTok{, }\AttributeTok{na.rm =} \ConstantTok{TRUE}\NormalTok{) }\SpecialCharTok{+}
  \FunctionTok{theme\_bw}\NormalTok{()}
\end{Highlighting}
\end{Shaded}

\includegraphics{Article---Sentiment-d-appartenance_files/figure-latex/unnamed-chunk-22-1.pdf}

Les réponses modales sont ``sans religion'' (36.36\%) et ``catholique''
(32.21\%). On remarque également un nombre significatif de réponses
``musulman'' (13.83\%).

\hypertarget{q20.-quand-vous-uxe9tiez-enfant-quelles-uxe9taient-la-langue-ou-les-langues-que-vous-parliez-le-plus-souvent-uxe0-luxe2ge-de-5-ans-et-que-vous-comprenez-toujours-plusieurs-ruxe9ponses-possibles.}{%
\subsection{Q20. Quand vous étiez enfant, quelles étaient la langue ou
les langues que vous parliez le plus souvent, à l'âge de 5 ans et que
vous comprenez toujours? (Plusieurs réponses
possibles).}\label{q20.-quand-vous-uxe9tiez-enfant-quelles-uxe9taient-la-langue-ou-les-langues-que-vous-parliez-le-plus-souvent-uxe0-luxe2ge-de-5-ans-et-que-vous-comprenez-toujours-plusieurs-ruxe9ponses-possibles.}}

\begin{enumerate}
\def\labelenumi{\arabic{enumi}.}
\tightlist
\item
  Français
\item
  Anglais
\item
  Autre : préciser, possibilité d'en déclarer deux
\item
  Je ne sais pas
\item
  Je préfère ne pas répondre
\end{enumerate}

Type de variable = catégoriel Niveau de mesure = nominal

\begin{Shaded}
\begin{Highlighting}[]
\NormalTok{Base\_Trajipaq\_Finale\_Ok }\OtherTok{\textless{}{-}}
\NormalTok{  Base\_Trajipaq\_Finale\_Ok }\SpecialCharTok{\%\textgreater{}\%}
  \FunctionTok{mutate}\NormalTok{(}\AttributeTok{Q20M1 =} \FunctionTok{case\_when}\NormalTok{(}
\NormalTok{    Q20M1 }\SpecialCharTok{==} \DecValTok{1} \SpecialCharTok{\textasciitilde{}} \DecValTok{1}\NormalTok{,}
\NormalTok{    Q20M1 }\SpecialCharTok{==} \DecValTok{2} \SpecialCharTok{\textasciitilde{}} \DecValTok{2}\NormalTok{,}
\NormalTok{    Q20M1 }\SpecialCharTok{==} \DecValTok{97} \SpecialCharTok{\textasciitilde{}} \DecValTok{3}\NormalTok{,}
\NormalTok{    Q20M1 }\SpecialCharTok{==} \DecValTok{98} \SpecialCharTok{\textasciitilde{}} \ConstantTok{NA}\NormalTok{,}
\NormalTok{    Q20M1 }\SpecialCharTok{==} \DecValTok{99} \SpecialCharTok{\textasciitilde{}} \ConstantTok{NA}
\NormalTok{  ))}
\end{Highlighting}
\end{Shaded}

\begin{Shaded}
\begin{Highlighting}[]
\FunctionTok{freq}\NormalTok{(Base\_Trajipaq\_Finale\_Ok}\SpecialCharTok{$}\NormalTok{Q20M1)}
\end{Highlighting}
\end{Shaded}

\begin{verbatim}
## Frequencies  
## Base_Trajipaq_Finale_Ok$Q20M1  
## Type: Numeric  
## 
##               Freq   % Valid   % Valid Cum.   % Total   % Total Cum.
## ----------- ------ --------- -------------- --------- --------------
##           1    987     63.84          63.84     63.15          63.15
##           2    129      8.34          72.19      8.25          71.40
##           3    430     27.81         100.00     27.51          98.91
##        <NA>     17                               1.09         100.00
##       Total   1563    100.00         100.00    100.00         100.00
\end{verbatim}

\begin{Shaded}
\begin{Highlighting}[]
\FunctionTok{ggplot}\NormalTok{(}\AttributeTok{data =}\NormalTok{ Base\_Trajipaq\_Finale\_Ok, }\AttributeTok{mapping =} \FunctionTok{aes}\NormalTok{(}\AttributeTok{x =}\NormalTok{ Q20M1)) }\SpecialCharTok{+}
  \FunctionTok{geom\_bar}\NormalTok{(}\AttributeTok{fill =} \StringTok{"pink"}\NormalTok{, }\AttributeTok{na.rm =} \ConstantTok{TRUE}\NormalTok{) }\SpecialCharTok{+}
  \FunctionTok{theme\_bw}\NormalTok{()}
\end{Highlighting}
\end{Shaded}

\includegraphics{Article---Sentiment-d-appartenance_files/figure-latex/unnamed-chunk-25-1.pdf}

La réponse modale est ``français'' (63.84\%).

\hypertarget{i3.-indice-de-la-compuxe9tence-linguistique-franuxe7ais}{%
\subsection{I3. Indice de la compétence linguistique
(français)}\label{i3.-indice-de-la-compuxe9tence-linguistique-franuxe7ais}}

Type de variable = numeric Niveau de mesure = ratio

\begin{Shaded}
\begin{Highlighting}[]
\NormalTok{Base\_Trajipaq\_Finale\_Ok }\OtherTok{\textless{}{-}}
\NormalTok{  Base\_Trajipaq\_Finale\_Ok }\SpecialCharTok{\%\textgreater{}\%}
  \FunctionTok{mutate}\NormalTok{(}\AttributeTok{Q21A =} \FunctionTok{case\_when}\NormalTok{(}
\NormalTok{    Q21A }\SpecialCharTok{==} \DecValTok{1} \SpecialCharTok{\textasciitilde{}} \DecValTok{1}\NormalTok{,}
\NormalTok{    Q21A }\SpecialCharTok{==} \DecValTok{2} \SpecialCharTok{\textasciitilde{}} \DecValTok{2}\NormalTok{,}
\NormalTok{    Q21A }\SpecialCharTok{==} \DecValTok{3} \SpecialCharTok{\textasciitilde{}} \DecValTok{3}\NormalTok{,}
\NormalTok{    Q21A }\SpecialCharTok{==} \DecValTok{4} \SpecialCharTok{\textasciitilde{}} \DecValTok{4}\NormalTok{,}
\NormalTok{    Q21A }\SpecialCharTok{==} \DecValTok{98} \SpecialCharTok{\textasciitilde{}} \ConstantTok{NA}\NormalTok{,}
\NormalTok{    Q21A }\SpecialCharTok{==} \DecValTok{99} \SpecialCharTok{\textasciitilde{}} \ConstantTok{NA}
\NormalTok{  ))}

\NormalTok{Base\_Trajipaq\_Finale\_Ok }\OtherTok{\textless{}{-}}
\NormalTok{  Base\_Trajipaq\_Finale\_Ok }\SpecialCharTok{\%\textgreater{}\%}
  \FunctionTok{mutate}\NormalTok{(}\AttributeTok{Q21B =} \FunctionTok{case\_when}\NormalTok{(}
\NormalTok{    Q21B }\SpecialCharTok{==} \DecValTok{1} \SpecialCharTok{\textasciitilde{}} \DecValTok{1}\NormalTok{,}
\NormalTok{    Q21B }\SpecialCharTok{==} \DecValTok{2} \SpecialCharTok{\textasciitilde{}} \DecValTok{2}\NormalTok{,}
\NormalTok{    Q21B }\SpecialCharTok{==} \DecValTok{3} \SpecialCharTok{\textasciitilde{}} \DecValTok{3}\NormalTok{,}
\NormalTok{    Q21B }\SpecialCharTok{==} \DecValTok{4} \SpecialCharTok{\textasciitilde{}} \DecValTok{4}\NormalTok{,}
\NormalTok{    Q21B }\SpecialCharTok{==} \DecValTok{98} \SpecialCharTok{\textasciitilde{}} \ConstantTok{NA}\NormalTok{,}
\NormalTok{    Q21B }\SpecialCharTok{==} \DecValTok{99} \SpecialCharTok{\textasciitilde{}} \ConstantTok{NA}
\NormalTok{  ))}

\NormalTok{Base\_Trajipaq\_Finale\_Ok }\OtherTok{\textless{}{-}}
\NormalTok{  Base\_Trajipaq\_Finale\_Ok }\SpecialCharTok{\%\textgreater{}\%}
  \FunctionTok{mutate}\NormalTok{(}\AttributeTok{Q21C =} \FunctionTok{case\_when}\NormalTok{(}
\NormalTok{    Q21C }\SpecialCharTok{==} \DecValTok{1} \SpecialCharTok{\textasciitilde{}} \DecValTok{1}\NormalTok{,}
\NormalTok{    Q21C }\SpecialCharTok{==} \DecValTok{2} \SpecialCharTok{\textasciitilde{}} \DecValTok{2}\NormalTok{,}
\NormalTok{    Q21C }\SpecialCharTok{==} \DecValTok{3} \SpecialCharTok{\textasciitilde{}} \DecValTok{3}\NormalTok{,}
\NormalTok{    Q21C }\SpecialCharTok{==} \DecValTok{4} \SpecialCharTok{\textasciitilde{}} \DecValTok{4}\NormalTok{,}
\NormalTok{    Q21C }\SpecialCharTok{==} \DecValTok{98} \SpecialCharTok{\textasciitilde{}} \ConstantTok{NA}\NormalTok{,}
\NormalTok{    Q21C }\SpecialCharTok{==} \DecValTok{99} \SpecialCharTok{\textasciitilde{}} \ConstantTok{NA}
\NormalTok{  ))}

\NormalTok{Base\_Trajipaq\_Finale\_Ok }\OtherTok{\textless{}{-}}
\NormalTok{  Base\_Trajipaq\_Finale\_Ok }\SpecialCharTok{\%\textgreater{}\%}
  \FunctionTok{mutate}\NormalTok{(}\AttributeTok{Q21D =} \FunctionTok{case\_when}\NormalTok{(}
\NormalTok{    Q21D }\SpecialCharTok{==} \DecValTok{1} \SpecialCharTok{\textasciitilde{}} \DecValTok{1}\NormalTok{,}
\NormalTok{    Q21D }\SpecialCharTok{==} \DecValTok{2} \SpecialCharTok{\textasciitilde{}} \DecValTok{2}\NormalTok{,}
\NormalTok{    Q21D }\SpecialCharTok{==} \DecValTok{3} \SpecialCharTok{\textasciitilde{}} \DecValTok{3}\NormalTok{,}
\NormalTok{    Q21D }\SpecialCharTok{==} \DecValTok{4} \SpecialCharTok{\textasciitilde{}} \DecValTok{4}\NormalTok{,}
\NormalTok{    Q21D }\SpecialCharTok{==} \DecValTok{98} \SpecialCharTok{\textasciitilde{}} \ConstantTok{NA}\NormalTok{,}
\NormalTok{    Q21D }\SpecialCharTok{==} \DecValTok{99} \SpecialCharTok{\textasciitilde{}} \ConstantTok{NA}
\NormalTok{  ))}
\end{Highlighting}
\end{Shaded}

\begin{Shaded}
\begin{Highlighting}[]
\NormalTok{v\_I3 }\OtherTok{\textless{}{-}} \FunctionTok{c}\NormalTok{(}\StringTok{"Q21A"}\NormalTok{,}
          \StringTok{"Q21B"}\NormalTok{,}
          \StringTok{"Q21C"}\NormalTok{,}
          \StringTok{"Q21D"}
\NormalTok{          )}

\NormalTok{Base\_Trajipaq\_Finale\_Ok }\OtherTok{\textless{}{-}}
\NormalTok{  Base\_Trajipaq\_Finale\_Ok }\SpecialCharTok{\%\textgreater{}\%}
  \FunctionTok{rowwise}\NormalTok{() }\SpecialCharTok{\%\textgreater{}\%}
  \FunctionTok{mutate}\NormalTok{(}\AttributeTok{I3 =}\NormalTok{ (}\FunctionTok{sum}\NormalTok{(}\FunctionTok{c\_across}\NormalTok{(}\FunctionTok{all\_of}\NormalTok{(v\_I3)), }\AttributeTok{na.rm =} \ConstantTok{TRUE}\NormalTok{) }\SpecialCharTok{{-}} \FunctionTok{length}\NormalTok{(v\_I3)) }\SpecialCharTok{/}\NormalTok{ (}\DecValTok{3}\SpecialCharTok{*}\FunctionTok{length}\NormalTok{(v\_I3))) }\SpecialCharTok{\%\textgreater{}\%}
  \FunctionTok{mutate}\NormalTok{(}\AttributeTok{I3 =} \ControlFlowTok{if}\NormalTok{ (I3 }\SpecialCharTok{\textless{}} \DecValTok{0}\NormalTok{) }\ConstantTok{NA\_real\_} \ControlFlowTok{else}\NormalTok{ I3) }\SpecialCharTok{\%\textgreater{}\%}
  \FunctionTok{ungroup}\NormalTok{()}
\end{Highlighting}
\end{Shaded}

\begin{Shaded}
\begin{Highlighting}[]
\FunctionTok{descr}\NormalTok{(Base\_Trajipaq\_Finale\_Ok}\SpecialCharTok{$}\NormalTok{I3)}
\end{Highlighting}
\end{Shaded}

\begin{verbatim}
## Descriptive Statistics  
## Base_Trajipaq_Finale_Ok$I3  
## N: 1563  
## 
##                          I3
## ----------------- ---------
##              Mean      0.87
##           Std.Dev      0.24
##               Min      0.00
##                Q1      0.83
##            Median      1.00
##                Q3      1.00
##               Max      1.00
##               MAD      0.00
##               IQR      0.17
##                CV      0.28
##          Skewness     -1.98
##       SE.Skewness      0.07
##          Kurtosis      3.23
##           N.Valid   1117.00
##         Pct.Valid     71.47
\end{verbatim}

\begin{Shaded}
\begin{Highlighting}[]
\FunctionTok{ggplot}\NormalTok{(}\AttributeTok{data =}\NormalTok{ Base\_Trajipaq\_Finale\_Ok, }\AttributeTok{mapping =} \FunctionTok{aes}\NormalTok{(}\AttributeTok{x =}\NormalTok{ I3)) }\SpecialCharTok{+}
  \FunctionTok{geom\_histogram}\NormalTok{(}\AttributeTok{na.rm =} \ConstantTok{TRUE}\NormalTok{, }\AttributeTok{fill =} \StringTok{"pink"}\NormalTok{, }\AttributeTok{binwidth =} \FloatTok{0.01}\NormalTok{) }\SpecialCharTok{+}
  \FunctionTok{theme\_bw}\NormalTok{()}
\end{Highlighting}
\end{Shaded}

\includegraphics{Article---Sentiment-d-appartenance_files/figure-latex/unnamed-chunk-29-1.pdf}

La distribution est asymétrique vers la gauche. L'indice moyen de
maîtrise du français est 0.87.

\hypertarget{i4.-indice-de-la-compuxe9tence-linguistique-anglais.}{%
\subsection{I4. Indice de la compétence linguistique
(anglais).}\label{i4.-indice-de-la-compuxe9tence-linguistique-anglais.}}

Type de variable = numeric Niveau de mesure = ratio

\begin{Shaded}
\begin{Highlighting}[]
\NormalTok{Base\_Trajipaq\_Finale\_Ok }\OtherTok{\textless{}{-}}
\NormalTok{  Base\_Trajipaq\_Finale\_Ok }\SpecialCharTok{\%\textgreater{}\%}
  \FunctionTok{mutate}\NormalTok{(}\AttributeTok{Q22A =} \FunctionTok{case\_when}\NormalTok{(}
\NormalTok{    Q22A }\SpecialCharTok{==} \DecValTok{1} \SpecialCharTok{\textasciitilde{}} \DecValTok{1}\NormalTok{,}
\NormalTok{    Q22A }\SpecialCharTok{==} \DecValTok{2} \SpecialCharTok{\textasciitilde{}} \DecValTok{2}\NormalTok{,}
\NormalTok{    Q22A }\SpecialCharTok{==} \DecValTok{3} \SpecialCharTok{\textasciitilde{}} \DecValTok{3}\NormalTok{,}
\NormalTok{    Q22A }\SpecialCharTok{==} \DecValTok{4} \SpecialCharTok{\textasciitilde{}} \DecValTok{4}\NormalTok{,}
\NormalTok{    Q22A }\SpecialCharTok{==} \DecValTok{98} \SpecialCharTok{\textasciitilde{}} \ConstantTok{NA}\NormalTok{,}
\NormalTok{    Q22A }\SpecialCharTok{==} \DecValTok{99} \SpecialCharTok{\textasciitilde{}} \ConstantTok{NA}
\NormalTok{  ))}

\NormalTok{Base\_Trajipaq\_Finale\_Ok }\OtherTok{\textless{}{-}}
\NormalTok{  Base\_Trajipaq\_Finale\_Ok }\SpecialCharTok{\%\textgreater{}\%}
  \FunctionTok{mutate}\NormalTok{(}\AttributeTok{Q22B =} \FunctionTok{case\_when}\NormalTok{(}
\NormalTok{    Q22B }\SpecialCharTok{==} \DecValTok{1} \SpecialCharTok{\textasciitilde{}} \DecValTok{1}\NormalTok{,}
\NormalTok{    Q22B }\SpecialCharTok{==} \DecValTok{2} \SpecialCharTok{\textasciitilde{}} \DecValTok{2}\NormalTok{,}
\NormalTok{    Q22B }\SpecialCharTok{==} \DecValTok{3} \SpecialCharTok{\textasciitilde{}} \DecValTok{3}\NormalTok{,}
\NormalTok{    Q22B }\SpecialCharTok{==} \DecValTok{4} \SpecialCharTok{\textasciitilde{}} \DecValTok{4}\NormalTok{,}
\NormalTok{    Q22B }\SpecialCharTok{==} \DecValTok{98} \SpecialCharTok{\textasciitilde{}} \ConstantTok{NA}\NormalTok{,}
\NormalTok{    Q22B }\SpecialCharTok{==} \DecValTok{99} \SpecialCharTok{\textasciitilde{}} \ConstantTok{NA}
\NormalTok{  ))}

\NormalTok{Base\_Trajipaq\_Finale\_Ok }\OtherTok{\textless{}{-}}
\NormalTok{  Base\_Trajipaq\_Finale\_Ok }\SpecialCharTok{\%\textgreater{}\%}
  \FunctionTok{mutate}\NormalTok{(}\AttributeTok{Q22C =} \FunctionTok{case\_when}\NormalTok{(}
\NormalTok{    Q22C }\SpecialCharTok{==} \DecValTok{1} \SpecialCharTok{\textasciitilde{}} \DecValTok{1}\NormalTok{,}
\NormalTok{    Q22C }\SpecialCharTok{==} \DecValTok{2} \SpecialCharTok{\textasciitilde{}} \DecValTok{2}\NormalTok{,}
\NormalTok{    Q22C }\SpecialCharTok{==} \DecValTok{3} \SpecialCharTok{\textasciitilde{}} \DecValTok{3}\NormalTok{,}
\NormalTok{    Q22C }\SpecialCharTok{==} \DecValTok{4} \SpecialCharTok{\textasciitilde{}} \DecValTok{4}\NormalTok{,}
\NormalTok{    Q22C }\SpecialCharTok{==} \DecValTok{98} \SpecialCharTok{\textasciitilde{}} \ConstantTok{NA}\NormalTok{,}
\NormalTok{    Q22C }\SpecialCharTok{==} \DecValTok{99} \SpecialCharTok{\textasciitilde{}} \ConstantTok{NA}
\NormalTok{  ))}

\NormalTok{Base\_Trajipaq\_Finale\_Ok }\OtherTok{\textless{}{-}}
\NormalTok{  Base\_Trajipaq\_Finale\_Ok }\SpecialCharTok{\%\textgreater{}\%}
  \FunctionTok{mutate}\NormalTok{(}\AttributeTok{Q22D =} \FunctionTok{case\_when}\NormalTok{(}
\NormalTok{    Q22D }\SpecialCharTok{==} \DecValTok{1} \SpecialCharTok{\textasciitilde{}} \DecValTok{1}\NormalTok{,}
\NormalTok{    Q22D }\SpecialCharTok{==} \DecValTok{2} \SpecialCharTok{\textasciitilde{}} \DecValTok{2}\NormalTok{,}
\NormalTok{    Q22D }\SpecialCharTok{==} \DecValTok{3} \SpecialCharTok{\textasciitilde{}} \DecValTok{3}\NormalTok{,}
\NormalTok{    Q22D }\SpecialCharTok{==} \DecValTok{4} \SpecialCharTok{\textasciitilde{}} \DecValTok{4}\NormalTok{,}
\NormalTok{    Q22D }\SpecialCharTok{==} \DecValTok{98} \SpecialCharTok{\textasciitilde{}} \ConstantTok{NA}\NormalTok{,}
\NormalTok{    Q22D }\SpecialCharTok{==} \DecValTok{99} \SpecialCharTok{\textasciitilde{}} \ConstantTok{NA}
\NormalTok{  ))}
\end{Highlighting}
\end{Shaded}

\begin{Shaded}
\begin{Highlighting}[]
\NormalTok{v\_I4 }\OtherTok{\textless{}{-}} \FunctionTok{c}\NormalTok{(}\StringTok{"Q22A"}\NormalTok{,}
          \StringTok{"Q22B"}\NormalTok{,}
          \StringTok{"Q22C"}\NormalTok{,}
          \StringTok{"Q22D"}
\NormalTok{          )}

\NormalTok{Base\_Trajipaq\_Finale\_Ok }\OtherTok{\textless{}{-}}
\NormalTok{  Base\_Trajipaq\_Finale\_Ok }\SpecialCharTok{\%\textgreater{}\%}
  \FunctionTok{rowwise}\NormalTok{() }\SpecialCharTok{\%\textgreater{}\%}
  \FunctionTok{mutate}\NormalTok{(}\AttributeTok{I4 =}\NormalTok{ (}\FunctionTok{sum}\NormalTok{(}\FunctionTok{c\_across}\NormalTok{(}\FunctionTok{all\_of}\NormalTok{(v\_I4)), }\AttributeTok{na.rm =} \ConstantTok{TRUE}\NormalTok{) }\SpecialCharTok{{-}} \FunctionTok{length}\NormalTok{(v\_I4)) }\SpecialCharTok{/}\NormalTok{ (}\DecValTok{3}\SpecialCharTok{*}\FunctionTok{length}\NormalTok{(v\_I4))) }\SpecialCharTok{\%\textgreater{}\%}
  \FunctionTok{mutate}\NormalTok{(}\AttributeTok{I4 =} \ControlFlowTok{if}\NormalTok{ (I4 }\SpecialCharTok{\textless{}} \DecValTok{0}\NormalTok{) }\ConstantTok{NA\_real\_} \ControlFlowTok{else}\NormalTok{ I4) }\SpecialCharTok{\%\textgreater{}\%}
  \FunctionTok{ungroup}\NormalTok{()}
\end{Highlighting}
\end{Shaded}

\begin{Shaded}
\begin{Highlighting}[]
\FunctionTok{descr}\NormalTok{(Base\_Trajipaq\_Finale\_Ok}\SpecialCharTok{$}\NormalTok{I4)}
\end{Highlighting}
\end{Shaded}

\begin{verbatim}
## Descriptive Statistics  
## Base_Trajipaq_Finale_Ok$I4  
## N: 1563  
## 
##                          I4
## ----------------- ---------
##              Mean      0.73
##           Std.Dev      0.29
##               Min      0.00
##                Q1      0.58
##            Median      0.75
##                Q3      1.00
##               Max      1.00
##               MAD      0.37
##               IQR      0.42
##                CV      0.39
##          Skewness     -0.83
##       SE.Skewness      0.07
##          Kurtosis     -0.21
##           N.Valid   1106.00
##         Pct.Valid     70.76
\end{verbatim}

\begin{Shaded}
\begin{Highlighting}[]
\FunctionTok{ggplot}\NormalTok{(}\AttributeTok{data =}\NormalTok{ Base\_Trajipaq\_Finale\_Ok, }\AttributeTok{mapping =} \FunctionTok{aes}\NormalTok{(}\AttributeTok{x =}\NormalTok{ I4)) }\SpecialCharTok{+}
  \FunctionTok{geom\_histogram}\NormalTok{(}\AttributeTok{na.rm =} \ConstantTok{TRUE}\NormalTok{, }\AttributeTok{fill =} \StringTok{"pink"}\NormalTok{, }\AttributeTok{binwidth =} \FloatTok{0.01}\NormalTok{) }\SpecialCharTok{+}
  \FunctionTok{theme\_bw}\NormalTok{()}
\end{Highlighting}
\end{Shaded}

\includegraphics{Article---Sentiment-d-appartenance_files/figure-latex/unnamed-chunk-33-1.pdf}

La distribution est asymétrique vers la gauche. L'indice moyen de
maîtrise du français est 0.73.

\hypertarget{q23.-comment-est-votre-uxe9tat-de-santuxe9-en-guxe9nuxe9ral}{%
\subsection{Q23. Comment est votre état de santé en général
?}\label{q23.-comment-est-votre-uxe9tat-de-santuxe9-en-guxe9nuxe9ral}}

\begin{enumerate}
\def\labelenumi{\arabic{enumi}.}
\tightlist
\item
  Très bon
\item
  Bon
\item
  Moyen
\item
  Mauvais
\item
  Très mauvais
\item
  Je ne sais pas
\item
  Je préfère ne pas répondre
\end{enumerate}

Type de variable = factoriel Niveau de mesure = ordinal

\begin{Shaded}
\begin{Highlighting}[]
\NormalTok{Base\_Trajipaq\_Finale\_Ok }\OtherTok{\textless{}{-}}
\NormalTok{  Base\_Trajipaq\_Finale\_Ok }\SpecialCharTok{\%\textgreater{}\%}
  \FunctionTok{mutate}\NormalTok{(}\AttributeTok{Q23 =} \FunctionTok{case\_when}\NormalTok{(}
\NormalTok{    Q23 }\SpecialCharTok{==} \DecValTok{1} \SpecialCharTok{\textasciitilde{}} \DecValTok{5}\NormalTok{,}
\NormalTok{    Q23 }\SpecialCharTok{==} \DecValTok{2} \SpecialCharTok{\textasciitilde{}} \DecValTok{4}\NormalTok{,}
\NormalTok{    Q23 }\SpecialCharTok{==} \DecValTok{3} \SpecialCharTok{\textasciitilde{}} \DecValTok{3}\NormalTok{,}
\NormalTok{    Q23 }\SpecialCharTok{==} \DecValTok{4} \SpecialCharTok{\textasciitilde{}} \DecValTok{2}\NormalTok{,}
\NormalTok{    Q23 }\SpecialCharTok{==} \DecValTok{5} \SpecialCharTok{\textasciitilde{}} \DecValTok{1}\NormalTok{,}
\NormalTok{    Q23 }\SpecialCharTok{==} \DecValTok{98} \SpecialCharTok{\textasciitilde{}} \ConstantTok{NA}\NormalTok{,}
\NormalTok{    Q23 }\SpecialCharTok{==} \DecValTok{99} \SpecialCharTok{\textasciitilde{}} \ConstantTok{NA}
\NormalTok{  ))}
\end{Highlighting}
\end{Shaded}

\begin{Shaded}
\begin{Highlighting}[]
\FunctionTok{freq}\NormalTok{(Base\_Trajipaq\_Finale\_Ok}\SpecialCharTok{$}\NormalTok{Q23)}
\end{Highlighting}
\end{Shaded}

\begin{verbatim}
## Frequencies  
## Base_Trajipaq_Finale_Ok$Q23  
## Type: Numeric  
## 
##               Freq   % Valid   % Valid Cum.   % Total   % Total Cum.
## ----------- ------ --------- -------------- --------- --------------
##           2      4      0.28           0.28      0.26           0.26
##           3     18      1.24           1.52      1.15           1.41
##           4    578     39.94          41.47     36.98          38.39
##           5    847     58.53         100.00     54.19          92.58
##        <NA>    116                               7.42         100.00
##       Total   1563    100.00         100.00    100.00         100.00
\end{verbatim}

\begin{Shaded}
\begin{Highlighting}[]
\FunctionTok{ggplot}\NormalTok{(}\AttributeTok{data =}\NormalTok{ Base\_Trajipaq\_Finale\_Ok, }\AttributeTok{mapping =} \FunctionTok{aes}\NormalTok{(}\AttributeTok{x =}\NormalTok{ Q23)) }\SpecialCharTok{+}
  \FunctionTok{geom\_bar}\NormalTok{(}\AttributeTok{fill =} \StringTok{"pink"}\NormalTok{, }\AttributeTok{na.rm =} \ConstantTok{TRUE}\NormalTok{) }\SpecialCharTok{+}
  \FunctionTok{theme\_bw}\NormalTok{()}
\end{Highlighting}
\end{Shaded}

\includegraphics{Article---Sentiment-d-appartenance_files/figure-latex/unnamed-chunk-36-1.pdf}

La distribution est asymétrique vers la gauche. 98.47\% des répondants
ont un état de santé moyen ou bon.

\hypertarget{i5.-indice-de-luxe9tat-de-santuxe9-au-cours-des-12-derniers-mois}{%
\subsection{I5. Indice de l'état de santé au cours des 12 derniers
mois}\label{i5.-indice-de-luxe9tat-de-santuxe9-au-cours-des-12-derniers-mois}}

\begin{Shaded}
\begin{Highlighting}[]
\NormalTok{Base\_Trajipaq\_Finale\_Ok }\OtherTok{\textless{}{-}}
\NormalTok{  Base\_Trajipaq\_Finale\_Ok }\SpecialCharTok{\%\textgreater{}\%}
  \FunctionTok{mutate}\NormalTok{(}\AttributeTok{Q24A =} \FunctionTok{case\_when}\NormalTok{(}
\NormalTok{    Q24A }\SpecialCharTok{==} \DecValTok{1} \SpecialCharTok{\textasciitilde{}} \DecValTok{1}\NormalTok{,}
\NormalTok{    Q24A }\SpecialCharTok{==} \DecValTok{2} \SpecialCharTok{\textasciitilde{}} \DecValTok{2}\NormalTok{,}
\NormalTok{    Q24A }\SpecialCharTok{==} \DecValTok{96} \SpecialCharTok{\textasciitilde{}} \ConstantTok{NA}\NormalTok{,}
\NormalTok{    Q24A }\SpecialCharTok{==} \DecValTok{97} \SpecialCharTok{\textasciitilde{}} \ConstantTok{NA}\NormalTok{,}
\NormalTok{    Q24A }\SpecialCharTok{==} \DecValTok{98} \SpecialCharTok{\textasciitilde{}} \ConstantTok{NA}\NormalTok{,}
\NormalTok{    Q24A }\SpecialCharTok{==} \DecValTok{99} \SpecialCharTok{\textasciitilde{}} \ConstantTok{NA}
\NormalTok{  ))}

\NormalTok{Base\_Trajipaq\_Finale\_Ok }\OtherTok{\textless{}{-}}
\NormalTok{  Base\_Trajipaq\_Finale\_Ok }\SpecialCharTok{\%\textgreater{}\%}
  \FunctionTok{mutate}\NormalTok{(}\AttributeTok{Q24B =} \FunctionTok{case\_when}\NormalTok{(}
\NormalTok{    Q24B }\SpecialCharTok{==} \DecValTok{1} \SpecialCharTok{\textasciitilde{}} \DecValTok{1}\NormalTok{,}
\NormalTok{    Q24B }\SpecialCharTok{==} \DecValTok{2} \SpecialCharTok{\textasciitilde{}} \DecValTok{2}\NormalTok{,}
\NormalTok{    Q24B }\SpecialCharTok{==} \DecValTok{96} \SpecialCharTok{\textasciitilde{}} \ConstantTok{NA}\NormalTok{,}
\NormalTok{    Q24B }\SpecialCharTok{==} \DecValTok{97} \SpecialCharTok{\textasciitilde{}} \ConstantTok{NA}\NormalTok{,}
\NormalTok{    Q24B }\SpecialCharTok{==} \DecValTok{98} \SpecialCharTok{\textasciitilde{}} \ConstantTok{NA}\NormalTok{,}
\NormalTok{    Q24B }\SpecialCharTok{==} \DecValTok{99} \SpecialCharTok{\textasciitilde{}} \ConstantTok{NA}
\NormalTok{  ))}

\NormalTok{Base\_Trajipaq\_Finale\_Ok }\OtherTok{\textless{}{-}}
\NormalTok{  Base\_Trajipaq\_Finale\_Ok }\SpecialCharTok{\%\textgreater{}\%}
  \FunctionTok{mutate}\NormalTok{(}\AttributeTok{Q24C =} \FunctionTok{case\_when}\NormalTok{(}
\NormalTok{    Q24C }\SpecialCharTok{==} \DecValTok{1} \SpecialCharTok{\textasciitilde{}} \DecValTok{1}\NormalTok{,}
\NormalTok{    Q24C }\SpecialCharTok{==} \DecValTok{2} \SpecialCharTok{\textasciitilde{}} \DecValTok{2}\NormalTok{,}
\NormalTok{    Q24C }\SpecialCharTok{==} \DecValTok{96} \SpecialCharTok{\textasciitilde{}} \ConstantTok{NA}\NormalTok{,}
\NormalTok{    Q24C }\SpecialCharTok{==} \DecValTok{97} \SpecialCharTok{\textasciitilde{}} \ConstantTok{NA}\NormalTok{,}
\NormalTok{    Q24C }\SpecialCharTok{==} \DecValTok{98} \SpecialCharTok{\textasciitilde{}} \ConstantTok{NA}\NormalTok{,}
\NormalTok{    Q24C }\SpecialCharTok{==} \DecValTok{99} \SpecialCharTok{\textasciitilde{}} \ConstantTok{NA}
\NormalTok{  ))}

\NormalTok{Base\_Trajipaq\_Finale\_Ok }\OtherTok{\textless{}{-}}
\NormalTok{  Base\_Trajipaq\_Finale\_Ok }\SpecialCharTok{\%\textgreater{}\%}
  \FunctionTok{mutate}\NormalTok{(}\AttributeTok{Q24D =} \FunctionTok{case\_when}\NormalTok{(}
\NormalTok{    Q24D }\SpecialCharTok{==} \DecValTok{1} \SpecialCharTok{\textasciitilde{}} \DecValTok{1}\NormalTok{,}
\NormalTok{    Q24D }\SpecialCharTok{==} \DecValTok{2} \SpecialCharTok{\textasciitilde{}} \DecValTok{2}\NormalTok{,}
\NormalTok{    Q24D }\SpecialCharTok{==} \DecValTok{96} \SpecialCharTok{\textasciitilde{}} \ConstantTok{NA}\NormalTok{,}
\NormalTok{    Q24D }\SpecialCharTok{==} \DecValTok{97} \SpecialCharTok{\textasciitilde{}} \ConstantTok{NA}\NormalTok{,}
\NormalTok{    Q24D }\SpecialCharTok{==} \DecValTok{98} \SpecialCharTok{\textasciitilde{}} \ConstantTok{NA}\NormalTok{,}
\NormalTok{    Q24D }\SpecialCharTok{==} \DecValTok{99} \SpecialCharTok{\textasciitilde{}} \ConstantTok{NA}
\NormalTok{  ))}

\NormalTok{Base\_Trajipaq\_Finale\_Ok }\OtherTok{\textless{}{-}}
\NormalTok{  Base\_Trajipaq\_Finale\_Ok }\SpecialCharTok{\%\textgreater{}\%}
  \FunctionTok{mutate}\NormalTok{(}\AttributeTok{Q24E =} \FunctionTok{case\_when}\NormalTok{(}
\NormalTok{    Q24E }\SpecialCharTok{==} \DecValTok{1} \SpecialCharTok{\textasciitilde{}} \DecValTok{1}\NormalTok{,}
\NormalTok{    Q24E }\SpecialCharTok{==} \DecValTok{2} \SpecialCharTok{\textasciitilde{}} \DecValTok{2}\NormalTok{,}
\NormalTok{    Q24E }\SpecialCharTok{==} \DecValTok{96} \SpecialCharTok{\textasciitilde{}} \ConstantTok{NA}\NormalTok{,}
\NormalTok{    Q24E }\SpecialCharTok{==} \DecValTok{97} \SpecialCharTok{\textasciitilde{}} \ConstantTok{NA}\NormalTok{,}
\NormalTok{    Q24E }\SpecialCharTok{==} \DecValTok{98} \SpecialCharTok{\textasciitilde{}} \ConstantTok{NA}\NormalTok{,}
\NormalTok{    Q24E }\SpecialCharTok{==} \DecValTok{99} \SpecialCharTok{\textasciitilde{}} \ConstantTok{NA}
\NormalTok{  ))}
\end{Highlighting}
\end{Shaded}

\begin{Shaded}
\begin{Highlighting}[]
\NormalTok{v\_I5 }\OtherTok{\textless{}{-}} \FunctionTok{c}\NormalTok{(}\StringTok{"Q24A"}\NormalTok{,}
          \StringTok{"Q24B"}\NormalTok{,}
          \StringTok{"Q24C"}\NormalTok{,}
          \StringTok{"Q24D"}\NormalTok{,}
          \StringTok{"Q24E"}
\NormalTok{          )}

\NormalTok{Base\_Trajipaq\_Finale\_Ok }\OtherTok{\textless{}{-}}
\NormalTok{  Base\_Trajipaq\_Finale\_Ok }\SpecialCharTok{\%\textgreater{}\%}
  \FunctionTok{rowwise}\NormalTok{() }\SpecialCharTok{\%\textgreater{}\%}
  \FunctionTok{mutate}\NormalTok{(}\AttributeTok{I5 =}\NormalTok{ (}\FunctionTok{sum}\NormalTok{(}\FunctionTok{c\_across}\NormalTok{(}\FunctionTok{all\_of}\NormalTok{(v\_I5)), }\AttributeTok{na.rm =} \ConstantTok{TRUE}\NormalTok{) }\SpecialCharTok{{-}} \FunctionTok{length}\NormalTok{(v\_I5)) }\SpecialCharTok{/}\NormalTok{ (}\DecValTok{1}\SpecialCharTok{*}\FunctionTok{length}\NormalTok{(v\_I5))) }\SpecialCharTok{\%\textgreater{}\%}
  \FunctionTok{mutate}\NormalTok{(}\AttributeTok{I5 =} \ControlFlowTok{if}\NormalTok{ (I5 }\SpecialCharTok{\textless{}} \DecValTok{0}\NormalTok{) }\ConstantTok{NA\_real\_} \ControlFlowTok{else}\NormalTok{ I5) }\SpecialCharTok{\%\textgreater{}\%}
  \FunctionTok{ungroup}\NormalTok{()}
\end{Highlighting}
\end{Shaded}

\begin{Shaded}
\begin{Highlighting}[]
\FunctionTok{descr}\NormalTok{(Base\_Trajipaq\_Finale\_Ok}\SpecialCharTok{$}\NormalTok{I5)}
\end{Highlighting}
\end{Shaded}

\begin{verbatim}
## Descriptive Statistics  
## Base_Trajipaq_Finale_Ok$I5  
## N: 1563  
## 
##                          I5
## ----------------- ---------
##              Mean      0.78
##           Std.Dev      0.29
##               Min      0.00
##                Q1      0.60
##            Median      1.00
##                Q3      1.00
##               Max      1.00
##               MAD      0.00
##               IQR      0.40
##                CV      0.37
##          Skewness     -1.05
##       SE.Skewness      0.06
##          Kurtosis      0.12
##           N.Valid   1430.00
##         Pct.Valid     91.49
\end{verbatim}

\begin{Shaded}
\begin{Highlighting}[]
\FunctionTok{ggplot}\NormalTok{(}\AttributeTok{data =}\NormalTok{ Base\_Trajipaq\_Finale\_Ok, }\AttributeTok{mapping =} \FunctionTok{aes}\NormalTok{(}\AttributeTok{x =}\NormalTok{ I5)) }\SpecialCharTok{+}
  \FunctionTok{geom\_histogram}\NormalTok{(}\AttributeTok{na.rm =} \ConstantTok{TRUE}\NormalTok{, }\AttributeTok{fill =} \StringTok{"pink"}\NormalTok{, }\AttributeTok{binwidth =} \FloatTok{0.01}\NormalTok{) }\SpecialCharTok{+}
  \FunctionTok{theme\_bw}\NormalTok{()}
\end{Highlighting}
\end{Shaded}

\includegraphics{Article---Sentiment-d-appartenance_files/figure-latex/unnamed-chunk-40-1.pdf}

La distribution est asymétrique vers la gauche. L'indice moyen de l'état
de santé au cours des 12 derniers mois est de 0.78.

\hypertarget{q26.-quel-uxe9tait-le-plus-haut-dipluxf4me-duxe9tudes-obtenu-par-votre-muxe8re}{%
\subsection{Q26. Quel était le plus haut diplôme d'études obtenu par
votre mère
?}\label{q26.-quel-uxe9tait-le-plus-haut-dipluxf4me-duxe9tudes-obtenu-par-votre-muxe8re}}

\begin{enumerate}
\def\labelenumi{\arabic{enumi}.}
\tightlist
\item
  Moins que le secondaire
\item
  Secondaire ou équivalent
\item
  Collégial
\item
  Formation professionnelle ou post-secondaire
\item
  Université
\item
  Inconnu
\item
  Je ne sais pas
\item
  Je préfère ne pas répondre
\end{enumerate}

Type de variable = factoriel Niveau de mesure = ordinal

\begin{Shaded}
\begin{Highlighting}[]
\NormalTok{Base\_Trajipaq\_Finale\_Ok }\OtherTok{\textless{}{-}}
\NormalTok{  Base\_Trajipaq\_Finale\_Ok }\SpecialCharTok{\%\textgreater{}\%}
  \FunctionTok{mutate}\NormalTok{(}\AttributeTok{Q26 =} \FunctionTok{case\_when}\NormalTok{(}
\NormalTok{    Q26 }\SpecialCharTok{==} \DecValTok{1} \SpecialCharTok{\textasciitilde{}} \DecValTok{1}\NormalTok{,}
\NormalTok{    Q26 }\SpecialCharTok{==} \DecValTok{2} \SpecialCharTok{\textasciitilde{}} \DecValTok{2}\NormalTok{,}
\NormalTok{    Q26 }\SpecialCharTok{==} \DecValTok{3} \SpecialCharTok{\textasciitilde{}} \DecValTok{3}\NormalTok{,}
\NormalTok{    Q26 }\SpecialCharTok{==} \DecValTok{4} \SpecialCharTok{\textasciitilde{}} \DecValTok{4}\NormalTok{,}
\NormalTok{    Q26 }\SpecialCharTok{==} \DecValTok{5} \SpecialCharTok{\textasciitilde{}} \DecValTok{5}\NormalTok{,}
\NormalTok{    Q26 }\SpecialCharTok{==} \DecValTok{6} \SpecialCharTok{\textasciitilde{}} \ConstantTok{NA}\NormalTok{,}
\NormalTok{    Q26 }\SpecialCharTok{==} \DecValTok{98} \SpecialCharTok{\textasciitilde{}} \ConstantTok{NA}\NormalTok{,}
\NormalTok{    Q26 }\SpecialCharTok{==} \DecValTok{99} \SpecialCharTok{\textasciitilde{}} \ConstantTok{NA}
\NormalTok{  ))}
\end{Highlighting}
\end{Shaded}

\begin{Shaded}
\begin{Highlighting}[]
\FunctionTok{freq}\NormalTok{(Base\_Trajipaq\_Finale\_Ok}\SpecialCharTok{$}\NormalTok{Q26)}
\end{Highlighting}
\end{Shaded}

\begin{verbatim}
## Frequencies  
## Base_Trajipaq_Finale_Ok$Q26  
## Type: Numeric  
## 
##               Freq   % Valid   % Valid Cum.   % Total   % Total Cum.
## ----------- ------ --------- -------------- --------- --------------
##           1    238     16.49          16.49     15.23          15.23
##           2    355     24.60          41.09     22.71          37.94
##           3    241     16.70          57.80     15.42          53.36
##           4    196     13.58          71.38     12.54          65.90
##           5    413     28.62         100.00     26.42          92.32
##        <NA>    120                               7.68         100.00
##       Total   1563    100.00         100.00    100.00         100.00
\end{verbatim}

\begin{Shaded}
\begin{Highlighting}[]
\FunctionTok{ggplot}\NormalTok{(}\AttributeTok{data =}\NormalTok{ Base\_Trajipaq\_Finale\_Ok, }\AttributeTok{mapping =} \FunctionTok{aes}\NormalTok{(}\AttributeTok{x =}\NormalTok{ Q26)) }\SpecialCharTok{+}
  \FunctionTok{geom\_bar}\NormalTok{(}\AttributeTok{fill =} \StringTok{"pink"}\NormalTok{, }\AttributeTok{na.rm =} \ConstantTok{TRUE}\NormalTok{) }\SpecialCharTok{+}
  \FunctionTok{theme\_bw}\NormalTok{()}
\end{Highlighting}
\end{Shaded}

\includegraphics{Article---Sentiment-d-appartenance_files/figure-latex/unnamed-chunk-43-1.pdf}

La valeur modale est ``diplôme universitaire'' (28.62\%).

\hypertarget{q28.-quel-uxe9tait-le-plus-haut-dipluxf4me-duxe9tudes-obtenu-par-votre-puxe8re}{%
\subsection{Q28. Quel était le plus haut diplôme d'études obtenu par
votre père
?}\label{q28.-quel-uxe9tait-le-plus-haut-dipluxf4me-duxe9tudes-obtenu-par-votre-puxe8re}}

\begin{enumerate}
\def\labelenumi{\arabic{enumi}.}
\tightlist
\item
  Moins que le secondaire
\item
  Secondaire ou équivalent
\item
  Collégial
\item
  Formation professionnelle ou post-secondaire
\item
  Université
\item
  Inconnu
\item
  Je ne sais pas
\item
  Je préfère ne pas répondre
\end{enumerate}

Type de variable = factoriel Niveau de mesure = ordinal

\begin{Shaded}
\begin{Highlighting}[]
\NormalTok{Base\_Trajipaq\_Finale\_Ok }\OtherTok{\textless{}{-}}
\NormalTok{  Base\_Trajipaq\_Finale\_Ok }\SpecialCharTok{\%\textgreater{}\%}
  \FunctionTok{mutate}\NormalTok{(}\AttributeTok{Q28 =} \FunctionTok{case\_when}\NormalTok{(}
\NormalTok{    Q28 }\SpecialCharTok{==} \DecValTok{1} \SpecialCharTok{\textasciitilde{}} \DecValTok{1}\NormalTok{,}
\NormalTok{    Q28 }\SpecialCharTok{==} \DecValTok{2} \SpecialCharTok{\textasciitilde{}} \DecValTok{2}\NormalTok{,}
\NormalTok{    Q28 }\SpecialCharTok{==} \DecValTok{3} \SpecialCharTok{\textasciitilde{}} \DecValTok{3}\NormalTok{,}
\NormalTok{    Q28 }\SpecialCharTok{==} \DecValTok{4} \SpecialCharTok{\textasciitilde{}} \DecValTok{4}\NormalTok{,}
\NormalTok{    Q28 }\SpecialCharTok{==} \DecValTok{5} \SpecialCharTok{\textasciitilde{}} \DecValTok{5}\NormalTok{,}
\NormalTok{    Q28 }\SpecialCharTok{==} \DecValTok{6} \SpecialCharTok{\textasciitilde{}} \ConstantTok{NA}\NormalTok{,}
\NormalTok{    Q28 }\SpecialCharTok{==} \DecValTok{98} \SpecialCharTok{\textasciitilde{}} \ConstantTok{NA}\NormalTok{,}
\NormalTok{    Q28 }\SpecialCharTok{==} \DecValTok{99} \SpecialCharTok{\textasciitilde{}} \ConstantTok{NA}
\NormalTok{  ))}
\end{Highlighting}
\end{Shaded}

\begin{Shaded}
\begin{Highlighting}[]
\FunctionTok{freq}\NormalTok{(Base\_Trajipaq\_Finale\_Ok}\SpecialCharTok{$}\NormalTok{Q28)}
\end{Highlighting}
\end{Shaded}

\begin{verbatim}
## Frequencies  
## Base_Trajipaq_Finale_Ok$Q28  
## Type: Numeric  
## 
##               Freq   % Valid   % Valid Cum.   % Total   % Total Cum.
## ----------- ------ --------- -------------- --------- --------------
##           1    206     14.56          14.56     13.18          13.18
##           2    288     20.35          34.91     18.43          31.61
##           3    201     14.20          49.12     12.86          44.47
##           4    229     16.18          65.30     14.65          59.12
##           5    491     34.70         100.00     31.41          90.53
##        <NA>    148                               9.47         100.00
##       Total   1563    100.00         100.00    100.00         100.00
\end{verbatim}

\begin{Shaded}
\begin{Highlighting}[]
\FunctionTok{ggplot}\NormalTok{(}\AttributeTok{data =}\NormalTok{ Base\_Trajipaq\_Finale\_Ok, }\AttributeTok{mapping =} \FunctionTok{aes}\NormalTok{(}\AttributeTok{x =}\NormalTok{ Q28)) }\SpecialCharTok{+}
  \FunctionTok{geom\_bar}\NormalTok{(}\AttributeTok{fill =} \StringTok{"pink"}\NormalTok{, }\AttributeTok{na.rm =} \ConstantTok{TRUE}\NormalTok{) }\SpecialCharTok{+}
  \FunctionTok{theme\_bw}\NormalTok{()}
\end{Highlighting}
\end{Shaded}

\includegraphics{Article---Sentiment-d-appartenance_files/figure-latex/unnamed-chunk-46-1.pdf}

La valeur modale est ``diplôme universitaire'' (34.70\%).

\hypertarget{ruxe9sumuxe9}{%
\subsection{Résumé}\label{ruxe9sumuxe9}}

L'échantillon possède les caractéristiques socio-démographiques
suivantes :

\begin{itemize}
\tightlist
\item
  faible prépondérance de femmes (56.69\%)
\item
  faible prépondérance de personnes âgées de 35 à 39 ans (24.63\%)
\item
  forte prépondérance de personnes mariées (47.53\%)
\item
  prépondérance de français (26.685) et d'algériens (6.649\%)
\item
  forte prépondérance de personnes nées au Canada (71.21\%)
\item
  forte prépondérance d'anciens travailleurs temporaires parmi les
  immigrés de première génération (57.79\%)
\item
  forte prépondérance de personnes possédant un diplôme universitaire
  (67.31\%)
\item
  prépondérance de personnes athées (36.36\%), catholiques (32.21\%) et
  musulmanes (13.83\%)
\item
  prépondérance de personnes ayant pour langue maternelle le français
  (63.84\%)
\item
  forte maîtrise moyenne du français (0.87)
\item
  forte maîtrise moyenne de l'anglais (0.73)
\item
  très forte prépondérance de personnes actuellement en bonne ou moyenne
  santé (98.47\%)
\item
  bon état de santé moyen durant les 12 derniers mois (0.78)
\item
  prépondérance d'un diplôme universitaire chez la mère (28.62\%)
\item
  prépondérance d'un diplôme universitaire chez la mère (34.70\%)
\end{itemize}

\hypertarget{pruxe9sentation-de-la-variable-induxe9pendante}{%
\section{Présentation de la variable
indépendante}\label{pruxe9sentation-de-la-variable-induxe9pendante}}

\hypertarget{q39a.-dites-si-vous-uxeates-tout-uxe0-fait-daccord-plutuxf4t-daccord-plutuxf4t-pas-daccord-ou-pas-du-tout-daccord-avec-les-propositions-suivantes-participation-identitaire-et-sentiment-dappartenance-je-me-sens-chez-moi-au-quuxe9bec.}{%
\subsection{Q39A. Dites si vous êtes tout à fait d'accord, plutôt
d'accord, plutôt pas d'accord ou pas du tout d'accord avec les
propositions suivantes (participation identitaire et sentiment
d'appartenance) : je me sens chez moi au
Québec.}\label{q39a.-dites-si-vous-uxeates-tout-uxe0-fait-daccord-plutuxf4t-daccord-plutuxf4t-pas-daccord-ou-pas-du-tout-daccord-avec-les-propositions-suivantes-participation-identitaire-et-sentiment-dappartenance-je-me-sens-chez-moi-au-quuxe9bec.}}

\begin{Shaded}
\begin{Highlighting}[]
\NormalTok{Base\_Trajipaq\_Finale\_Ok }\OtherTok{\textless{}{-}}
\NormalTok{  Base\_Trajipaq\_Finale\_Ok }\SpecialCharTok{\%\textgreater{}\%}
  \FunctionTok{mutate}\NormalTok{(}\AttributeTok{Q39A =} \FunctionTok{case\_when}\NormalTok{(}
\NormalTok{    Q39A }\SpecialCharTok{==} \DecValTok{1} \SpecialCharTok{\textasciitilde{}} \DecValTok{4}\NormalTok{,}
\NormalTok{    Q39A }\SpecialCharTok{==} \DecValTok{2} \SpecialCharTok{\textasciitilde{}} \DecValTok{3}\NormalTok{,}
\NormalTok{    Q39A }\SpecialCharTok{==} \DecValTok{3} \SpecialCharTok{\textasciitilde{}} \DecValTok{2}\NormalTok{,}
\NormalTok{    Q39A }\SpecialCharTok{==} \DecValTok{4} \SpecialCharTok{\textasciitilde{}} \DecValTok{1}\NormalTok{,}
\NormalTok{    Q39A }\SpecialCharTok{==} \DecValTok{98} \SpecialCharTok{\textasciitilde{}} \ConstantTok{NA}\NormalTok{,}
\NormalTok{    Q39A }\SpecialCharTok{==} \DecValTok{99} \SpecialCharTok{\textasciitilde{}} \ConstantTok{NA}
\NormalTok{  ))}
\end{Highlighting}
\end{Shaded}

\begin{Shaded}
\begin{Highlighting}[]
\FunctionTok{freq}\NormalTok{(Base\_Trajipaq\_Finale\_Ok}\SpecialCharTok{$}\NormalTok{Q39A)}
\end{Highlighting}
\end{Shaded}

\begin{verbatim}
## Frequencies  
## Base_Trajipaq_Finale_Ok$Q39A  
## Type: Numeric  
## 
##               Freq   % Valid   % Valid Cum.   % Total   % Total Cum.
## ----------- ------ --------- -------------- --------- --------------
##           1     59      3.84           3.84      3.77           3.77
##           2    116      7.56          11.40      7.42          11.20
##           3    513     33.42          44.82     32.82          44.02
##           4    847     55.18         100.00     54.19          98.21
##        <NA>     28                               1.79         100.00
##       Total   1563    100.00         100.00    100.00         100.00
\end{verbatim}

\begin{Shaded}
\begin{Highlighting}[]
\FunctionTok{ggplot}\NormalTok{(}\AttributeTok{data =}\NormalTok{ Base\_Trajipaq\_Finale\_Ok, }\AttributeTok{mapping =} \FunctionTok{aes}\NormalTok{(}\AttributeTok{x =}\NormalTok{ Q39A)) }\SpecialCharTok{+}
  \FunctionTok{geom\_bar}\NormalTok{(}\AttributeTok{fill =} \StringTok{"pink"}\NormalTok{,}
           \AttributeTok{na.rm =} \ConstantTok{TRUE}
\NormalTok{           ) }\SpecialCharTok{+}
  \FunctionTok{theme\_bw}\NormalTok{()}
\end{Highlighting}
\end{Shaded}

\includegraphics{Article---Sentiment-d-appartenance_files/figure-latex/unnamed-chunk-49-1.pdf}

La distribution est asymétrique vers la gauche. 87.01\% des personnes
interrogées sont tout à fait d'accord ou plutôt d'accord pour dire
qu'elles se sentent chez elles au Québec.

\hypertarget{q39b.-dites-si-vous-uxeates-tout-uxe0-fait-daccord-plutuxf4t-daccord-plutuxf4t-pas-daccord-ou-pas-du-tout-daccord-avec-les-propositions-suivantes-participation-identitaire-et-sentiment-dappartenance-il-faut-faire-oublier-ses-origines-pour-se-faire-accepter-au-quuxe9bec.}{%
\subsection{Q39B. Dites si vous êtes tout à fait d'accord, plutôt
d'accord, plutôt pas d'accord ou pas du tout d'accord avec les
propositions suivantes (participation identitaire et sentiment
d'appartenance) : il faut faire oublier ses origines pour se faire
accepter au
Québec.}\label{q39b.-dites-si-vous-uxeates-tout-uxe0-fait-daccord-plutuxf4t-daccord-plutuxf4t-pas-daccord-ou-pas-du-tout-daccord-avec-les-propositions-suivantes-participation-identitaire-et-sentiment-dappartenance-il-faut-faire-oublier-ses-origines-pour-se-faire-accepter-au-quuxe9bec.}}

\begin{Shaded}
\begin{Highlighting}[]
\NormalTok{Base\_Trajipaq\_Finale\_Ok }\OtherTok{\textless{}{-}}
\NormalTok{  Base\_Trajipaq\_Finale\_Ok }\SpecialCharTok{\%\textgreater{}\%}
  \FunctionTok{mutate}\NormalTok{(}\AttributeTok{Q39B =} \FunctionTok{case\_when}\NormalTok{(}
\NormalTok{    Q39B }\SpecialCharTok{==} \DecValTok{1} \SpecialCharTok{\textasciitilde{}} \DecValTok{4}\NormalTok{,}
\NormalTok{    Q39B }\SpecialCharTok{==} \DecValTok{2} \SpecialCharTok{\textasciitilde{}} \DecValTok{3}\NormalTok{,}
\NormalTok{    Q39B }\SpecialCharTok{==} \DecValTok{3} \SpecialCharTok{\textasciitilde{}} \DecValTok{2}\NormalTok{,}
\NormalTok{    Q39B }\SpecialCharTok{==} \DecValTok{4} \SpecialCharTok{\textasciitilde{}} \DecValTok{1}\NormalTok{,}
\NormalTok{    Q39B }\SpecialCharTok{==} \DecValTok{98} \SpecialCharTok{\textasciitilde{}} \ConstantTok{NA}\NormalTok{,}
\NormalTok{    Q39B }\SpecialCharTok{==} \DecValTok{99} \SpecialCharTok{\textasciitilde{}} \ConstantTok{NA}
\NormalTok{  ))}
\end{Highlighting}
\end{Shaded}

\begin{Shaded}
\begin{Highlighting}[]
\FunctionTok{freq}\NormalTok{(Base\_Trajipaq\_Finale\_Ok}\SpecialCharTok{$}\NormalTok{Q39B)}
\end{Highlighting}
\end{Shaded}

\begin{verbatim}
## Frequencies  
## Base_Trajipaq_Finale_Ok$Q39B  
## Type: Numeric  
## 
##               Freq   % Valid   % Valid Cum.   % Total   % Total Cum.
## ----------- ------ --------- -------------- --------- --------------
##           1    671     44.70          44.70     42.93          42.93
##           2    390     25.98          70.69     24.95          67.88
##           3    298     19.85          90.54     19.07          86.95
##           4    142      9.46         100.00      9.09          96.03
##        <NA>     62                               3.97         100.00
##       Total   1563    100.00         100.00    100.00         100.00
\end{verbatim}

\begin{Shaded}
\begin{Highlighting}[]
\FunctionTok{ggplot}\NormalTok{(}\AttributeTok{data =}\NormalTok{ Base\_Trajipaq\_Finale\_Ok, }\AttributeTok{mapping =} \FunctionTok{aes}\NormalTok{(}\AttributeTok{x =}\NormalTok{ Q39B)) }\SpecialCharTok{+}
  \FunctionTok{geom\_bar}\NormalTok{(}\AttributeTok{fill =} \StringTok{"pink"}\NormalTok{,}
           \AttributeTok{na.rm =} \ConstantTok{TRUE}
\NormalTok{           ) }\SpecialCharTok{+}
  \FunctionTok{theme\_bw}\NormalTok{()}
\end{Highlighting}
\end{Shaded}

\includegraphics{Article---Sentiment-d-appartenance_files/figure-latex/unnamed-chunk-52-1.pdf}

La distribution est asymétrique vers la droite. 67.88\% des personnes
interrogées ne sont pas du tout d'accord ou plutôt pas d'accord pour
dire qu'il faut faire oublier ses origines pour se faire accepter au
Québec.

\hypertarget{q39c.-dites-si-vous-uxeates-tout-uxe0-fait-daccord-plutuxf4t-daccord-plutuxf4t-pas-daccord-ou-pas-du-tout-daccord-avec-les-propositions-suivantes-participation-identitaire-et-sentiment-dappartenance-on-me-voit-comme-une-quuxe9buxe9coise.}{%
\subsection{Q39C. Dites si vous êtes tout à fait d'accord, plutôt
d'accord, plutôt pas d'accord ou pas du tout d'accord avec les
propositions suivantes (participation identitaire et sentiment
d'appartenance) : on me voit comme une
Québécoise.}\label{q39c.-dites-si-vous-uxeates-tout-uxe0-fait-daccord-plutuxf4t-daccord-plutuxf4t-pas-daccord-ou-pas-du-tout-daccord-avec-les-propositions-suivantes-participation-identitaire-et-sentiment-dappartenance-on-me-voit-comme-une-quuxe9buxe9coise.}}

\begin{Shaded}
\begin{Highlighting}[]
\NormalTok{Base\_Trajipaq\_Finale\_Ok }\OtherTok{\textless{}{-}}
\NormalTok{  Base\_Trajipaq\_Finale\_Ok }\SpecialCharTok{\%\textgreater{}\%}
  \FunctionTok{mutate}\NormalTok{(}\AttributeTok{Q39C =} \FunctionTok{case\_when}\NormalTok{(}
\NormalTok{    Q39C }\SpecialCharTok{==} \DecValTok{1} \SpecialCharTok{\textasciitilde{}} \DecValTok{4}\NormalTok{,}
\NormalTok{    Q39C }\SpecialCharTok{==} \DecValTok{2} \SpecialCharTok{\textasciitilde{}} \DecValTok{3}\NormalTok{,}
\NormalTok{    Q39C }\SpecialCharTok{==} \DecValTok{3} \SpecialCharTok{\textasciitilde{}} \DecValTok{2}\NormalTok{,}
\NormalTok{    Q39C }\SpecialCharTok{==} \DecValTok{4} \SpecialCharTok{\textasciitilde{}} \DecValTok{1}\NormalTok{,}
\NormalTok{    Q39C }\SpecialCharTok{==} \DecValTok{98} \SpecialCharTok{\textasciitilde{}} \ConstantTok{NA}\NormalTok{,}
\NormalTok{    Q39C }\SpecialCharTok{==} \DecValTok{99} \SpecialCharTok{\textasciitilde{}} \ConstantTok{NA}
\NormalTok{  ))}
\end{Highlighting}
\end{Shaded}

\begin{Shaded}
\begin{Highlighting}[]
\FunctionTok{freq}\NormalTok{(Base\_Trajipaq\_Finale\_Ok}\SpecialCharTok{$}\NormalTok{Q39C)}
\end{Highlighting}
\end{Shaded}

\begin{verbatim}
## Frequencies  
## Base_Trajipaq_Finale_Ok$Q39C  
## Type: Numeric  
## 
##               Freq   % Valid   % Valid Cum.   % Total   % Total Cum.
## ----------- ------ --------- -------------- --------- --------------
##           1    442     29.97          29.97     28.28          28.28
##           2    341     23.12          53.08     21.82          50.10
##           3    259     17.56          70.64     16.57          66.67
##           4    433     29.36         100.00     27.70          94.37
##        <NA>     88                               5.63         100.00
##       Total   1563    100.00         100.00    100.00         100.00
\end{verbatim}

\begin{Shaded}
\begin{Highlighting}[]
\FunctionTok{ggplot}\NormalTok{(}\AttributeTok{data =}\NormalTok{ Base\_Trajipaq\_Finale\_Ok, }\AttributeTok{mapping =} \FunctionTok{aes}\NormalTok{(}\AttributeTok{x =}\NormalTok{ Q39C)) }\SpecialCharTok{+}
  \FunctionTok{geom\_bar}\NormalTok{(}\AttributeTok{fill =} \StringTok{"pink"}\NormalTok{,}
           \AttributeTok{na.rm =} \ConstantTok{TRUE}
\NormalTok{           ) }\SpecialCharTok{+}
  \FunctionTok{theme\_bw}\NormalTok{()}
\end{Highlighting}
\end{Shaded}

\includegraphics{Article---Sentiment-d-appartenance_files/figure-latex/unnamed-chunk-55-1.pdf}

Les réponses modales sont ``tout à fait d'accord'' (29.36\%) et ``pas du
tout d'accord'' (29.97\%).

\hypertarget{q39d.-dites-si-vous-uxeates-tout-uxe0-fait-daccord-plutuxf4t-daccord-plutuxf4t-pas-daccord-ou-pas-du-tout-daccord-avec-les-propositions-suivantes-participation-identitaire-et-sentiment-dappartenance-je-me-sens-quuxe9buxe9coise.}{%
\subsection{Q39D. Dites si vous êtes tout à fait d'accord, plutôt
d'accord, plutôt pas d'accord ou pas du tout d'accord avec les
propositions suivantes (participation identitaire et sentiment
d'appartenance) : je me sens
Québécois(e).}\label{q39d.-dites-si-vous-uxeates-tout-uxe0-fait-daccord-plutuxf4t-daccord-plutuxf4t-pas-daccord-ou-pas-du-tout-daccord-avec-les-propositions-suivantes-participation-identitaire-et-sentiment-dappartenance-je-me-sens-quuxe9buxe9coise.}}

\begin{Shaded}
\begin{Highlighting}[]
\NormalTok{Base\_Trajipaq\_Finale\_Ok }\OtherTok{\textless{}{-}}
\NormalTok{  Base\_Trajipaq\_Finale\_Ok }\SpecialCharTok{\%\textgreater{}\%}
  \FunctionTok{mutate}\NormalTok{(}\AttributeTok{Q39D =} \FunctionTok{case\_when}\NormalTok{(}
\NormalTok{    Q39D }\SpecialCharTok{==} \DecValTok{1} \SpecialCharTok{\textasciitilde{}} \DecValTok{4}\NormalTok{,}
\NormalTok{    Q39D }\SpecialCharTok{==} \DecValTok{2} \SpecialCharTok{\textasciitilde{}} \DecValTok{3}\NormalTok{,}
\NormalTok{    Q39D }\SpecialCharTok{==} \DecValTok{3} \SpecialCharTok{\textasciitilde{}} \DecValTok{2}\NormalTok{,}
\NormalTok{    Q39D }\SpecialCharTok{==} \DecValTok{4} \SpecialCharTok{\textasciitilde{}} \DecValTok{1}\NormalTok{,}
\NormalTok{    Q39D }\SpecialCharTok{==} \DecValTok{98} \SpecialCharTok{\textasciitilde{}} \ConstantTok{NA}\NormalTok{,}
\NormalTok{    Q39D }\SpecialCharTok{==} \DecValTok{99} \SpecialCharTok{\textasciitilde{}} \ConstantTok{NA}
\NormalTok{  ))}
\end{Highlighting}
\end{Shaded}

\begin{Shaded}
\begin{Highlighting}[]
\FunctionTok{freq}\NormalTok{(Base\_Trajipaq\_Finale\_Ok}\SpecialCharTok{$}\NormalTok{Q39D)}
\end{Highlighting}
\end{Shaded}

\begin{verbatim}
## Frequencies  
## Base_Trajipaq_Finale_Ok$Q39D  
## Type: Numeric  
## 
##               Freq   % Valid   % Valid Cum.   % Total   % Total Cum.
## ----------- ------ --------- -------------- --------- --------------
##           1    259     17.27          17.27     16.57          16.57
##           2    292     19.47          36.73     18.68          35.25
##           3    406     27.07          63.80     25.98          61.23
##           4    543     36.20         100.00     34.74          95.97
##        <NA>     63                               4.03         100.00
##       Total   1563    100.00         100.00    100.00         100.00
\end{verbatim}

\begin{Shaded}
\begin{Highlighting}[]
\FunctionTok{ggplot}\NormalTok{(}\AttributeTok{data =}\NormalTok{ Base\_Trajipaq\_Finale\_Ok, }\AttributeTok{mapping =} \FunctionTok{aes}\NormalTok{(}\AttributeTok{x =}\NormalTok{ Q39D)) }\SpecialCharTok{+}
  \FunctionTok{geom\_bar}\NormalTok{(}\AttributeTok{fill =} \StringTok{"pink"}\NormalTok{,}
           \AttributeTok{na.rm =} \ConstantTok{TRUE}
\NormalTok{           ) }\SpecialCharTok{+}
  \FunctionTok{theme\_bw}\NormalTok{()}
\end{Highlighting}
\end{Shaded}

\includegraphics{Article---Sentiment-d-appartenance_files/figure-latex/unnamed-chunk-58-1.pdf}

La distribution est asymétrique vers la gauche. 60.72\% des personnes
interrogées sont tout à fait d'accord ou plutôt d'accord pour dire
qu'elles se sentent Québécois(e)s.

\hypertarget{q39e.-dites-si-vous-uxeates-tout-uxe0-fait-daccord-plutuxf4t-daccord-plutuxf4t-pas-daccord-ou-pas-du-tout-daccord-avec-les-propositions-suivantes-participation-identitaire-et-sentiment-dappartenance-je-me-sens-canadienne.}{%
\subsection{Q39E. Dites si vous êtes tout à fait d'accord, plutôt
d'accord, plutôt pas d'accord ou pas du tout d'accord avec les
propositions suivantes (participation identitaire et sentiment
d'appartenance) : je me sens
Canadien(ne).}\label{q39e.-dites-si-vous-uxeates-tout-uxe0-fait-daccord-plutuxf4t-daccord-plutuxf4t-pas-daccord-ou-pas-du-tout-daccord-avec-les-propositions-suivantes-participation-identitaire-et-sentiment-dappartenance-je-me-sens-canadienne.}}

\begin{Shaded}
\begin{Highlighting}[]
\NormalTok{Base\_Trajipaq\_Finale\_Ok }\OtherTok{\textless{}{-}}
\NormalTok{  Base\_Trajipaq\_Finale\_Ok }\SpecialCharTok{\%\textgreater{}\%}
  \FunctionTok{mutate}\NormalTok{(}\AttributeTok{Q39E =} \FunctionTok{case\_when}\NormalTok{(}
\NormalTok{    Q39E }\SpecialCharTok{==} \DecValTok{1} \SpecialCharTok{\textasciitilde{}} \DecValTok{4}\NormalTok{,}
\NormalTok{    Q39E }\SpecialCharTok{==} \DecValTok{2} \SpecialCharTok{\textasciitilde{}} \DecValTok{3}\NormalTok{,}
\NormalTok{    Q39E }\SpecialCharTok{==} \DecValTok{3} \SpecialCharTok{\textasciitilde{}} \DecValTok{2}\NormalTok{,}
\NormalTok{    Q39E }\SpecialCharTok{==} \DecValTok{4} \SpecialCharTok{\textasciitilde{}} \DecValTok{1}\NormalTok{,}
\NormalTok{    Q39E }\SpecialCharTok{==} \DecValTok{98} \SpecialCharTok{\textasciitilde{}} \ConstantTok{NA}\NormalTok{,}
\NormalTok{    Q39E }\SpecialCharTok{==} \DecValTok{99} \SpecialCharTok{\textasciitilde{}} \ConstantTok{NA}
\NormalTok{  ))}
\end{Highlighting}
\end{Shaded}

\begin{Shaded}
\begin{Highlighting}[]
\FunctionTok{freq}\NormalTok{(Base\_Trajipaq\_Finale\_Ok}\SpecialCharTok{$}\NormalTok{Q39E)}
\end{Highlighting}
\end{Shaded}

\begin{verbatim}
## Frequencies  
## Base_Trajipaq_Finale_Ok$Q39E  
## Type: Numeric  
## 
##               Freq   % Valid   % Valid Cum.   % Total   % Total Cum.
## ----------- ------ --------- -------------- --------- --------------
##           1    100      6.57           6.57      6.40           6.40
##           2    171     11.24          17.82     10.94          17.34
##           3    550     36.16          53.98     35.19          52.53
##           4    700     46.02         100.00     44.79          97.31
##        <NA>     42                               2.69         100.00
##       Total   1563    100.00         100.00    100.00         100.00
\end{verbatim}

\begin{Shaded}
\begin{Highlighting}[]
\FunctionTok{ggplot}\NormalTok{(}\AttributeTok{data =}\NormalTok{ Base\_Trajipaq\_Finale\_Ok, }\AttributeTok{mapping =} \FunctionTok{aes}\NormalTok{(}\AttributeTok{x =}\NormalTok{ Q39E)) }\SpecialCharTok{+}
  \FunctionTok{geom\_bar}\NormalTok{(}\AttributeTok{fill =} \StringTok{"pink"}\NormalTok{,}
           \AttributeTok{na.rm =} \ConstantTok{TRUE}
\NormalTok{           ) }\SpecialCharTok{+}
  \FunctionTok{theme\_bw}\NormalTok{()}
\end{Highlighting}
\end{Shaded}

\includegraphics{Article---Sentiment-d-appartenance_files/figure-latex/unnamed-chunk-61-1.pdf}

La distribution est asymétrique vers la gauche. 79.98\% des personnes
interrogées sont tout à fait d'accord ou plutôt d'accord pour dire
qu'elles se sentent Canadien(ne).

\hypertarget{q39f.-dites-si-vous-uxeates-tout-uxe0-fait-daccord-plutuxf4t-daccord-plutuxf4t-pas-daccord-ou-pas-du-tout-daccord-avec-les-propositions-suivantes-participation-identitaire-et-sentiment-dappartenance-jai-un-sentiment-dappartenance-uxe0-mon-pays-dorigine.}{%
\subsection{Q39F. Dites si vous êtes tout à fait d'accord, plutôt
d'accord, plutôt pas d'accord ou pas du tout d'accord avec les
propositions suivantes (participation identitaire et sentiment
d'appartenance) : j'ai un sentiment d'appartenance à mon pays
d'origine.}\label{q39f.-dites-si-vous-uxeates-tout-uxe0-fait-daccord-plutuxf4t-daccord-plutuxf4t-pas-daccord-ou-pas-du-tout-daccord-avec-les-propositions-suivantes-participation-identitaire-et-sentiment-dappartenance-jai-un-sentiment-dappartenance-uxe0-mon-pays-dorigine.}}

\begin{Shaded}
\begin{Highlighting}[]
\NormalTok{Base\_Trajipaq\_Finale\_Ok }\OtherTok{\textless{}{-}}
\NormalTok{  Base\_Trajipaq\_Finale\_Ok }\SpecialCharTok{\%\textgreater{}\%}
  \FunctionTok{mutate}\NormalTok{(}\AttributeTok{Q39F =} \FunctionTok{case\_when}\NormalTok{(}
\NormalTok{    Q39F }\SpecialCharTok{==} \DecValTok{1} \SpecialCharTok{\textasciitilde{}} \DecValTok{4}\NormalTok{,}
\NormalTok{    Q39F }\SpecialCharTok{==} \DecValTok{2} \SpecialCharTok{\textasciitilde{}} \DecValTok{3}\NormalTok{,}
\NormalTok{    Q39F }\SpecialCharTok{==} \DecValTok{3} \SpecialCharTok{\textasciitilde{}} \DecValTok{2}\NormalTok{,}
\NormalTok{    Q39F }\SpecialCharTok{==} \DecValTok{4} \SpecialCharTok{\textasciitilde{}} \DecValTok{1}\NormalTok{,}
\NormalTok{    Q39F }\SpecialCharTok{==} \DecValTok{98} \SpecialCharTok{\textasciitilde{}} \ConstantTok{NA}\NormalTok{,}
\NormalTok{    Q39F }\SpecialCharTok{==} \DecValTok{99} \SpecialCharTok{\textasciitilde{}} \ConstantTok{NA}
\NormalTok{  ))}
\end{Highlighting}
\end{Shaded}

\begin{Shaded}
\begin{Highlighting}[]
\FunctionTok{freq}\NormalTok{(Base\_Trajipaq\_Finale\_Ok}\SpecialCharTok{$}\NormalTok{Q39F)}
\end{Highlighting}
\end{Shaded}

\begin{verbatim}
## Frequencies  
## Base_Trajipaq_Finale_Ok$Q39F  
## Type: Numeric  
## 
##               Freq   % Valid   % Valid Cum.   % Total   % Total Cum.
## ----------- ------ --------- -------------- --------- --------------
##           1     82      5.44           5.44      5.25           5.25
##           2    198     13.14          18.58     12.67          17.91
##           3    550     36.50          55.08     35.19          53.10
##           4    677     44.92         100.00     43.31          96.42
##        <NA>     56                               3.58         100.00
##       Total   1563    100.00         100.00    100.00         100.00
\end{verbatim}

\begin{Shaded}
\begin{Highlighting}[]
\FunctionTok{ggplot}\NormalTok{(}\AttributeTok{data =}\NormalTok{ Base\_Trajipaq\_Finale\_Ok, }\AttributeTok{mapping =} \FunctionTok{aes}\NormalTok{(}\AttributeTok{x =}\NormalTok{ Q39F)) }\SpecialCharTok{+}
  \FunctionTok{geom\_bar}\NormalTok{(}\AttributeTok{fill =} \StringTok{"pink"}\NormalTok{,}
           \AttributeTok{na.rm =} \ConstantTok{TRUE}
\NormalTok{           ) }\SpecialCharTok{+}
  \FunctionTok{theme\_bw}\NormalTok{()}
\end{Highlighting}
\end{Shaded}

\includegraphics{Article---Sentiment-d-appartenance_files/figure-latex/unnamed-chunk-64-1.pdf}

La distribution est asymétrique vers la gauche. 81.42\% des personnes
interrogées sont tout à fait d'accord ou plutôt d'accord pour dire
qu'elles possèdent un sentiment d'appartenance envers leur pays
d'origine.

\hypertarget{i1-lindice-du-sentiment-dappartenance-au-quuxe9bec}{%
\subsection{I1 : L'indice du sentiment d'appartenance au
Québec}\label{i1-lindice-du-sentiment-dappartenance-au-quuxe9bec}}

A la différence des variables précédentes, l'indice du sentiment
d'appartenance est une variable numérique continue dont le niveau de
mesure est ratio.

\begin{Shaded}
\begin{Highlighting}[]
\NormalTok{v\_I1 }\OtherTok{\textless{}{-}} \FunctionTok{c}\NormalTok{(}\StringTok{"Q39A"}\NormalTok{,}
          \StringTok{"Q39B"}\NormalTok{,}
          \StringTok{"Q39C"}\NormalTok{,}
          \StringTok{"Q39D"}\NormalTok{,}
          \StringTok{"Q39E"}\NormalTok{,}
          \StringTok{"Q39F"}
\NormalTok{          )}

\NormalTok{Base\_Trajipaq\_Finale\_Ok }\OtherTok{\textless{}{-}}
\NormalTok{  Base\_Trajipaq\_Finale\_Ok }\SpecialCharTok{\%\textgreater{}\%}
  \FunctionTok{rowwise}\NormalTok{() }\SpecialCharTok{\%\textgreater{}\%}
  \FunctionTok{mutate}\NormalTok{(}\AttributeTok{I1 =}\NormalTok{ (}\FunctionTok{sum}\NormalTok{(}\FunctionTok{c\_across}\NormalTok{(}\FunctionTok{all\_of}\NormalTok{(v\_I1)), }\AttributeTok{na.rm =} \ConstantTok{TRUE}\NormalTok{) }\SpecialCharTok{{-}} \FunctionTok{length}\NormalTok{(v\_I1)) }\SpecialCharTok{/}\NormalTok{ (}\DecValTok{3}\SpecialCharTok{*}\FunctionTok{length}\NormalTok{(v\_I1))) }\SpecialCharTok{\%\textgreater{}\%}
  \FunctionTok{mutate}\NormalTok{(}\AttributeTok{I1 =} \ControlFlowTok{if}\NormalTok{ (I1 }\SpecialCharTok{\textless{}} \DecValTok{0}\NormalTok{) }\ConstantTok{NA\_real\_} \ControlFlowTok{else}\NormalTok{ I1) }\SpecialCharTok{\%\textgreater{}\%}
  \FunctionTok{ungroup}\NormalTok{()}
\end{Highlighting}
\end{Shaded}

\begin{Shaded}
\begin{Highlighting}[]
\FunctionTok{descr}\NormalTok{(Base\_Trajipaq\_Finale\_Ok}\SpecialCharTok{$}\NormalTok{I1)}
\end{Highlighting}
\end{Shaded}

\begin{verbatim}
## Descriptive Statistics  
## Base_Trajipaq_Finale_Ok$I1  
## N: 1563  
## 
##                          I1
## ----------------- ---------
##              Mean      0.59
##           Std.Dev      0.21
##               Min      0.00
##                Q1      0.44
##            Median      0.61
##                Q3      0.72
##               Max      1.00
##               MAD      0.25
##               IQR      0.28
##                CV      0.35
##          Skewness     -0.30
##       SE.Skewness      0.06
##          Kurtosis     -0.41
##           N.Valid   1550.00
##         Pct.Valid     99.17
\end{verbatim}

\begin{Shaded}
\begin{Highlighting}[]
\FunctionTok{ggplot}\NormalTok{(}\AttributeTok{data =}\NormalTok{ Base\_Trajipaq\_Finale\_Ok, }\AttributeTok{mapping =} \FunctionTok{aes}\NormalTok{(}\AttributeTok{x =}\NormalTok{ I1)) }\SpecialCharTok{+}
  \FunctionTok{geom\_histogram}\NormalTok{(}\AttributeTok{na.rm =} \ConstantTok{TRUE}\NormalTok{, }\AttributeTok{fill =} \StringTok{"pink"}\NormalTok{, }\AttributeTok{binwidth =} \FloatTok{0.01}\NormalTok{) }\SpecialCharTok{+}
  \FunctionTok{theme\_bw}\NormalTok{()}
\end{Highlighting}
\end{Shaded}

\includegraphics{Article---Sentiment-d-appartenance_files/figure-latex/unnamed-chunk-67-1.pdf}

La distribution est asymétrique vers la gauche. L'indice moyen du
sentiment d'appartenance au Québec est 0.59.

\hypertarget{pruxe9sentation-de-la-variable-duxe9pendante}{%
\section{Présentation de la variable
dépendante}\label{pruxe9sentation-de-la-variable-duxe9pendante}}

\hypertarget{q40a.-au-cours-des-12-mois-avant-la-crise-de-la-covid-19-pensez-vous-avoir-subi-des-traitements-inuxe9galitaires-ou-des-discriminations-un-traitement-inuxe9galitaire-ou-une-discrimination-se-repuxe8re-si-vous-avez-uxe9tuxe9-moins-bien-traituxe9-que-les-autres-sans-raison-valable-dans-lemploi}{%
\subsection{Q40A. Au cours des 12 mois (avant la crise de la COVID-19),
pensez-vous avoir subi des traitements inégalitaires ou des
discriminations ? Un traitement inégalitaire ou une discrimination se
repère si « vous avez été moins bien traité que les autres, sans raison
valable » : dans l'emploi
?}\label{q40a.-au-cours-des-12-mois-avant-la-crise-de-la-covid-19-pensez-vous-avoir-subi-des-traitements-inuxe9galitaires-ou-des-discriminations-un-traitement-inuxe9galitaire-ou-une-discrimination-se-repuxe8re-si-vous-avez-uxe9tuxe9-moins-bien-traituxe9-que-les-autres-sans-raison-valable-dans-lemploi}}

\begin{Shaded}
\begin{Highlighting}[]
\NormalTok{Base\_Trajipaq\_Finale\_Ok }\OtherTok{\textless{}{-}}
\NormalTok{  Base\_Trajipaq\_Finale\_Ok }\SpecialCharTok{\%\textgreater{}\%}
  \FunctionTok{mutate}\NormalTok{(}\AttributeTok{Q40A =} \FunctionTok{case\_when}\NormalTok{(}
\NormalTok{    Q40A }\SpecialCharTok{==} \DecValTok{1} \SpecialCharTok{\textasciitilde{}} \DecValTok{3}\NormalTok{,}
\NormalTok{    Q40A }\SpecialCharTok{==} \DecValTok{2} \SpecialCharTok{\textasciitilde{}} \DecValTok{2}\NormalTok{,}
\NormalTok{    Q40A }\SpecialCharTok{==} \DecValTok{3} \SpecialCharTok{\textasciitilde{}} \DecValTok{1}\NormalTok{,}
\NormalTok{    Q40A }\SpecialCharTok{==} \DecValTok{96} \SpecialCharTok{\textasciitilde{}} \ConstantTok{NA}\NormalTok{,}
\NormalTok{    Q40A }\SpecialCharTok{==} \DecValTok{98} \SpecialCharTok{\textasciitilde{}} \ConstantTok{NA}\NormalTok{,}
\NormalTok{    Q40A }\SpecialCharTok{==} \DecValTok{99} \SpecialCharTok{\textasciitilde{}} \ConstantTok{NA}
\NormalTok{  ))}
\end{Highlighting}
\end{Shaded}

\begin{Shaded}
\begin{Highlighting}[]
\FunctionTok{freq}\NormalTok{(Base\_Trajipaq\_Finale\_Ok}\SpecialCharTok{$}\NormalTok{Q40A)}
\end{Highlighting}
\end{Shaded}

\begin{verbatim}
## Frequencies  
## Base_Trajipaq_Finale_Ok$Q40A  
## Type: Numeric  
## 
##               Freq   % Valid   % Valid Cum.   % Total   % Total Cum.
## ----------- ------ --------- -------------- --------- --------------
##           1    926     65.44          65.44     59.25          59.25
##           2    383     27.07          92.51     24.50          83.75
##           3    106      7.49         100.00      6.78          90.53
##        <NA>    148                               9.47         100.00
##       Total   1563    100.00         100.00    100.00         100.00
\end{verbatim}

\begin{Shaded}
\begin{Highlighting}[]
\FunctionTok{ggplot}\NormalTok{(}\AttributeTok{data =}\NormalTok{ Base\_Trajipaq\_Finale\_Ok, }\AttributeTok{mapping =} \FunctionTok{aes}\NormalTok{(}\AttributeTok{x =}\NormalTok{ Q40A)) }\SpecialCharTok{+}
  \FunctionTok{geom\_bar}\NormalTok{(}\AttributeTok{fill =} \StringTok{"pink"}\NormalTok{, }\AttributeTok{na.rm =} \ConstantTok{TRUE}\NormalTok{) }\SpecialCharTok{+}
  \FunctionTok{theme\_bw}\NormalTok{()}
\end{Highlighting}
\end{Shaded}

\includegraphics{Article---Sentiment-d-appartenance_files/figure-latex/unnamed-chunk-70-1.pdf}

La distribution est asymétrique vers la droite. 65.44\% des personnes
interrogées disent ne jamais avoir subit de discrimination dans l'emploi
au cours des 12 derniers mois précédent la crise du covid 19.

\hypertarget{q40b.-au-cours-des-12-mois-avant-la-crise-de-la-covid-19-pensez-vous-avoir-subi-des-traitements-inuxe9galitaires-ou-des-discriminations-un-traitement-inuxe9galitaire-ou-une-discrimination-se-repuxe8re-si-vous-avez-uxe9tuxe9-moins-bien-traituxe9-que-les-autres-sans-raison-valable-dans-la-recherche-de-logement}{%
\subsection{Q40B. Au cours des 12 mois (avant la crise de la COVID-19),
pensez-vous avoir subi des traitements inégalitaires ou des
discriminations ? Un traitement inégalitaire ou une discrimination se
repère si « vous avez été moins bien traité que les autres, sans raison
valable » : dans la recherche de logement
?}\label{q40b.-au-cours-des-12-mois-avant-la-crise-de-la-covid-19-pensez-vous-avoir-subi-des-traitements-inuxe9galitaires-ou-des-discriminations-un-traitement-inuxe9galitaire-ou-une-discrimination-se-repuxe8re-si-vous-avez-uxe9tuxe9-moins-bien-traituxe9-que-les-autres-sans-raison-valable-dans-la-recherche-de-logement}}

\begin{Shaded}
\begin{Highlighting}[]
\NormalTok{Base\_Trajipaq\_Finale\_Ok }\OtherTok{\textless{}{-}}
\NormalTok{  Base\_Trajipaq\_Finale\_Ok }\SpecialCharTok{\%\textgreater{}\%}
  \FunctionTok{mutate}\NormalTok{(}\AttributeTok{Q40B =} \FunctionTok{case\_when}\NormalTok{(}
\NormalTok{    Q40B }\SpecialCharTok{==} \DecValTok{1} \SpecialCharTok{\textasciitilde{}} \DecValTok{3}\NormalTok{,}
\NormalTok{    Q40B }\SpecialCharTok{==} \DecValTok{2} \SpecialCharTok{\textasciitilde{}} \DecValTok{2}\NormalTok{,}
\NormalTok{    Q40B }\SpecialCharTok{==} \DecValTok{3} \SpecialCharTok{\textasciitilde{}} \DecValTok{1}\NormalTok{,}
\NormalTok{    Q40B }\SpecialCharTok{==} \DecValTok{96} \SpecialCharTok{\textasciitilde{}} \ConstantTok{NA}\NormalTok{,}
\NormalTok{    Q40B }\SpecialCharTok{==} \DecValTok{98} \SpecialCharTok{\textasciitilde{}} \ConstantTok{NA}\NormalTok{,}
\NormalTok{    Q40B }\SpecialCharTok{==} \DecValTok{99} \SpecialCharTok{\textasciitilde{}} \ConstantTok{NA}
\NormalTok{  ))}
\end{Highlighting}
\end{Shaded}

\begin{Shaded}
\begin{Highlighting}[]
\FunctionTok{freq}\NormalTok{(Base\_Trajipaq\_Finale\_Ok}\SpecialCharTok{$}\NormalTok{Q40B)}
\end{Highlighting}
\end{Shaded}

\begin{verbatim}
## Frequencies  
## Base_Trajipaq_Finale_Ok$Q40B  
## Type: Numeric  
## 
##               Freq   % Valid   % Valid Cum.   % Total   % Total Cum.
## ----------- ------ --------- -------------- --------- --------------
##           1   1029     79.64          79.64     65.83          65.83
##           2    195     15.09          94.74     12.48          78.31
##           3     68      5.26         100.00      4.35          82.66
##        <NA>    271                              17.34         100.00
##       Total   1563    100.00         100.00    100.00         100.00
\end{verbatim}

\begin{Shaded}
\begin{Highlighting}[]
\FunctionTok{ggplot}\NormalTok{(}\AttributeTok{data =}\NormalTok{ Base\_Trajipaq\_Finale\_Ok, }\AttributeTok{mapping =} \FunctionTok{aes}\NormalTok{(}\AttributeTok{x =}\NormalTok{ Q40B)) }\SpecialCharTok{+}
  \FunctionTok{geom\_bar}\NormalTok{(}\AttributeTok{fill =} \StringTok{"pink"}\NormalTok{, }\AttributeTok{na.rm =} \ConstantTok{TRUE}\NormalTok{) }\SpecialCharTok{+}
  \FunctionTok{theme\_bw}\NormalTok{()}
\end{Highlighting}
\end{Shaded}

\includegraphics{Article---Sentiment-d-appartenance_files/figure-latex/unnamed-chunk-73-1.pdf}

La distribution est asymétrique vers la droite. 79.64\% des personnes
interrogées disent ne jamais avoir subit de discrimination dans la
recherche de logement au cours des 12 derniers mois précédent la crise
du covid 19.

\hypertarget{q40c.-au-cours-des-12-mois-avant-la-crise-de-la-covid-19-pensez-vous-avoir-subi-des-traitements-inuxe9galitaires-ou-des-discriminations-un-traitement-inuxe9galitaire-ou-une-discrimination-se-repuxe8re-si-vous-avez-uxe9tuxe9-moins-bien-traituxe9-que-les-autres-sans-raison-valable-dans-la-santuxe9}{%
\subsection{Q40C. Au cours des 12 mois (avant la crise de la COVID-19),
pensez-vous avoir subi des traitements inégalitaires ou des
discriminations ? Un traitement inégalitaire ou une discrimination se
repère si « vous avez été moins bien traité que les autres, sans raison
valable » : dans la santé
?}\label{q40c.-au-cours-des-12-mois-avant-la-crise-de-la-covid-19-pensez-vous-avoir-subi-des-traitements-inuxe9galitaires-ou-des-discriminations-un-traitement-inuxe9galitaire-ou-une-discrimination-se-repuxe8re-si-vous-avez-uxe9tuxe9-moins-bien-traituxe9-que-les-autres-sans-raison-valable-dans-la-santuxe9}}

\begin{Shaded}
\begin{Highlighting}[]
\NormalTok{Base\_Trajipaq\_Finale\_Ok }\OtherTok{\textless{}{-}}
\NormalTok{  Base\_Trajipaq\_Finale\_Ok }\SpecialCharTok{\%\textgreater{}\%}
  \FunctionTok{mutate}\NormalTok{(}\AttributeTok{Q40C =} \FunctionTok{case\_when}\NormalTok{(}
\NormalTok{    Q40C }\SpecialCharTok{==} \DecValTok{1} \SpecialCharTok{\textasciitilde{}} \DecValTok{3}\NormalTok{,}
\NormalTok{    Q40C }\SpecialCharTok{==} \DecValTok{2} \SpecialCharTok{\textasciitilde{}} \DecValTok{2}\NormalTok{,}
\NormalTok{    Q40C }\SpecialCharTok{==} \DecValTok{3} \SpecialCharTok{\textasciitilde{}} \DecValTok{1}\NormalTok{,}
\NormalTok{    Q40C }\SpecialCharTok{==} \DecValTok{96} \SpecialCharTok{\textasciitilde{}} \ConstantTok{NA}\NormalTok{,}
\NormalTok{    Q40C }\SpecialCharTok{==} \DecValTok{98} \SpecialCharTok{\textasciitilde{}} \ConstantTok{NA}\NormalTok{,}
\NormalTok{    Q40C }\SpecialCharTok{==} \DecValTok{99} \SpecialCharTok{\textasciitilde{}} \ConstantTok{NA}
\NormalTok{  ))}
\end{Highlighting}
\end{Shaded}

\begin{Shaded}
\begin{Highlighting}[]
\FunctionTok{freq}\NormalTok{(Base\_Trajipaq\_Finale\_Ok}\SpecialCharTok{$}\NormalTok{Q40C)}
\end{Highlighting}
\end{Shaded}

\begin{verbatim}
## Frequencies  
## Base_Trajipaq_Finale_Ok$Q40C  
## Type: Numeric  
## 
##               Freq   % Valid   % Valid Cum.   % Total   % Total Cum.
## ----------- ------ --------- -------------- --------- --------------
##           1   1155     82.26          82.26     73.90          73.90
##           2    184     13.11          95.37     11.77          85.67
##           3     65      4.63         100.00      4.16          89.83
##        <NA>    159                              10.17         100.00
##       Total   1563    100.00         100.00    100.00         100.00
\end{verbatim}

\begin{Shaded}
\begin{Highlighting}[]
\FunctionTok{ggplot}\NormalTok{(}\AttributeTok{data =}\NormalTok{ Base\_Trajipaq\_Finale\_Ok, }\AttributeTok{mapping =} \FunctionTok{aes}\NormalTok{(}\AttributeTok{x =}\NormalTok{ Q40C)) }\SpecialCharTok{+}
  \FunctionTok{geom\_bar}\NormalTok{(}\AttributeTok{fill =} \StringTok{"pink"}\NormalTok{, }\AttributeTok{na.rm =} \ConstantTok{TRUE}\NormalTok{) }\SpecialCharTok{+}
  \FunctionTok{theme\_bw}\NormalTok{()}
\end{Highlighting}
\end{Shaded}

\includegraphics{Article---Sentiment-d-appartenance_files/figure-latex/unnamed-chunk-76-1.pdf}

La distribution est asymétrique vers la droite. 82.26\% des personnes
interrogées disent ne jamais avoir subit de discrimination dans la santé
au cours des 12 derniers mois précédent la crise du covid 19.

\hypertarget{q40d.-au-cours-des-12-mois-avant-la-crise-de-la-covid-19-pensez-vous-avoir-subi-des-traitements-inuxe9galitaires-ou-des-discriminations-un-traitement-inuxe9galitaire-ou-une-discrimination-se-repuxe8re-si-vous-avez-uxe9tuxe9-moins-bien-traituxe9-que-les-autres-sans-raison-valable-sur-le-lieu-de-formation}{%
\subsection{Q40D. Au cours des 12 mois (avant la crise de la COVID-19),
pensez-vous avoir subi des traitements inégalitaires ou des
discriminations ? Un traitement inégalitaire ou une discrimination se
repère si « vous avez été moins bien traité que les autres, sans raison
valable » : sur le lieu de formation
?}\label{q40d.-au-cours-des-12-mois-avant-la-crise-de-la-covid-19-pensez-vous-avoir-subi-des-traitements-inuxe9galitaires-ou-des-discriminations-un-traitement-inuxe9galitaire-ou-une-discrimination-se-repuxe8re-si-vous-avez-uxe9tuxe9-moins-bien-traituxe9-que-les-autres-sans-raison-valable-sur-le-lieu-de-formation}}

\begin{Shaded}
\begin{Highlighting}[]
\NormalTok{Base\_Trajipaq\_Finale\_Ok }\OtherTok{\textless{}{-}}
\NormalTok{  Base\_Trajipaq\_Finale\_Ok }\SpecialCharTok{\%\textgreater{}\%}
  \FunctionTok{mutate}\NormalTok{(}\AttributeTok{Q40D =} \FunctionTok{case\_when}\NormalTok{(}
\NormalTok{    Q40D }\SpecialCharTok{==} \DecValTok{1} \SpecialCharTok{\textasciitilde{}} \DecValTok{3}\NormalTok{,}
\NormalTok{    Q40D }\SpecialCharTok{==} \DecValTok{2} \SpecialCharTok{\textasciitilde{}} \DecValTok{2}\NormalTok{,}
\NormalTok{    Q40D }\SpecialCharTok{==} \DecValTok{3} \SpecialCharTok{\textasciitilde{}} \DecValTok{1}\NormalTok{,}
\NormalTok{    Q40D }\SpecialCharTok{==} \DecValTok{96} \SpecialCharTok{\textasciitilde{}} \ConstantTok{NA}\NormalTok{,}
\NormalTok{    Q40D }\SpecialCharTok{==} \DecValTok{98} \SpecialCharTok{\textasciitilde{}} \ConstantTok{NA}\NormalTok{,}
\NormalTok{    Q40D }\SpecialCharTok{==} \DecValTok{99} \SpecialCharTok{\textasciitilde{}} \ConstantTok{NA}
\NormalTok{  ))}
\end{Highlighting}
\end{Shaded}

\begin{Shaded}
\begin{Highlighting}[]
\FunctionTok{freq}\NormalTok{(Base\_Trajipaq\_Finale\_Ok}\SpecialCharTok{$}\NormalTok{Q40D)}
\end{Highlighting}
\end{Shaded}

\begin{verbatim}
## Frequencies  
## Base_Trajipaq_Finale_Ok$Q40D  
## Type: Numeric  
## 
##               Freq   % Valid   % Valid Cum.   % Total   % Total Cum.
## ----------- ------ --------- -------------- --------- --------------
##           1   1072     81.40          81.40     68.59          68.59
##           2    194     14.73          96.13     12.41          81.00
##           3     51      3.87         100.00      3.26          84.26
##        <NA>    246                              15.74         100.00
##       Total   1563    100.00         100.00    100.00         100.00
\end{verbatim}

\begin{Shaded}
\begin{Highlighting}[]
\FunctionTok{ggplot}\NormalTok{(}\AttributeTok{data =}\NormalTok{ Base\_Trajipaq\_Finale\_Ok, }\AttributeTok{mapping =} \FunctionTok{aes}\NormalTok{(}\AttributeTok{x =}\NormalTok{ Q40D)) }\SpecialCharTok{+}
  \FunctionTok{geom\_bar}\NormalTok{(}\AttributeTok{fill =} \StringTok{"pink"}\NormalTok{, }\AttributeTok{na.rm =} \ConstantTok{TRUE}\NormalTok{) }\SpecialCharTok{+}
  \FunctionTok{theme\_bw}\NormalTok{()}
\end{Highlighting}
\end{Shaded}

\includegraphics{Article---Sentiment-d-appartenance_files/figure-latex/unnamed-chunk-79-1.pdf}

La distribution est asymétrique vers la droite. 81.40\% des personnes
interrogées disent ne jamais avoir subit de discrimination sur le lieu
de formation au cours des 12 derniers mois précédent la crise du covid
19.

\hypertarget{i2-lindice-de-discrimination-lors-des-12-derniers-mois-pruxe9cuxe9dent-la-crise-du-covid-19}{%
\subsection{I2 : L'indice de discrimination lors des 12 derniers mois
précédent la crise du covid
19}\label{i2-lindice-de-discrimination-lors-des-12-derniers-mois-pruxe9cuxe9dent-la-crise-du-covid-19}}

A la différence des variables précédentes, l'indice du sentiment
d'appartenance est une variable numérique continue dont le niveau de
mesure est ratio.

\begin{Shaded}
\begin{Highlighting}[]
\NormalTok{v\_I2 }\OtherTok{\textless{}{-}} \FunctionTok{c}\NormalTok{(}\StringTok{"Q40A"}\NormalTok{,}
          \StringTok{"Q40B"}\NormalTok{,}
          \StringTok{"Q40C"}\NormalTok{,}
          \StringTok{"Q39D"}
\NormalTok{          )}

\NormalTok{Base\_Trajipaq\_Finale\_Ok }\OtherTok{\textless{}{-}}
\NormalTok{  Base\_Trajipaq\_Finale\_Ok }\SpecialCharTok{\%\textgreater{}\%}
  \FunctionTok{rowwise}\NormalTok{() }\SpecialCharTok{\%\textgreater{}\%}
  \FunctionTok{mutate}\NormalTok{(}\AttributeTok{I2 =}\NormalTok{ (}\FunctionTok{sum}\NormalTok{(}\FunctionTok{c\_across}\NormalTok{(}\FunctionTok{all\_of}\NormalTok{(v\_I2)), }\AttributeTok{na.rm =} \ConstantTok{TRUE}\NormalTok{) }\SpecialCharTok{{-}} \FunctionTok{length}\NormalTok{(v\_I2)) }\SpecialCharTok{/}\NormalTok{ (}\FloatTok{2.25}\SpecialCharTok{*}\FunctionTok{length}\NormalTok{(v\_I2))) }\SpecialCharTok{\%\textgreater{}\%}
  \FunctionTok{mutate}\NormalTok{(}\AttributeTok{I2 =} \ControlFlowTok{if}\NormalTok{ (I2 }\SpecialCharTok{\textless{}} \DecValTok{0}\NormalTok{) }\ConstantTok{NA\_real\_} \ControlFlowTok{else}\NormalTok{ I2) }\SpecialCharTok{\%\textgreater{}\%}
  \FunctionTok{ungroup}\NormalTok{()}
\end{Highlighting}
\end{Shaded}

\begin{Shaded}
\begin{Highlighting}[]
\FunctionTok{descr}\NormalTok{(Base\_Trajipaq\_Finale\_Ok}\SpecialCharTok{$}\NormalTok{I2)}
\end{Highlighting}
\end{Shaded}

\begin{verbatim}
## Descriptive Statistics  
## Base_Trajipaq_Finale_Ok$I2  
## N: 1563  
## 
##                          I2
## ----------------- ---------
##              Mean      0.27
##           Std.Dev      0.17
##               Min      0.00
##                Q1      0.11
##            Median      0.22
##                Q3      0.33
##               Max      1.00
##               MAD      0.16
##               IQR      0.22
##                CV      0.64
##          Skewness      0.94
##       SE.Skewness      0.06
##          Kurtosis      2.17
##           N.Valid   1440.00
##         Pct.Valid     92.13
\end{verbatim}

\begin{Shaded}
\begin{Highlighting}[]
\FunctionTok{ggplot}\NormalTok{(}\AttributeTok{data =}\NormalTok{ Base\_Trajipaq\_Finale\_Ok, }\AttributeTok{mapping =} \FunctionTok{aes}\NormalTok{(}\AttributeTok{x =}\NormalTok{ I2)) }\SpecialCharTok{+}
  \FunctionTok{geom\_histogram}\NormalTok{(}\AttributeTok{na.rm =} \ConstantTok{TRUE}\NormalTok{, }\AttributeTok{fill =} \StringTok{"pink"}\NormalTok{, }\AttributeTok{binwidth =} \FloatTok{0.01}\NormalTok{) }\SpecialCharTok{+}
  \FunctionTok{theme\_bw}\NormalTok{()}
\end{Highlighting}
\end{Shaded}

\includegraphics{Article---Sentiment-d-appartenance_files/figure-latex/unnamed-chunk-82-1.pdf}

La distribution est asymétrique vers la droite. L'indice moyen de
discrimination subie est 0.28.

\begin{Shaded}
\begin{Highlighting}[]
\CommentTok{\# Lorsque l\textquotesingle{}on regarde la distribution de chaque question composant l\textquotesingle{}indice, le mode correspond à l\textquotesingle{}absence de discrimination (1). Or, l\textquotesingle{}indice présente une moyenne non nulle (0.28 \textgreater{} 0), suggérant que peu de répondants ne subissent aucune discrimination.}
\end{Highlighting}
\end{Shaded}

\hypertarget{analyse}{%
\section{Analyse}\label{analyse}}

\hypertarget{moduxe8le-de-ruxe9gression}{%
\subsection{Modèle de régression}\label{moduxe8le-de-ruxe9gression}}

\begin{Shaded}
\begin{Highlighting}[]
\NormalTok{reg\_1 }\OtherTok{\textless{}{-}} \FunctionTok{lm}\NormalTok{(Base\_Trajipaq\_Finale\_Ok}\SpecialCharTok{$}\NormalTok{I2 }\SpecialCharTok{\textasciitilde{}}\NormalTok{ Base\_Trajipaq\_Finale\_Ok}\SpecialCharTok{$}\NormalTok{I1 }\SpecialCharTok{+}
\NormalTok{              Base\_Trajipaq\_Finale\_Ok}\SpecialCharTok{$}\NormalTok{Q10 }\SpecialCharTok{+}
\NormalTok{              Base\_Trajipaq\_Finale\_Ok}\SpecialCharTok{$}\NormalTok{Q8B }\SpecialCharTok{+}
\NormalTok{              Base\_Trajipaq\_Finale\_Ok}\SpecialCharTok{$}\NormalTok{Q4 }\SpecialCharTok{+}
\NormalTok{              Base\_Trajipaq\_Finale\_Ok}\SpecialCharTok{$}\NormalTok{Q16 }\SpecialCharTok{+}
\NormalTok{              Base\_Trajipaq\_Finale\_Ok}\SpecialCharTok{$}\NormalTok{Q17 }\SpecialCharTok{+}
\NormalTok{              Base\_Trajipaq\_Finale\_Ok}\SpecialCharTok{$}\NormalTok{Q20 }\SpecialCharTok{+}
\NormalTok{              Base\_Trajipaq\_Finale\_Ok}\SpecialCharTok{$}\NormalTok{Q23}
\NormalTok{            )}
\FunctionTok{summary}\NormalTok{(reg\_1)}
\end{Highlighting}
\end{Shaded}

\begin{verbatim}
## 
## Call:
## lm(formula = Base_Trajipaq_Finale_Ok$I2 ~ Base_Trajipaq_Finale_Ok$I1 + 
##     Base_Trajipaq_Finale_Ok$Q10 + Base_Trajipaq_Finale_Ok$Q8B + 
##     Base_Trajipaq_Finale_Ok$Q4 + Base_Trajipaq_Finale_Ok$Q16 + 
##     Base_Trajipaq_Finale_Ok$Q17 + Base_Trajipaq_Finale_Ok$Q20 + 
##     Base_Trajipaq_Finale_Ok$Q23)
## 
## Residuals:
##      Min       1Q   Median       3Q      Max 
## -0.26036 -0.10944  0.00757  0.06785  0.62439 
## 
## Coefficients:
##                                   Estimate Std. Error t value Pr(>|t|)    
## (Intercept)                       0.112370   0.290556   0.387    0.701    
## Base_Trajipaq_Finale_Ok$I1        0.665029   0.129454   5.137 6.81e-06 ***
## Base_Trajipaq_Finale_Ok$Q10       0.044990   0.058790   0.765    0.448    
## Base_Trajipaq_Finale_Ok$Q8B      -0.006314   0.015015  -0.421    0.676    
## Base_Trajipaq_Finale_Ok$Q4       -0.073409   0.075565  -0.971    0.337    
## Base_Trajipaq_Finale_Ok$Q16       0.020841   0.021606   0.965    0.340    
## Base_Trajipaq_Finale_Ok$Q17      -0.004538   0.007530  -0.603    0.550    
## Base_Trajipaq_Finale_Ok$Q2001 02  0.021275   0.089699   0.237    0.814    
## Base_Trajipaq_Finale_Ok$Q2001 97  0.042064   0.143345   0.293    0.771    
## Base_Trajipaq_Finale_Ok$Q2002     0.098238   0.090287   1.088    0.283    
## Base_Trajipaq_Finale_Ok$Q2002 97  0.010253   0.132639   0.077    0.939    
## Base_Trajipaq_Finale_Ok$Q2097    -0.017889   0.058807  -0.304    0.762    
## Base_Trajipaq_Finale_Ok$Q2099     0.112975   0.106832   1.057    0.296    
## Base_Trajipaq_Finale_Ok$Q23      -0.052585   0.044384  -1.185    0.243    
## ---
## Signif. codes:  0 '***' 0.001 '**' 0.01 '*' 0.05 '.' 0.1 ' ' 1
## 
## Residual standard error: 0.1709 on 42 degrees of freedom
##   (1507 observations effacées parce que manquantes)
## Multiple R-squared:  0.4171, Adjusted R-squared:  0.2367 
## F-statistic: 2.312 on 13 and 42 DF,  p-value: 0.02026
\end{verbatim}

\hypertarget{coefficient-de-corruxe9lation}{%
\subsection{Coefficient de
corrélation}\label{coefficient-de-corruxe9lation}}

\begin{Shaded}
\begin{Highlighting}[]
\FunctionTok{cor.test}\NormalTok{(Base\_Trajipaq\_Finale\_Ok}\SpecialCharTok{$}\NormalTok{I2, Base\_Trajipaq\_Finale\_Ok}\SpecialCharTok{$}\NormalTok{I1)}
\end{Highlighting}
\end{Shaded}

\begin{verbatim}
## 
##  Pearson's product-moment correlation
## 
## data:  Base_Trajipaq_Finale_Ok$I2 and Base_Trajipaq_Finale_Ok$I1
## t = 17.57, df = 1436, p-value < 2.2e-16
## alternative hypothesis: true correlation is not equal to 0
## 95 percent confidence interval:
##  0.3771562 0.4622894
## sample estimates:
##       cor 
## 0.4206484
\end{verbatim}

\hypertarget{visualisation}{%
\subsection{Visualisation}\label{visualisation}}

\begin{Shaded}
\begin{Highlighting}[]
\NormalTok{vis }\OtherTok{\textless{}{-}} \FunctionTok{ggplot}\NormalTok{(Base\_Trajipaq\_Finale\_Ok }\SpecialCharTok{\%\textgreater{}\%}
                \FunctionTok{filter}\NormalTok{(}\SpecialCharTok{!}\FunctionTok{is.na}\NormalTok{(I1), }\SpecialCharTok{!}\FunctionTok{is.na}\NormalTok{(I2)), }\FunctionTok{aes}\NormalTok{(}\AttributeTok{x =}\NormalTok{ I2, }\AttributeTok{y =}\NormalTok{ I1)) }\SpecialCharTok{+}
  \FunctionTok{geom\_jitter}\NormalTok{(}\AttributeTok{color =} \StringTok{"pink"}\NormalTok{) }\SpecialCharTok{+}
  \FunctionTok{stat\_smooth}\NormalTok{(}\AttributeTok{color =} \StringTok{"purple"}\NormalTok{,}\AttributeTok{se =} \ConstantTok{FALSE}\NormalTok{) }\SpecialCharTok{+}
  \FunctionTok{labs}\NormalTok{(}\AttributeTok{title =} \StringTok{"Discrimination et sentiment d\textquotesingle{}appartenance"}\NormalTok{, }
       \AttributeTok{x =} \StringTok{"Discrimination subie au cours des 12 derniers mois"}\NormalTok{, }
       \AttributeTok{y =} \StringTok{"Sentiment d\textquotesingle{}appartenance au Quebec"}\NormalTok{) }\SpecialCharTok{+}
  \FunctionTok{theme\_bw}\NormalTok{()}
\FunctionTok{print}\NormalTok{(vis)}
\end{Highlighting}
\end{Shaded}

\begin{verbatim}
## `geom_smooth()` using method = 'gam' and formula = 'y ~ s(x, bs = "cs")'
\end{verbatim}

\includegraphics{Article---Sentiment-d-appartenance_files/figure-latex/unnamed-chunk-86-1.pdf}

\hypertarget{visualisation-contruxf4luxe9e-sexe}{%
\subsection{Visualisation contrôlée :
sexe}\label{visualisation-contruxf4luxe9e-sexe}}

\begin{Shaded}
\begin{Highlighting}[]
\NormalTok{vis\_sex }\OtherTok{\textless{}{-}} \FunctionTok{ggplot}\NormalTok{(Base\_Trajipaq\_Finale\_Ok }\SpecialCharTok{\%\textgreater{}\%}
                \FunctionTok{filter}\NormalTok{(}\SpecialCharTok{!}\FunctionTok{is.na}\NormalTok{(I1), }\SpecialCharTok{!}\FunctionTok{is.na}\NormalTok{(I2), }\SpecialCharTok{!}\FunctionTok{is.na}\NormalTok{(Q10)), }\FunctionTok{aes}\NormalTok{(}\AttributeTok{x =}\NormalTok{ I2, }\AttributeTok{y =}\NormalTok{ I1, }\AttributeTok{color =} \FunctionTok{as.factor}\NormalTok{(Q10))) }\SpecialCharTok{+}
  \FunctionTok{geom\_jitter}\NormalTok{(}\AttributeTok{color =} \StringTok{"pink"}\NormalTok{) }\SpecialCharTok{+}
  \FunctionTok{stat\_smooth}\NormalTok{(}\AttributeTok{se =} \ConstantTok{FALSE}\NormalTok{) }\SpecialCharTok{+}
  \FunctionTok{scale\_color\_brewer}\NormalTok{(}\AttributeTok{name =} \StringTok{"Sexe"}\NormalTok{, }\AttributeTok{palette =} \DecValTok{3}\NormalTok{) }\SpecialCharTok{+}
  \FunctionTok{labs}\NormalTok{(}\AttributeTok{title =} \StringTok{"Discrimination et sentiment d\textquotesingle{}appartenance"}\NormalTok{, }
       \AttributeTok{x =} \StringTok{"Discrimination subie au cours des 12 derniers mois"}\NormalTok{, }
       \AttributeTok{y =} \StringTok{"Sentiment d\textquotesingle{}appartenance au Quebec"}\NormalTok{) }\SpecialCharTok{+}
  \FunctionTok{theme\_bw}\NormalTok{()}
\FunctionTok{print}\NormalTok{(vis\_sex)}
\end{Highlighting}
\end{Shaded}

\begin{verbatim}
## `geom_smooth()` using method = 'loess' and formula = 'y ~ x'
\end{verbatim}

\includegraphics{Article---Sentiment-d-appartenance_files/figure-latex/unnamed-chunk-87-1.pdf}

\hypertarget{visualisation-contruxf4luxe9e-uxe2ge}{%
\subsection{Visualisation contrôlée :
âge}\label{visualisation-contruxf4luxe9e-uxe2ge}}

\begin{Shaded}
\begin{Highlighting}[]
\NormalTok{Base\_Trajipaq\_Finale\_Ok }\OtherTok{\textless{}{-}}
\NormalTok{  Base\_Trajipaq\_Finale\_Ok }\SpecialCharTok{\%\textgreater{}\%}
  \FunctionTok{mutate}\NormalTok{(}\AttributeTok{Q8A\_F =} \FunctionTok{case\_when}\NormalTok{(}
\NormalTok{    Q8A }\SpecialCharTok{\textless{}=} \DecValTok{1963} \SpecialCharTok{\textasciitilde{}} \StringTok{"\textgreater{}=60"}\NormalTok{,}
\NormalTok{    Q8A }\SpecialCharTok{\textgreater{}=} \DecValTok{1963} \SpecialCharTok{\&}\NormalTok{ Q8A }\SpecialCharTok{\textless{}} \DecValTok{1968} \SpecialCharTok{\textasciitilde{}} \StringTok{"[59;55]"}\NormalTok{,}
\NormalTok{    Q8A }\SpecialCharTok{\textgreater{}=} \DecValTok{1968} \SpecialCharTok{\&}\NormalTok{ Q8A }\SpecialCharTok{\textless{}} \DecValTok{1973} \SpecialCharTok{\textasciitilde{}} \StringTok{"[54;50]"}\NormalTok{,}
\NormalTok{    Q8A }\SpecialCharTok{\textgreater{}=} \DecValTok{1973} \SpecialCharTok{\&}\NormalTok{ Q8A }\SpecialCharTok{\textless{}} \DecValTok{1978} \SpecialCharTok{\textasciitilde{}} \StringTok{"[49;45]"}\NormalTok{,}
\NormalTok{    Q8A }\SpecialCharTok{\textgreater{}=} \DecValTok{1978} \SpecialCharTok{\&}\NormalTok{ Q8A }\SpecialCharTok{\textless{}} \DecValTok{1983} \SpecialCharTok{\textasciitilde{}} \StringTok{"[44;35]"}\NormalTok{,}
\NormalTok{    Q8A }\SpecialCharTok{\textgreater{}=} \DecValTok{1983} \SpecialCharTok{\&}\NormalTok{ Q8A }\SpecialCharTok{\textless{}} \DecValTok{1988} \SpecialCharTok{\textasciitilde{}} \StringTok{"[34;30]"}\NormalTok{,}
\NormalTok{    Q8A }\SpecialCharTok{\textgreater{}=} \DecValTok{1988} \SpecialCharTok{\&}\NormalTok{ Q8A }\SpecialCharTok{\textless{}} \DecValTok{1993} \SpecialCharTok{\textasciitilde{}} \StringTok{"[29;25]"}\NormalTok{,}
\NormalTok{    Q8A }\SpecialCharTok{\textgreater{}=} \DecValTok{1993} \SpecialCharTok{\&}\NormalTok{ Q8A }\SpecialCharTok{\textless{}} \DecValTok{1998} \SpecialCharTok{\textasciitilde{}} \StringTok{"[24;20]"}\NormalTok{,}
\NormalTok{    Q8A }\SpecialCharTok{\textgreater{}=} \DecValTok{1998} \SpecialCharTok{\&}\NormalTok{ Q8A }\SpecialCharTok{\textless{}} \DecValTok{2023} \SpecialCharTok{\textasciitilde{}} \StringTok{"\textless{}19"}
\NormalTok{  ))}
\end{Highlighting}
\end{Shaded}

\begin{Shaded}
\begin{Highlighting}[]
\NormalTok{vis\_age }\OtherTok{\textless{}{-}} \FunctionTok{ggplot}\NormalTok{(Base\_Trajipaq\_Finale\_Ok }\SpecialCharTok{\%\textgreater{}\%}
                \FunctionTok{filter}\NormalTok{(}\SpecialCharTok{!}\FunctionTok{is.na}\NormalTok{(I1), }\SpecialCharTok{!}\FunctionTok{is.na}\NormalTok{(I2), }\SpecialCharTok{!}\FunctionTok{is.na}\NormalTok{(Q8A\_F)), }\FunctionTok{aes}\NormalTok{(}\AttributeTok{x =}\NormalTok{ I2, }\AttributeTok{y =}\NormalTok{ I1, }\AttributeTok{color =}\NormalTok{ Q8A\_F)) }\SpecialCharTok{+}
  \FunctionTok{geom\_jitter}\NormalTok{(}\AttributeTok{color =} \StringTok{"pink"}\NormalTok{) }\SpecialCharTok{+}
  \FunctionTok{stat\_smooth}\NormalTok{(}\AttributeTok{se =} \ConstantTok{FALSE}\NormalTok{, }\AttributeTok{span =} \FloatTok{0.9}\NormalTok{) }\SpecialCharTok{+}
  \FunctionTok{scale\_color\_brewer}\NormalTok{(}\AttributeTok{name =} \StringTok{"Age"}\NormalTok{, }\AttributeTok{palette =} \DecValTok{3}\NormalTok{) }\SpecialCharTok{+}
  \FunctionTok{labs}\NormalTok{(}\AttributeTok{title =} \StringTok{"Discrimination et sentiment d\textquotesingle{}appartenance"}\NormalTok{, }
       \AttributeTok{x =} \StringTok{"Discrimination subie au cours des 12 derniers mois"}\NormalTok{, }
       \AttributeTok{y =} \StringTok{"Sentiment d\textquotesingle{}appartenance au Quebec"}\NormalTok{,}
\NormalTok{       ) }\SpecialCharTok{+}
  \FunctionTok{theme\_bw}\NormalTok{()}
\FunctionTok{print}\NormalTok{(vis\_age)}
\end{Highlighting}
\end{Shaded}

\begin{verbatim}
## `geom_smooth()` using method = 'loess' and formula = 'y ~ x'
\end{verbatim}

\includegraphics{Article---Sentiment-d-appartenance_files/figure-latex/unnamed-chunk-89-1.pdf}

\hypertarget{visualisation-contruxf4luxe9e-statut-matrimonial}{%
\subsection{Visualisation contrôlée : statut
matrimonial}\label{visualisation-contruxf4luxe9e-statut-matrimonial}}

\begin{Shaded}
\begin{Highlighting}[]
\NormalTok{Base\_Trajipaq\_Finale\_Ok }\OtherTok{\textless{}{-}}
\NormalTok{  Base\_Trajipaq\_Finale\_Ok }\SpecialCharTok{\%\textgreater{}\%}
  \FunctionTok{mutate}\NormalTok{(}\AttributeTok{STATUT\_F =} \FunctionTok{case\_when}\NormalTok{(}
\NormalTok{    STATUT }\SpecialCharTok{==} \DecValTok{1} \SpecialCharTok{\textasciitilde{}} \StringTok{"Celibataire"}\NormalTok{,}
\NormalTok{    STATUT }\SpecialCharTok{==} \DecValTok{2} \SpecialCharTok{\textasciitilde{}} \StringTok{"Marie(e)"}\NormalTok{,}
\NormalTok{    STATUT }\SpecialCharTok{==} \DecValTok{3} \SpecialCharTok{\textasciitilde{}} \StringTok{"Conjoint(e) de fait"}\NormalTok{,}
\NormalTok{    STATUT }\SpecialCharTok{==} \DecValTok{4} \SpecialCharTok{\textasciitilde{}} \StringTok{"Separe(e)"}\NormalTok{,}
\NormalTok{    STATUT }\SpecialCharTok{==} \DecValTok{5} \SpecialCharTok{\textasciitilde{}} \StringTok{"Divorce(e)"}\NormalTok{,}
\NormalTok{    STATUT }\SpecialCharTok{==} \DecValTok{6} \SpecialCharTok{\textasciitilde{}} \ConstantTok{NA}
\NormalTok{  ))}
\end{Highlighting}
\end{Shaded}

\begin{Shaded}
\begin{Highlighting}[]
\NormalTok{vis\_born }\OtherTok{\textless{}{-}} \FunctionTok{ggplot}\NormalTok{(Base\_Trajipaq\_Finale\_Ok }\SpecialCharTok{\%\textgreater{}\%}
                \FunctionTok{filter}\NormalTok{(}\SpecialCharTok{!}\FunctionTok{is.na}\NormalTok{(I1), }\SpecialCharTok{!}\FunctionTok{is.na}\NormalTok{(I2), }\SpecialCharTok{!}\FunctionTok{is.na}\NormalTok{(STATUT\_F)), }\FunctionTok{aes}\NormalTok{(}\AttributeTok{x =}\NormalTok{ I2, }\AttributeTok{y =}\NormalTok{ I1, }\AttributeTok{color =}\NormalTok{ STATUT\_F)) }\SpecialCharTok{+}
  \FunctionTok{geom\_jitter}\NormalTok{(}\AttributeTok{color =} \StringTok{"pink"}\NormalTok{) }\SpecialCharTok{+}
  \FunctionTok{stat\_smooth}\NormalTok{(}\AttributeTok{se =} \ConstantTok{FALSE}\NormalTok{, }\AttributeTok{span =} \FloatTok{0.9}\NormalTok{) }\SpecialCharTok{+}
  \FunctionTok{scale\_color\_brewer}\NormalTok{(}\AttributeTok{name =} \StringTok{"Statut matrimonial"}\NormalTok{, }\AttributeTok{palette =} \DecValTok{3}\NormalTok{) }\SpecialCharTok{+}
  \FunctionTok{labs}\NormalTok{(}\AttributeTok{title =} \StringTok{"Discrimination et sentiment d\textquotesingle{}appartenance"}\NormalTok{, }
       \AttributeTok{x =} \StringTok{"Discrimination subie au cours des 12 derniers mois"}\NormalTok{, }
       \AttributeTok{y =} \StringTok{"Sentiment d\textquotesingle{}appartenance au Quebec"}\NormalTok{) }\SpecialCharTok{+}
  \FunctionTok{theme\_bw}\NormalTok{()}
\FunctionTok{print}\NormalTok{(vis\_born)}
\end{Highlighting}
\end{Shaded}

\begin{verbatim}
## `geom_smooth()` using method = 'loess' and formula = 'y ~ x'
\end{verbatim}

\includegraphics{Article---Sentiment-d-appartenance_files/figure-latex/unnamed-chunk-91-1.pdf}

\hypertarget{visualisation-contruxf4luxe9e-statut-migratoire}{%
\subsection{Visualisation contrôlée : statut
migratoire}\label{visualisation-contruxf4luxe9e-statut-migratoire}}

\begin{Shaded}
\begin{Highlighting}[]
\NormalTok{vis\_born }\OtherTok{\textless{}{-}} \FunctionTok{ggplot}\NormalTok{(Base\_Trajipaq\_Finale\_Ok }\SpecialCharTok{\%\textgreater{}\%}
                \FunctionTok{filter}\NormalTok{(}\SpecialCharTok{!}\FunctionTok{is.na}\NormalTok{(I1), }\SpecialCharTok{!}\FunctionTok{is.na}\NormalTok{(I2), }\SpecialCharTok{!}\FunctionTok{is.na}\NormalTok{(Q4)), }\FunctionTok{aes}\NormalTok{(}\AttributeTok{x =}\NormalTok{ I2, }\AttributeTok{y =}\NormalTok{ I1, }\AttributeTok{color =} \FunctionTok{as.factor}\NormalTok{(Q4))) }\SpecialCharTok{+}
  \FunctionTok{geom\_jitter}\NormalTok{(}\AttributeTok{color =} \StringTok{"pink"}\NormalTok{) }\SpecialCharTok{+}
  \FunctionTok{stat\_smooth}\NormalTok{(}\AttributeTok{se =} \ConstantTok{FALSE}\NormalTok{, }\AttributeTok{span =} \FloatTok{0.9}\NormalTok{) }\SpecialCharTok{+}
  \FunctionTok{scale\_color\_brewer}\NormalTok{(}\AttributeTok{name =} \StringTok{"Ne au Canada"}\NormalTok{, }\AttributeTok{palette =} \DecValTok{3}\NormalTok{) }\SpecialCharTok{+}
  \FunctionTok{labs}\NormalTok{(}\AttributeTok{title =} \StringTok{"Discrimination et sentiment d\textquotesingle{}appartenance"}\NormalTok{, }
       \AttributeTok{x =} \StringTok{"Discrimination subie au cours des 12 derniers mois"}\NormalTok{, }
       \AttributeTok{y =} \StringTok{"Sentiment d\textquotesingle{}appartenance au Quebec"}\NormalTok{) }\SpecialCharTok{+}
  \FunctionTok{theme\_bw}\NormalTok{()}
\FunctionTok{print}\NormalTok{(vis\_born)}
\end{Highlighting}
\end{Shaded}

\begin{verbatim}
## `geom_smooth()` using method = 'loess' and formula = 'y ~ x'
\end{verbatim}

\includegraphics{Article---Sentiment-d-appartenance_files/figure-latex/unnamed-chunk-92-1.pdf}

\hypertarget{visualisation-contruxf4luxe9e-niveau-duxe9ducation}{%
\subsection{Visualisation contrôlée : niveau
d'éducation}\label{visualisation-contruxf4luxe9e-niveau-duxe9ducation}}

\begin{Shaded}
\begin{Highlighting}[]
\NormalTok{vis\_educ }\OtherTok{\textless{}{-}} \FunctionTok{ggplot}\NormalTok{(Base\_Trajipaq\_Finale\_Ok }\SpecialCharTok{\%\textgreater{}\%}
                \FunctionTok{filter}\NormalTok{(}\SpecialCharTok{!}\FunctionTok{is.na}\NormalTok{(I1), }\SpecialCharTok{!}\FunctionTok{is.na}\NormalTok{(I2), }\SpecialCharTok{!}\FunctionTok{is.na}\NormalTok{(Q16)), }\FunctionTok{aes}\NormalTok{(}\AttributeTok{x =}\NormalTok{ I2, }\AttributeTok{y =}\NormalTok{ I1, }\AttributeTok{color =} \FunctionTok{as.factor}\NormalTok{(Q16))) }\SpecialCharTok{+}
  \FunctionTok{geom\_jitter}\NormalTok{(}\AttributeTok{color =} \StringTok{"pink"}\NormalTok{) }\SpecialCharTok{+}
  \FunctionTok{stat\_smooth}\NormalTok{(}\AttributeTok{se =} \ConstantTok{FALSE}\NormalTok{, }\AttributeTok{span =} \DecValTok{1}\NormalTok{) }\SpecialCharTok{+}
  \FunctionTok{scale\_color\_brewer}\NormalTok{(}\AttributeTok{name =} \StringTok{"Niveau d\textquotesingle{}education"}\NormalTok{, }\AttributeTok{palette =} \DecValTok{3}\NormalTok{) }\SpecialCharTok{+}
  \FunctionTok{labs}\NormalTok{(}\AttributeTok{title =} \StringTok{"Discrimination et sentiment d\textquotesingle{}appartenance"}\NormalTok{, }
       \AttributeTok{x =} \StringTok{"Discrimination subie au cours des 12 derniers mois"}\NormalTok{, }
       \AttributeTok{y =} \StringTok{"Sentiment d\textquotesingle{}appartenance au Quebec"}\NormalTok{) }\SpecialCharTok{+}
  \FunctionTok{theme\_bw}\NormalTok{()}
\FunctionTok{print}\NormalTok{(vis\_educ)}
\end{Highlighting}
\end{Shaded}

\begin{verbatim}
## `geom_smooth()` using method = 'loess' and formula = 'y ~ x'
\end{verbatim}

\includegraphics{Article---Sentiment-d-appartenance_files/figure-latex/unnamed-chunk-93-1.pdf}

\hypertarget{visualisation-contruxf4luxe9e-religion}{%
\subsection{Visualisation contrôlée :
religion}\label{visualisation-contruxf4luxe9e-religion}}

\begin{Shaded}
\begin{Highlighting}[]
\NormalTok{Base\_Trajipaq\_Finale\_Ok }\OtherTok{\textless{}{-}}
\NormalTok{  Base\_Trajipaq\_Finale\_Ok }\SpecialCharTok{\%\textgreater{}\%}
  \FunctionTok{mutate}\NormalTok{(}\AttributeTok{Q17\_F =} \FunctionTok{case\_when}\NormalTok{(}
\NormalTok{    Q17 }\SpecialCharTok{==} \DecValTok{0} \SpecialCharTok{\textasciitilde{}} \StringTok{"Sans religion"}\NormalTok{,}
\NormalTok{    Q17 }\SpecialCharTok{==} \DecValTok{1} \SpecialCharTok{\textasciitilde{}} \StringTok{"Catholique"}\NormalTok{,}
\NormalTok{    Q17 }\SpecialCharTok{==} \DecValTok{8} \SpecialCharTok{\textasciitilde{}} \StringTok{"Musulman"}\NormalTok{,}
\NormalTok{    Q17 }\SpecialCharTok{!=} \DecValTok{0} \SpecialCharTok{\&}\NormalTok{ Q17 }\SpecialCharTok{!=} \DecValTok{1} \SpecialCharTok{\&}\NormalTok{ Q17 }\SpecialCharTok{!=} \DecValTok{8} \SpecialCharTok{\textasciitilde{}} \StringTok{"Autre"}
\NormalTok{  ))}
\end{Highlighting}
\end{Shaded}

\begin{Shaded}
\begin{Highlighting}[]
\NormalTok{vis\_relig }\OtherTok{\textless{}{-}} \FunctionTok{ggplot}\NormalTok{(Base\_Trajipaq\_Finale\_Ok }\SpecialCharTok{\%\textgreater{}\%}
                \FunctionTok{filter}\NormalTok{(}\SpecialCharTok{!}\FunctionTok{is.na}\NormalTok{(I1), }\SpecialCharTok{!}\FunctionTok{is.na}\NormalTok{(I2), }\SpecialCharTok{!}\FunctionTok{is.na}\NormalTok{(Q17\_F)), }\FunctionTok{aes}\NormalTok{(}\AttributeTok{x =}\NormalTok{ I2, }\AttributeTok{y =}\NormalTok{ I1, }\AttributeTok{color =}\NormalTok{ Q17\_F)) }\SpecialCharTok{+}
  \FunctionTok{geom\_jitter}\NormalTok{(}\AttributeTok{color =} \StringTok{"pink"}\NormalTok{) }\SpecialCharTok{+}
  \FunctionTok{stat\_smooth}\NormalTok{(}\AttributeTok{se =} \ConstantTok{FALSE}\NormalTok{, }\AttributeTok{span =} \FloatTok{0.9}\NormalTok{) }\SpecialCharTok{+}
  \FunctionTok{scale\_color\_brewer}\NormalTok{(}\AttributeTok{name =} \StringTok{"Religion"}\NormalTok{, }\AttributeTok{palette =} \DecValTok{3}\NormalTok{) }\SpecialCharTok{+}
  \FunctionTok{labs}\NormalTok{(}\AttributeTok{title =} \StringTok{"Discrimination et sentiment d\textquotesingle{}appartenance"}\NormalTok{, }
       \AttributeTok{x =} \StringTok{"Discrimination subie au cours des 12 derniers mois"}\NormalTok{, }
       \AttributeTok{y =} \StringTok{"Sentiment d\textquotesingle{}appartenance au Quebec"}\NormalTok{) }\SpecialCharTok{+}
  \FunctionTok{theme\_bw}\NormalTok{()}
\FunctionTok{print}\NormalTok{(vis\_relig)}
\end{Highlighting}
\end{Shaded}

\begin{verbatim}
## `geom_smooth()` using method = 'loess' and formula = 'y ~ x'
\end{verbatim}

\includegraphics{Article---Sentiment-d-appartenance_files/figure-latex/unnamed-chunk-95-1.pdf}

\hypertarget{visualisation-contruxf4luxe9e-langue-maternelle}{%
\subsection{Visualisation contrôlée : langue
maternelle}\label{visualisation-contruxf4luxe9e-langue-maternelle}}

\begin{Shaded}
\begin{Highlighting}[]
\NormalTok{Base\_Trajipaq\_Finale\_Ok }\OtherTok{\textless{}{-}}
\NormalTok{  Base\_Trajipaq\_Finale\_Ok }\SpecialCharTok{\%\textgreater{}\%}
  \FunctionTok{mutate}\NormalTok{(}\AttributeTok{Q20\_F =} \FunctionTok{case\_when}\NormalTok{(}
\NormalTok{    Q20M1 }\SpecialCharTok{==} \DecValTok{1} \SpecialCharTok{\textasciitilde{}} \StringTok{"Francais"}\NormalTok{,}
\NormalTok{    Q20M1 }\SpecialCharTok{==} \DecValTok{2} \SpecialCharTok{\textasciitilde{}} \StringTok{"Anglais"}\NormalTok{,}
\NormalTok{    Q20M1 }\SpecialCharTok{==} \DecValTok{3} \SpecialCharTok{\textasciitilde{}} \StringTok{"Autre"}
\NormalTok{  ))}
\end{Highlighting}
\end{Shaded}

\begin{Shaded}
\begin{Highlighting}[]
\NormalTok{vis\_lang }\OtherTok{\textless{}{-}} \FunctionTok{ggplot}\NormalTok{(Base\_Trajipaq\_Finale\_Ok }\SpecialCharTok{\%\textgreater{}\%}
                \FunctionTok{filter}\NormalTok{(}\SpecialCharTok{!}\FunctionTok{is.na}\NormalTok{(I1), }\SpecialCharTok{!}\FunctionTok{is.na}\NormalTok{(I2), }\SpecialCharTok{!}\FunctionTok{is.na}\NormalTok{(Q20\_F)), }\FunctionTok{aes}\NormalTok{(}\AttributeTok{x =}\NormalTok{ I2, }\AttributeTok{y =}\NormalTok{ I1, }\AttributeTok{color =}\NormalTok{ Q20\_F)) }\SpecialCharTok{+}
  \FunctionTok{geom\_jitter}\NormalTok{(}\AttributeTok{color =} \StringTok{"pink"}\NormalTok{) }\SpecialCharTok{+}
  \FunctionTok{stat\_smooth}\NormalTok{(}\AttributeTok{se =} \ConstantTok{FALSE}\NormalTok{, }\AttributeTok{span =} \FloatTok{0.8}\NormalTok{) }\SpecialCharTok{+}
  \FunctionTok{scale\_color\_brewer}\NormalTok{(}\AttributeTok{name =} \StringTok{"Langue maternelle"}\NormalTok{, }\AttributeTok{palette =} \DecValTok{3}\NormalTok{) }\SpecialCharTok{+}
  \FunctionTok{labs}\NormalTok{(}\AttributeTok{title =} \StringTok{"Discrimination et sentiment d\textquotesingle{}appartenance"}\NormalTok{, }
       \AttributeTok{x =} \StringTok{"Discrimination subie au cours des 12 derniers mois"}\NormalTok{, }
       \AttributeTok{y =} \StringTok{"Sentiment d\textquotesingle{}appartenance au Quebec"}\NormalTok{) }\SpecialCharTok{+}
  \FunctionTok{theme\_bw}\NormalTok{()}
\FunctionTok{print}\NormalTok{(vis\_lang)}
\end{Highlighting}
\end{Shaded}

\begin{verbatim}
## `geom_smooth()` using method = 'loess' and formula = 'y ~ x'
\end{verbatim}

\includegraphics{Article---Sentiment-d-appartenance_files/figure-latex/unnamed-chunk-97-1.pdf}

\hypertarget{visualisation-contruxf4luxe9e-uxe9tat-de-santuxe9}{%
\subsection{Visualisation contrôlée : état de
santé}\label{visualisation-contruxf4luxe9e-uxe9tat-de-santuxe9}}

\begin{Shaded}
\begin{Highlighting}[]
\NormalTok{vis\_health }\OtherTok{\textless{}{-}} \FunctionTok{ggplot}\NormalTok{(Base\_Trajipaq\_Finale\_Ok }\SpecialCharTok{\%\textgreater{}\%}
                \FunctionTok{filter}\NormalTok{(}\SpecialCharTok{!}\FunctionTok{is.na}\NormalTok{(I1), }\SpecialCharTok{!}\FunctionTok{is.na}\NormalTok{(I2), }\SpecialCharTok{!}\FunctionTok{is.na}\NormalTok{(Q23)), }\FunctionTok{aes}\NormalTok{(}\AttributeTok{x =}\NormalTok{ I2, }\AttributeTok{y =}\NormalTok{ I1, }\AttributeTok{color =} \FunctionTok{as.factor}\NormalTok{(Q23))) }\SpecialCharTok{+}
  \FunctionTok{geom\_jitter}\NormalTok{(}\AttributeTok{color =} \StringTok{"pink"}\NormalTok{) }\SpecialCharTok{+}
  \FunctionTok{stat\_smooth}\NormalTok{(}\AttributeTok{se =} \ConstantTok{FALSE}\NormalTok{, }\AttributeTok{span =} \FloatTok{1.1}\NormalTok{) }\SpecialCharTok{+}
  \FunctionTok{scale\_color\_brewer}\NormalTok{(}\AttributeTok{name =} \StringTok{"Etat de sante"}\NormalTok{, }\AttributeTok{palette =} \DecValTok{3}\NormalTok{) }\SpecialCharTok{+}
  \FunctionTok{labs}\NormalTok{(}\AttributeTok{title =} \StringTok{"Discrimination et sentiment d\textquotesingle{}appartenance"}\NormalTok{, }
       \AttributeTok{x =} \StringTok{"Discrimination subie au cours des 12 derniers mois"}\NormalTok{, }
       \AttributeTok{y =} \StringTok{"Sentiment d\textquotesingle{}appartenance au Quebec"}\NormalTok{) }\SpecialCharTok{+}
  \FunctionTok{theme\_bw}\NormalTok{()}
\FunctionTok{print}\NormalTok{(vis\_health)}
\end{Highlighting}
\end{Shaded}

\begin{verbatim}
## `geom_smooth()` using method = 'loess' and formula = 'y ~ x'
\end{verbatim}

\includegraphics{Article---Sentiment-d-appartenance_files/figure-latex/unnamed-chunk-98-1.pdf}

\end{document}
